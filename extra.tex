\begin{proposition}\label{proposition:guerinisavo}
    Let $\Omega \subseteq \bbR^{n}$ be a convex bounded domain. Then 
    Moreover, 
    \begin{align*}
        C_{\rm{PF},\Omega,0,2} \geq C_{\rm{PF},\Omega,1,2} \geq \dots \geq C_{\rm{PF},\Omega,n-1,2}
    \end{align*}
\end{proposition}
\begin{proof}
    The result is true if $\Omega$ has a smooth boundary. 
    Now suppose that $\Omega$ is a Lipschitz domain. 
\end{proof}

\begin{proof}
    The desired results are known for smoothly bounded domains;
    see Theorem~2.6 of \cite{guerini2004eigenvalue} for % decay in k
    see \cite[Theorem~1.3]{guerini2004spectre} for
    \begin{align*}
        C \diam(\Omega)^{-2} 
        \leq 
        \lambda_{k},
        \qquad 
        C := 
        \left\{\begin{array}{ll}
            \frac{n-1}{n e^{3}} & \text{ if } k=0
            \\
            \frac{k(n-k)}{ne^{3}} & \text{ if } 1 \leq k \leq n-1
            \\
            \lambda_{1,0}(B) & \text{ if } k = n % Dirichlet eigenvalue over a ball
        \end{array}\right. 
        % \frac{\max_{k(n-k),n-1}}{ne^{3}} \diam(\Omega)^{-2}        
    \end{align*}
    see \cite{guerini2004eigenvalue} for $n \geq 3$, 
    \begin{align*}
        \frac{4}{n^{2} \tbinom{n}{k-1} \diam(\Omega)^{2} } \leq \lambda_{k}
    \end{align*}
    
    For each $0 \leq k \leq n$ we let $\mu_{k}$ be the smallest non-zero eigenvalue of the Hodge-Laplace operator acting on $k$-forms. For each $0 \leq k \leq n-1$, we let $\sigma_{k}$ be the smallest non-zero singular value of the exterior derivative $\cartan : L^{2}\Lambda^{k}(\Omega) \rightarrow L^{2}\Lambda^{k+1}(\Omega)$. We have the relation
    \begin{align*}
        \mu_{k} = \min( \sigma_{k}, \sigma_{k-1} )^{2}.
    \end{align*}
    We first consider the case where $\Omega$ is strictly convex with smooth boundary. 
    According to Theorem~2.6 of \cite{guerini2004eigenvalue},
    \begin{align*}
        \mu_{0} = \mu_{1} < \dots < \mu_{n}.
    \end{align*}
    By definition, we have $\mu_{n} = \sigma_{n-1}^{2}$. 
    Suppose that we know $\mu_{k} = \sigma_{k-1}^{2}$ for some $2 \leq k \leq n$. Then 
    \begin{align*}
        \sigma_{k-1}^{2} = \mu_{k} > \mu_{k-1} = \min( \sigma_{k-1}, \sigma_{k-2} )^{2}
    \end{align*}
    implies $\mu_{k-1} = \sigma_{k-2}^{2}$. 
    Iterating this, we find 
    \begin{align*}
        \sigma_{0} < \sigma_{1} < \dots < \sigma_{n-1}.
    \end{align*}
    We have shown this when $\Omega$ is strictly convex with smooth boundary. 
    
    We now address the case where $\Omega$ is merely convex. 
    According to Theorem~A of \cite{blocki1997smooth}, 
    there exists a continuous convex function $\beta : \overline\Omega \rightarrow \bbR$ that equals zero along $\partial\Omega$, which is negative over $\Omega$, and whose Hessian over $\Omega$ has positive determinant. 
    This already implies that the Hessian of $\beta$ over $\Omega$ is always positive-definite, and so $\beta : \Omega \rightarrow \bbR$ is strictly convex.
    
    For $\epsilon > 0$ small enough, the sublevel sets 
    \begin{align*}
        \Omega_{\epsilon} := \left\{ x \in \Omega \suchthat \beta(x) < -\epsilon \right\}
    \end{align*}
    are strictly convex smoothly bounded open sets compactly contained in $\Omega$. 
    For convenience, we also write $\Omega_{0} := \Omega$. 
    
    We first show that for each $\rho > 0$ there exists $\epsilon > 0$ such that $\Omega \setminus \Omega_{\epsilon} \subseteq \partial\Omega + B_{\rho}(0)$. 
    Indeed, if that were not the case, then there would exist $\rho > 0$ such that for any $\epsilon > 0$ we can find $x_{\epsilon}$ with $\beta(x_{\epsilon}) > -\epsilon$ but distance at least $\rho$ from $\partial\Omega$. We could then pick a subsequence converging to a point $x \in \Omega$ with $\beta(x) = 0$, which is a contradiction. 
    
    %%%%%%%%%%%%%%%%%%%%%%%%%
    % TODO: from here on 
    For $\rho > 0$ small enough, the mapping 
    \begin{align*}
        A : \partial\Omega \times [0,\rho] \rightarrow \bbR^{n}, \quad (x,s) \mapsto x - \frac{s}{|x|} x
    \end{align*}
    is bijective. It satisfies the bounds 
    \begin{align*}
        ..
    \end{align*}
    and 
    \begin{align*}
        ..
    \end{align*}
    % TODO:
    Consequently, there exists $L > 0$ such that for all $\epsilon > 0$ small enough 
    there exists $B_{\epsilon} : \Omega \rightarrow \Omega_{\epsilon}$ bi-Lipschitz with
    \begin{align*}
        \| \Id - B_{\epsilon} \|_{L^{\infty}(\Omega)} < L \epsilon,
        \quad 
        \| \Id - \Jacobian B_{\epsilon} \|_{L^{\infty}(\Omega)} < L \epsilon.
    \end{align*}
    %%%%%%%%%%%%%%%%%%%%%%%%%
    
    %%%%%%%%%%%%%%%%%%%%%%%%%
    For any $\epsilon \geq 0$, we let $X^{k}_{\epsilon} := H\Lambda^{k}(\Omega) \cap \ker\cartan$. 
    Notice that $B_{\epsilon}^{\ast} X^{k}_{\epsilon} = X^{k}_{0}$. 
    %
    Consider any $0 \leq k \leq n-1$. 
    Let $f \in H\Lambda^{k+1}(\Omega)$ with $\cartan f = 0$.
    There exists $u_\star \in H\Lambda^{k}(\Omega)$ with $\cartan u = f$ 
    because $\Omega$ is convex and thus contractible. % TODO: theorem 
    Notice that $B_{\epsilon}^{-\ast} u_0 \in H\Lambda^{k}(\Omega_{\epsilon})$ 
    with $\cartan B_{\epsilon}^{-\ast} u_\star = B_{\epsilon}^{-\ast} f$ over $\Omega_{\epsilon}$.
    % 
    \begin{align*}
        ..
    \end{align*}
    % TODO
    We conclude that $P^{k}_{\epsilon} \rightarrow P^{k}_{0}$ in the operator norm. 
    In particular, the operator norms converge. 
    %%%%%%%%%%%%%%%%%%%%%%%%%
    
    For any $0 \leq k \leq n-2$ and $\epsilon > 0$ we have $\| P^{k}_{\epsilon} \| < \| P^{k+1}_{\epsilon} \|$,
    so that $\| P^{k}_{0} \| \leq \| P^{k+1}_{0} \|$.
    This had to be shown.
%     
%     There exists a bi-Lipschitz mapping $\varphi : \overline{B_1(0)} \rightarrow \Omega$. 
%     We let $\varphi_{\eps} : \overline{B_1(0)} \rightarrow \bbR^{n}$ be smooth,
%     uniformly bounded, and pointwise converging to $\varphi$ in the norm of $W^{1,\infty}$. 
%     For $\epsilon > 0$ small enough, 
%     the range $\Omega_{\eps}$ of each $\varphi_{\eps}$ is a smoothly bounded domain,
%     and $\varphi_{\epsilon} : \overline{B_1(0)} \rightarrow \Omega_{\eps}$ is a diffeomorphism.
%     The diameters of $\Omega_{\eps}$ converge to the diameter of $\Omega$.
%     The desired claim now follows by the dominated convergence theorem % Given $u \in W^{2}\Lambda^{k}(\Omega)$
    % TODO: show that the trafo converges to identity.
    % Write the trafo as perturbation of the identity.
\end{proof}





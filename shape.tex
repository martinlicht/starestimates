\begin{lemma}\label{lemma:measurerelationships}
    Let $T$ be an $n$-simplex and let $\varphi : \Delta^{n} \rightarrow T$ be an affine diffeomorphism from the reference $n$-simplex. Then 
    \begin{align*} % First pair 
        \frac{\diam(T)}{\sqrt 2} 
        \leq 
        \matnorm{ \Jacobian \varphi } 
        \leq 
        \sqrt{n} \diam(T).
    \end{align*}
    \begin{align*} % Second pair
        \sqrt{n}^{-1} \height(T)^{-1} 
        \leq 
        \matnorm{ \Jacobian \varphi^{-1} } 
        \leq 
        \sqrt{n} \cdot \height(T)^{-1}
        .
    \end{align*}
    \begin{align*} % Third / fourth pair 
        \frac{ \height(T)    }{\sqrt{n}} \leq \matnorm{ A      } \leq \algebraicshapemeasure(A) \sqrt{n} \height(T),
        \quad 
        \frac{ \diam(T)^{-1} }{\sqrt{n}} \leq \matnorm{ A^{-1} } \leq \algebraicshapemeasure(A) \sqrt{2} \diam(T)^{-1}.
    \end{align*}
    \begin{align*}
        \frac{\aspectratio(T)}{\sqrt{2n}} \leq \algebraicshapemeasure(T) \leq n \aspectratio(T).
    \end{align*}

\end{lemma}
\begin{proof}
    Let $\varphi : \Delta^{n} \rightarrow T$ be an affine transformation and let $M := \Jacobian\phi$ be its Jacobian. 
    We begin with observing that the largest $\ell^{2}$-norm of any column of $A$,
    which here denote by $c_{\max}(A)$, 
    equals the maximum of the quotient $\| A x\|_{\ell^{2}} / \| x \|_{\ell^{1}}$ over all non-zero $x \in \bbR^{n}$. 
    We also know that the diameter of $T$ is the length of its longest edge. 
    Our first pair of inequalities follows via standard comparisons of Euclidean norms
    \begin{align*}
        \frac{\diam(T)}{\sqrt 2} \leq \matnorm{ A } \leq \sqrt{n} c_{\max}(A) \leq \sqrt{n} \diam(T).
    \end{align*}
    The columns of the matrix $\Jacobian \varphi^{-1}$ are the gradients of the barycentric coordinates of the vertices of $T$,
    except for the vertex $\varphi(0)$. It immediately follows that 
    \begin{align*}
        \matnorm{ \Jacobian \varphi^{-1} } \leq \sqrt{n} c_{\max}(A^{-1}) \leq \sqrt{n} \cdot \height(T)^{-1}
        .
    \end{align*}
    The smallest height in the reference simplex $\Delta^{n}$ is $\height_{\Delta} = 1/\sqrt{n}$, 
    whence $\height(T) \geq \matnorm{A^{-1}}^{-1} / \sqrt{n}$. This yields our second pair of inequalities. 
    Notice that $\height(T)^{-1} \leq \aspectratio(T) \diam(T)$.
    All relevant results follow.
    
    Next, the above results also imply 
    \begin{align*}
        \matnorm{ A      } \geq \matnorm{ A^{-1} }^{-1} \geq \frac{ \height(T)    }{\sqrt{n}},
        \quad 
        \matnorm{ A^{-1} } \geq \matnorm{ A      }^{-1} \geq \frac{ \diam(T)^{-1} }{\sqrt{n}}.
    \end{align*}
    \begin{align*}
        \| A \| &\leq \algebraicshapemeasure(A) \matnorm{A^{-1}}^{-1} \leq \algebraicshapemeasure(A) \sqrt{n} \height(T),
        \\
        \| A^{-1} \| &\leq \algebraicshapemeasure(A) \matnorm{A}^{-1} \leq \algebraicshapemeasure(A) \sqrt{2} \diam(T)^{-1}.
    \end{align*}
    This shows the last two pairs of inequalities. The comparison of shape measures is a consequence. 
\end{proof}



   
\begin{lemma}\label{lemma:measurerelationships}
    Let $T$ be an $n$-simplex and let $\varphi : \Delta^{n} \rightarrow T$ be an affine diffeomorphism from the reference $n$-simplex. Then 
    \begin{gather*}
        \matnorm{\Jacobian \varphi}
        \leq 
        \Ceins{n} \cdot \diam(T) % DONE: can this improved?
        ,
        \qquad 
        \matnorm{ \Jacobian \varphi^{-1} }
        \leq 
        \Czwei{n} \cdot 
        \geometricshapemeasure(T) 
        \diam(T)^{-1}
        ,
        \\
        \algebraicshapemeasure(T)
        \leq 
        \Cdrei{n}
        % \sqrt{ \frac{n}{ (n-1)^{n-1} } }
        \geometricshapemeasure(T)
        ,
        \qquad 
        %         \geometricshapemeasure(T)^{\frac 1 n}
        %         \leq 
        \aspectratio(T)
        \leq 
        \geometricshapemeasure(T)
        .
    \end{gather*}
    Here,
    \begin{align*}
        \Ceins{n} \leq \sqrt{n},
        \quad 
        \Czwei{n} \leq          \left( \frac{n-1}{n} \right)^{\frac{2}{n-1}}, 
        \quad 
        \Cdrei{n} \leq \Ceins{n}\Czwei{n} %\sqrt{n} \left( \frac{n}{n-1} \right)^{\frac{n-1}{2}}
        . 
    \end{align*}
\end{lemma}
\begin{proof}
    We easily verify the first inequality. To prove the second inequality, let $\sigma_1 \geq \dots \geq \sigma_n \geq 0$ denote the singular values of the Jacobian $\Jacobian\varphi$ in descending order.
    The Jacobian determinant is the product of these singular values. 
    By a result of Hong and Pan \cite{hong1992lower},
    \begin{align*}
        \sigma_{n}
        &\geq 
        \left( \frac{n-1}{n} \right)^{\frac{n-1}{2}}
        \frac{ \det(\Jacobian\varphi) }{ \diam(T)^{n-1} }
        = 
        \left( \frac{n-1}{n} \right)^{\frac{n-1}{2}}
        \frac{ n! \vol(T) }{ \diam(T)^{n-1} }
        .
    \end{align*}
%     To get the second inequality, we multiple the last term by $\sigma_{1} / ( \sqrt{n} \diam(T) ) \leq 1$.
    The second inequality follows by taking the reciprocal. 
    The definition of the condition number now gives the third inequalities. 
    % To get the third inequality, we divide both sides by $\sigma_{1}$ and use $\sigma_{1} \leq \sqrt{n} \diam(T)$.
    Finally, 
    if $F$ is the face of $T$ opposite to the vertex with the smallest height, then 
    \begin{align*}
        \frac{ \diam(T) }{ \height(T) }
        \leq 
        \frac{ \diam(T) \vol(F) }{ n \vol(T) }
        \leq 
        \frac{ \diam(T)^{n} }{ n! \vol(T) }
        .
    \end{align*}
%     Since $n! \vol(T)$ is bounded from below by the product of any $n$ different heights of $T$, it follows that 
%     \begin{align*}
%         \frac{ \diam(T)^{n} }{ n! \vol(T) }
%         \leq 
%         \frac{ \diam(T)^{n} }{ \height(T)^{n} }
%         .
%     \end{align*}
    That shows the fourth and last of the inequalities. 
\end{proof}



 
\begin{remark}
    Whenever $T$ is an $n$-dimensional simplex, the ratio $\geometricshapemeasure(T)$ measures the ``shape quality'' of a simplex and is instance of a so-called \emph{shape measure}.
    In our convention, the reference simplex has unit shape measure.
    For example, $\geometricshapemeasure(T)=1$ whenever $T$ is a line segment, an isosceles right-angled triangle, or the generalizations of these reference simplices to higher dimensions. $\geometricshapemeasure(T)$ is slightly bigger than one for equilateral simplices and blows up with distortion.

% Whenever $T$ is an $n$-dimensional simplex, the ratio $\diam(T)^{n} / \vol(T)$ measures the ``shape quality'' of a simplex and is instance of a so-called \emph{shape measure}.
    % We can also rescale that by a factor of $1/n!$, as suggested from the above. 
    The geometric ratio $\geometricshapemeasure(T)$ equals what is known as \emph{fatness} in differential geometry~\cite{cheeger1984curvature} and is precisely the reciprocal of the \emph{fullness} discussed by Whitney~\cite{whitney2012geometric}. 
    The \emph{thickness} of a simplex is the ratio of its smallest height to its diameter~\cite{munkres2016elementary}. \mwl{Update this entire paragraph.}
    Alternative shape measures have been used throughout the literature of numerical analysis and computational geometry to quantify the quality of simplices.
    One example, commonly referenced in finite element literature, is the ratio of the diameter and the radius of the largest inscribed circle of a simplex 
% TODO     (see~\cite[p.61, Definition 5.1]{braess2001finite}, ~\cite[p.97, Definition (4.2.16)]{brenner2008mathematical},~\cite[Definition~11.2]{ern2021finite}). 
\end{remark}






\begin{comment}
    On the other hand, 
    \begin{align*}
        % h_1 &\leq \diam(T_1) \leq \frac{\diam(T_1)^{n}}{n \cdot \vol(T_1)} \diam(F),
%         \\
%         h_2 &= n \frac{\vol(T_2)}{\vol(F)} \geq n \cdot \frac{\vol(T_2)}{\diam(T_2)^{n}} \diam(T_2) \geq n \cdot \frac{\vol(T_2)}{\diam(T_2)^{n}} \diam(F)
        h_1 \leq \aspectratio(T_{1}) \diam(F),
        \qquad 
        h_2 \geq \aspectratio(T_{2})^{-1} \diam(T_2)\geq \aspectratio(T_{2})^{-1} \diam(F)
        .
    \end{align*}
    The second estimate follows. 
    Lastly, $\Xi : T_1 \rightarrow T_2$ be as above
    can be written as the composition $\Xi = \phi_{2} \phi^{-1}_{1}$ for some affine transformations $\phi_1 : \Delta^{n} \rightarrow T_1$ and $\phi_2 : \Delta^{n} \rightarrow T_2$. 
    Lemma~\ref{lemma:measurerelationships} and the first estimate imply 
    \begin{align*}
        \matnorm{ \Jacobian \Xi }
        \leq 
        \matnorm{ \Jacobian\phi_{2} } \cdot \matnorm{ \Jacobian\phi_{1}^{-1} }
        \leq 
        \Ceins{n}^{2} \diam(T_2) \height(T_1)^{-1}
        \leq 
        \Ceins{n}^{2} \aspectratio(T_2) \diam(F) \height(T_1)^{-1}
        \leq 
        \Ceins{n}^{2} \aspectratio(T_2) \aspectratio(T_1)
        .
    \end{align*}
%     \begin{align*}
%         \matnorm{ \Jacobian \Xi }
%         &
%         \leq 
%         \Ceins{n}
%         \Czwei{n}
%         \geometricshapemeasure(T_2)
%         \diam(T_1)
%         \diam(T_2)^{-1}
%         \\&
%         \leq 
%         \Ceins{n}
%         \Czwei{n}
%         \geometricshapemeasure(T_2)
%         \geometricshapemeasure(T_1) 
%         \diam(F)
%         \diam(T_2)^{-1}
%         \leq 
%         \Ceins{n}
%         \Czwei{n}
%         \geometricshapemeasure(T_2)
%         \geometricshapemeasure(T_1) 
%         .
%     \end{align*}
    All the desired results thus follow.
\end{comment}






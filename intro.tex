We illustrate the subject of our discussion with the curl operator. 
If $\bff \in \bfL^{2}(\Omega)$ is a square-integrable vector field on some domain $\Omega \subseteq \bbR^3$, 
we want to find a potential $\bfu \in \bfL^2(\Omega)$ under the curl operator, i.e., $\curl \bfu = \bff$, such that the $\bfL^2$ norm of the potential is bounded in terms of the source field $\bff$. Specifically, we want the Poincar\'e--Friedrichs inequality % DONE: Theo prefers not using ``expressed''
\begin{align*}
    \min\limits_{ \substack{ \bfv \in \bfL^{2}(\Omega) \\ \curl \bfv = \bff } } \| \bfv \|_{L^{2}(\Omega)}
    \leq 
    C_{\curl}
    \| \bff \|_{L^{2}(\Omega)}
    .
\end{align*}
The constant in this inequality corresponds to a lower bound for the Maxwell eigenvalues and quantifies the stability properties of the Maxwell system on that domain. 
Broadly speaking, upper bounds for the Poincar\'e--Friedrichs constants correspond to lower bounds for the non-zero eigenvalues of the scalar and vector Laplacians.
In numerical methods, estimates for the Poincar\'e--Friedrichs constant enter the stability and convergence theory of finite element methods, as well as spectral bounds for finite element matrices. 
\\

Finding a potential operator, whether linear or non-linear, and bounding the Poincar\'e--Friedrichs constant is generally not an easy undertaking. 
The majority of estimates in the literature address only gradient potentials. 
We carry out this endeavor in the special case where $\Omega$ admits a so-called \emph{shellable} triangulation. 
Even though only contractible domains can ever admit a shellable triangulation, 
having computable upper bounds for such domains is an important stepping stone towards the more general case. 
For example, local stars around a simplex within a larger 2D or 3D triangulation are shellable.
Poincar\'e--Friedrichs inequalities over local stars are our main goal in this exposition. 
\\


Before we outline the main results of this exposition in more detail, 
we first review the state of the literature on Poincar\'e--Friedrichs inequalities on specific classes of domains. 


Most research on optimal constants has addressed the case of convex domains. 
Notably, if the constants are required to depend on the convex domain only via its diameter, 
then the optimal constants $C_{\grad,\Omega,p}$ for any $1 \leq p < \infty$ are known explicitly~\cite{bebendorf2003note,acosta2004optimal,esposito2013poincare,ferone2012remark}.
When the domain is not convex but star-shaped with respect to a ball,
then polynomial interpolation estimates already imply the Poincar\'e inequality~\cite{brenner2008mathematical,ern2021finite}. 
Poincar\'e inequalities hold over star-shaped open sets~\cite[Theorem~3.1]{hurri1988poincare}. % DONE: strange sentence 

We can access some estimates from the literature in the formalism of exterior calculus. 
We are aware of work by Guerini and Savo~\cite{guerini2004eigenvalue},
who provide lower bounds for the spectrum of the Hodge--Laplace operator on convex domains with smooth boundary, % DONE: Theo : smoothly bounded -> with smooth boundary
thus giving upper bounds for the Poincar\'e--Friedrichs constant of the exterior derivative, and hence for the curl and divergence operators, when $p=2$.
By duality, that implies lower bounds for the case of full boundary conditions.




Actually, in the present gradient setting and since $\Omega$ is connected, already the gradient constraint determines $\Phi_{\grad}( \bff )$ up to a constant\cye{, so that~\eqref{math:pot_grad_def} is a one-dimensional minimization problem}. 




\subsection{Conceptual overview: Potential for the gradient operator and the corresponding Poincar\'e--Friedrichs inequality} \label{subsection:intro_grad}

Let $\Omega \subseteq \bbR^n$, $n \geq 1$, be a bounded connected open set. Given any $1 \leq p \leq \infty$, we let $L^{p}(\Omega)$ denote the Lebesgue space over $\Omega$ with integrability exponent $p$. Then $\bfL^{p}(\Omega) := L^{p}(\Omega)^{n}$ is the corresponding Lebesgue space of vector  fields. 

Let $W^{1,p}(\Omega)$ denote the first-order Sobolev space over $\Omega$ with integrability exponent $p$, 
\begin{align} \label{equation:W1p}
    W^{1,p}(\Omega) := \left\{u \in L^{p}(\Omega) \cye{\suchthat} \nabla u \in \bfL^{p}(\Omega) \right\}.
\end{align}
For a vector field $\bff$ that is a distributional gradient of an object from $W^{1,p}(\Omega)$, $\bff \in \nabla W^{1,p}(\Omega)$, we are \cye{interested in} \emph{potentials}, i.e., scalar-valued functions $\Phi_{\grad}( \bff ) \in W^{1,p}(\Omega)$ such that $\nabla \Phi_{\grad}( \bff ) = \bff$. Then, 
%
\begin{align*}
    \Phi_{\grad} : \nabla W^{1,p}(\Omega) \rightarrow L^{p}(\Omega)
\end{align*}
is the \emph{potential operator}. A particular example of our interest is
%
\begin{align}\label{math:pot_grad_def}
    \Phi_{\grad}( \bff ) := \argmin\limits_{ \substack{ u \in W^{1,p}(\Omega) \\ \nabla u = \bff } } \| u \|_{L^{p}(\Omega)}.
\end{align}
%
Actually, in the present gradient setting and since $\Omega$ is connected, already the gradient constraint determines $\Phi_{\grad}( \bff )$ up to a constant, so that~\eqref{math:pot_grad_def} is a one-dimensional minimization problem. 

We want to see whether the $L^{p}(\Omega)$ norm of the potential $\Phi_{\grad}( \bff )$ is bounded in terms of the $\bfL^{p}(\Omega)$ norm of the source field $\bff$, i.e., we want to evaluate the \emph{operator norm}
\begin{align}\label{math:pot_grad_norm}
    C_{\grad,\Omega,p} := \max\limits_{ u \in W^{1,p}(\Omega) \setminus \bbR } 
    \frac{ \| \Phi_{\grad}(\nabla u) \|_{L^{p}(\Omega)} }{ \| \nabla u \|_{\fooL^{p}(\Omega)} }.
\end{align}
This operator norm is indeed finite and it equivalently turns out to be the best constant in the \emph{Poincar\'e--Friedrichs inequality}: for every vector field $\bff \in \nabla W^{1,p}(\Omega)$, there exists $u \in W^{1,p}(\Omega)$ such that $\nabla u = \bff$ and 
\begin{align} \label{math:PF_gradient_1}
    \| u \|_{L^{p}(\Omega)}
    \leq 
    C_{\grad,\Omega,p} 
    \| \bff \|_{\fooL^{p}(\Omega)}.
\end{align}
Actually, \eqref{math:PF_gradient_1} is further equivalent to the more conventional writing
\begin{align} \label{math:PF_gradient_2}
    \min_{ c \in \bbR } \| u - c \|_{L^{p}(\Omega)}
    \leq 
    C_{\grad,\Omega,p}
    \| \nabla u \|_{\fooL^{p}(\Omega)},
    \quad 
    \forall 
    u \in W^{1,p}(\Omega)
    .
\end{align}
We call $C_{\grad,\Omega,p}$ the \emph{Poincar\'e--Friedrichs constant} for the gradient operator with exponent $p$. \cye{A concise writing of all~\eqref{math:pot_grad}--\eqref{math:PF_gradient} is
\begin{align} \label{math:pot_PF_grad}
    \min\limits_{ \substack{ u \in W^{1,p}(\Omega) \\ \nabla u = \bff } } 
    \| u \|_{L^{p}(\Omega)}
    &\leq 
    C_{\grad,\Omega,p}
    \| \bff \|_{\fooL^{p}(\Omega)}.
\end{align}
for all $\bff \in \nabla W^{1,p}(\Omega)$.}

\cye{We present a more profound discussion below in Section~\ref{section:poincare}, where we also discuss the difference with the Poincar\'e inequality, where the mean value of $u$ is fixed to zero, and address why we later employ linear potential operators to estimate the Poincar\'e--Friedrichs constant and not directly the minimizing but nonlinear (for $p \neq 2$) potential operators such as~\eqref{math:pot_grad_def}.} 










\subsection{Conceptual overview: Potentials for the curl and divergence operators and the corresponding Poincar\'e--Friedrichs inequalities} \label{section:intro_curl_div}

Consider the vector-valued spaces
\begin{align}\label{equation:W_curl_div_p} 
    \bfW^{p}(\curl,\Omega) &:= \left\{ \bfu \in L^{p}(\Omega)^{3} \cye{\suchthat} \curl \bfu \in L^{p}(\Omega)^{3} \right\}
    ,
    \\
    \bfW^{p}(\divergence,\Omega) &:= \left\{ \bfu \in L^{p}(\Omega)^{3} \cye{\suchthat} \divergence \bfu \in L^{p}(\Omega) \right\}
    .
\end{align}
The members of these spaces are those vector fields whose distributional curls and divergences are \cye{respectively} in \cye{the} Lebesgue spaces \cye{$\bfL^{p}$ or $L^{p}(\Omega)$}.  
This only requires that certain sums of distributional partial derivatives are integrable, 
and hence these spaces are not classical Sobolev spaces of vector  fields. 

\cye{Let a vector field $\bff$ that is a distributional curl of an object from $\bfW^{p}(\curl,\Omega)$, $\bff \in \curl \bfW^{p}(\curl,\Omega)$, and scalar-valued field $f$ that is a distributional divergence of an object from $\bfW^{p}(\divergence,\Omega)$, $f \in \divergence \bfW^{p}(\divergence,\Omega)$, be given.} 
We are interested in constants for inequalities such as 
\begin{align}\label{math:pot_PF_curl_div}
    \min\limits_{ \substack{ \bfu \in \bfW^{p}(\curl,\Omega) \\ \curl \bfu = \bff } } 
    \| \bfu \|_{L^{p}(\Omega)}
    &\leq 
    C_{\curl,\Omega,p}
    \| \bff \|_{L^{p}(\Omega)} \label{math:pot_PF_curl}
    ,
    \\
    \min\limits_{ \substack{ \bfu \in \bfW^{p}(\divergence,\Omega) \\ \divergence \bfu = f } } 
    \| \bfu \|_{L^{p}(\Omega)}
    &\leq 
    C_{\divergence,\Omega,p}
    \| f \|_{L^{p}(\Omega)} \label{math:pot_PF_div}
    ,
\end{align}
and how these constants can be estimated by the operator norms of \emph{linear} potential operators 
\begin{align*}
    &\Phi_{\curl} : \curl \bfW^{p}(\curl,\Omega) \rightarrow \bfW^{p}(\curl,\Omega)
    ,
    \\
    &\Phi_{\divergence} : \divergence \bfW^{p}(\divergence,\Omega) \rightarrow \bfW^{p}(\divergence,\Omega)
    .
\end{align*}
The fundamental difference \cye{with respect to Section~\ref{subsection:intro_grad}} is that the curl and divergence have infinite-dimensional kernels. In Section~\ref{subsection:intro_grad}, the kernel of the gradient is the one-dimensional space of constant functions. It is thus trivially complemented for all $p$, with a canonical choice of projection. By contrast, in the Banach space case $p \neq 2$, it is not immediately evident that the kernels of the curl and divergence operators are complemented, and a canonical projection only exists here in the Hilbert setting $p=2$. \cye{Actually,} the existence of a bounded linear potential operator here is a non-trivial fact even in the Hilbert space case $p = 2$. 

\cye{Let for a moment the topology of $\Omega$ be trivial, let $\Omega$ be weakly Lipschitz, and consider the Hilbert case $p=2$. Then, \eqref{math:pot_PF_curl} is related to the so-called Poincar\'e--Friedrichs--Weber inequality
\begin{align}
    \label{math:Weber}
    \| \bfu\|_{L^{2}(\Omega)} \leq \tilde C_{\curl,\Omega,p} \| \curl \bfu \|_{L^{2}(\Omega)},
\end{align}
for all $\bfu \in \bfW^{2}(\curl,\Omega) \cap \bfW^{2}(\divergence,\Omega)$ with either the tangential or the normal trace zero on $\partial \Omega$ and $\divergence \bfu = 0$. 
Yet another writing is 
\label{math:PF_orth} \begin{align}
    \| \bfu\|_{L^{2}(\Omega)} & \leq \tilde C_{\curl,\Omega,p} \| \curl \bfu \|_{L^{2}(\Omega)}, \label{math:PF_orth_curl}\\
    \| \bfu\|_{L^{2}(\Omega)} & \leq \tilde C_{\divergence,\Omega,p} \| \divergence \bfu \|_{L^{2}(\Omega)} \label{math:PF_orth_div}
\end{align}
for respectively all functions $\bfu$ from $\bfW^{2}(\curl,\Omega)$ such that $\int_{\Omega} \bfu(x) \cdot \bfv(x) \;dx = 0$ for all $\bfv$ from $\bfW^{2}(\curl,\Omega)$ with $\curl \bfv = 0$ and all functions $\bfu$ from $\bfW^{2}(\divergence,\Omega)$ such that $\int_{\Omega} \bfu(x) \cdot \bfv(x) \;dx = 0$ for all $\bfv$ from $\bfW^{2}(\divergence,\Omega)$ with $\divergence \bfv = 0$. We refer to Equation~(5) in~\cite{Fried_diff_forms_55}, Equation~(2) in Gaffney~\cite{Gaff_Hilbert_harm_55}, Weber~\cite{Web_compact_Maxw_80}, \cite[Lemmas~3.4 and~3.6]{Gir_Rav_NS_86}, \cite[Proposition~7.4]{Fer_Gil_Maxw_BC_97}, and the references therein. Still} in the Hilbert space case $p=2$, the Friedrichs inequality~\cite{burenkov1998sobolev} over the Sobolev space with \cye{homogeneous} Dirichlet boundary conditions \cye{on $\partial \Omega$} implies, via duality~\eqref{math:pot_PF_div}. However, that easy observation seems restricted to $p=2$. 

Not much attention\cye{, however,} seems to have been given to the study of \cye{computable} constants in~\eqref{math:pot_PF_curl_div}, even if the domain $\Omega$ is convex. \cye{Important results in the Hilbert case $p=2$ can be found in~\cite{guerini2004eigenvalue, Paul_Vald_PF_grad_curl_div_20} that we discuss now in the broader setting of} the exterior calculus formalism\cye{, but the Banach case $p \neq 2$ seems largely unexplored.}





In the Hilbert space case, $p=2$, and when the domain has a smooth boundary, the work of Guerini and Savo~\cite{guerini2004eigenvalue} shows that the Poincar\'e--Friedrichs constant for the gradient already estimates the corresponding constants for the curl and divergence operators. 
By duality, this also is an upper estimate of the Poincar\'e--Friedrichs constants for the gradient, curl, and divergence subject to Dirichlet, tangential, and normal boundary conditions, respectively, along the entire boundary. 
However, no such results are known over convex Lipschitz domains and for general Lebesgue exponents $1 \leq p \leq \infty$.


There is a wealth of literature for gradient Poincar\'e inequalities over convex domains, where optimal gradient Poincar\'e--Friedrichs constants for the entire range $1 \leq p < \infty$ are known.




The literature is less extensive it comes to the estimates for the curl and divergence potentials.



A conceptual approach to obtain computable estimates for the Poincar\'e--Friedrichs constants \cye{for any $1 \leq p \leq \infty$} and when the domain is star-shaped with respect to a ball is to bound operator norms of 
regularized Poincar\'e operators \cye{such as those of} Costabel and McIntosh~\cite{costabel2010bogovskiui} as mappings between Lebesgue spaces. 

When the domain is star-shaped with respect to a ball, 
then estimates for the Poincar\'e--Friedrichs constants 
follow from bounds on the operator norms of regularized Poincar\'e operators~\cite{costabel2010bogovskiui}
as mappings between Lebesgue spaces. 
We are aware of estimates for the higher-order seminorms of these potentials~\cite{guzman2021estimation} (see also~\cite{gallistl2023computational}),
but estimates in Lebesgue norms have not been made explicit in the literature yet, to the best of our knowledge.
\\



\cye{Let us only briefly discuss why don't we treat local \cye{patches} (stars) in triangulations as} domains star-shaped with respect to a ball, which they are. 
However, estimates that rely on this geometric condition deteriorate when the aforementioned ball has radius much smaller than the domain diameter. 
This is not as much a problem over patches around interior subsimplices, where the size of the interior ball only depends on the shape regularity of the triangulation. 
But that interior ball can be arbitrarily small when the patch is around a boundary simplex, even if the mesh has good shape regularity: 
this occurs when the domain has sharp reentrant corners, and some illustrative limit cases include the slit domain~\cite{veeser2012poincare} and the crossed bricks domain~\cite{licht2019smoothed}, 
which contain finite element patches are not star-shaped with respect to any ball. 

One possible avenue towards Poincar\'e--Friedrichs constants is the observation that local stars in triangulations are star-shaped with respect to a ball.









\subsection{Review: Triangulated domains}

% Intro DONE
Domains that admit finite triangulations allow for different techniques to attain Poincar\'e--Friedrichs inequalities. 
However, the majority of results only address such inequalities for scalar-valued functions, not for vector fields. 
We review some of the main outcomes. 

% veeser and verfurth ... unknown DONE 
Veeser and Verf\"urth provide computable upper bounds in the case of the classical Sobolev space $W^{1,p}(\Omega)$ over triangulated domains, with focus on efficient estimates for finite element stars~\cite{veeser2012poincare}. Naturally, their estimates depend on the shape regularity of the mesh.

% sequential approach 
The literature~\cite{Eym_Gal_Her_00,vohralik2005discrete,ern2020stable,ern2021finite,Chaum_Voh_p_rob_3D_H_curl_23,Voh_loc_glob_H1_24} knows different variations of a popular avenue towards Poincar\'e--Friedrichs constants that circumvents the effect of low boundary regularity and constructs potentials step-by-step: 
starting with a single triangle, the potential is constructed over increasing subdomains, until the entire domain is exhausted. 
That procedure applies to general triangulated domains, not only local stars, though the latter is our main interest here. 
The underlying idea is that we first construct a potential for the gradient over an initial simplex. 
Every time we have found a potential over a subdomain, we construct a potential over a neighboring simplex or patch:
along the interfacing intersection, the two potentials will only differ by a constant, 
and that difference can easily be removed to ensure continuity across that interface. 

% sequential approach DONE
Domains that admit finite triangulations allow for techniques \cye{using some form of passing through the triangulation and construction of the potentials step-by-step}. For scalar-valued functions in the context of Section~\ref{subsection:intro_grad}, Veeser and Verf\"urth~\cite{veeser2012poincare} provide computable upper bounds in the case of the classical Sobolev space $W^{1,p}(\Omega)$ over triangulated domains, with focus on efficient estimates for \cye{vertex} stars. 
%Naturally, these estimates deteriorates when the shape regularity of the mesh deteriorates. 
Such approaches were also previously used in the context of finite volume methods~\cite{Eym_Gal_Her_00}, broken (weakly continuous) Sobolev spaces~\cite{vohralik2005discrete}, or more recently in continuous--discrete comparison results~\cite{Brae_Pill_Sch_p_rob_09, ern2020stable, Chaum_Voh_p_rob_3D_H_curl_24,  Voh_loc_glob_H1_24}. Triangle by triangle, the potential is constructed over increasing subdomains, matching along the interfacing intersections, until the entire domain is exhausted.  

% eigenvalue and maxwell
We also remark that recent work~\cite{gallistl2023computational} addresses eigenvalues for the curl--curl operator over general triangulations,
which gives rise to an estimate for the Poincar\'e--Friedrichs constant when $p=2$. However, the use case of these estimates are sequences of triangulations after sufficiently many levels of refinement, whereas our main application are estimates that work with fixed but small triangulations, such as local stars. 
We are not aware of any estimates for the curl operator, even in the Hilbert space situation. 

\subsection{Review: Numerically computed guaranteed bounds on Poincar\'e--Friedrichs constants seen as eigenvalues}

Estimates for Laplacian eigenvalues over triangulated domains have received much attention. 
The constant in~\eqref{math:pot_PF_grad} for $p=2$ corresponds to a lower bound for the Laplace eigenvalues and quantifies the stability properties of the Laplacian on the domain $\Omega$. 
Similarly, the constant in~\eqref{math:pot_PF_curl} for $p=2$ corresponds to a lower bound for the Maxwell eigenvalues and quantifies the stability properties of the Maxwell system on $\Omega$. 
Thus, computable upper bounds on the Poincar\'e--Friedrichs constants also give computable lower bounds for the eigenvalues of the associated Laplacians and vice versa. 
Prominent methods numerically compute guaranteed upper bounds on the Poincar\'e--Friedrichs constants upon solving a finite element system over a sufficiently fine triangulation and using some clever post-processing estimates.
This technique is used for estimating scalar Laplacian eigenvalues~\cite{Cars_Ged_LB_eigs_14,Liu_fram_eigs_15} and vector Laplacian eigenvalues~\cite{gallistl2023computational}. 
We remark that we wish to avoid the solution of (global) finite element systems in our computations. 
% Our approaches, in contrast, are either analytical or combinatorial, and in any case we only employ the given geometry but no computer simulation.



 
 

 
Important results in the Hilbert case $p=2$ can be found in~\cite{guerini2004eigenvalue, Paul_Vald_PF_grad_curl_div_20} that we discuss now in the broader setting of the exterior calculus formalism, 
but the Banach case $p \neq 2$ seems largely unexplored.














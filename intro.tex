We illustrate the subject of our discussion with the curl operator. 
If $\bff \in \bfL^{2}(\Omega)$ is a square-integrable vector field on some domain $\Omega \subseteq \bbR^3$, 
we want to find a potential $\bfu \in \bfL^2(\Omega)$ under the curl operator, i.e., $\curl \bfu = \bff$, such that the $\bfL^2$ norm of the potential is bounded in terms of the source field $\bff$. Specifically, we want the Poincar\'e--Friedrichs inequality % DONE: Theo prefers not using ``expressed''
\begin{align*}
    \min\limits_{ \substack{ \bfv \in \bfL^{2}(\Omega) \\ \curl \bfv = \bff } } \| \bfv \|_{L^{2}(\Omega)}
    \leq 
    C_{\curl}
    \| \bff \|_{L^{2}(\Omega)}
    .
\end{align*}
The constant in this inequality corresponds to a lower bound for the Maxwell eigenvalues and quantifies the stability properties of the Maxwell system on that domain. 
Broadly speaking, upper bounds for the Poincar\'e--Friedrichs constants correspond to lower bounds for the non-zero eigenvalues of the scalar and vector Laplacians.
In numerical methods, estimates for the Poincar\'e--Friedrichs constant enter the stability and convergence theory of finite element methods, as well as spectral bounds for finite element matrices. 
\\

Finding a potential operator, whether linear or non-linear, and bounding the Poincar\'e--Friedrichs constant is generally not an easy undertaking. 
The majority of estimates in the literature address only gradient potentials. 
We carry out this endeavor in the special case where $\Omega$ admits a so-called \emph{shellable} triangulation. 
Even though only contractible domains can ever admit a shellable triangulation, 
having computable upper bounds for such domains is an important stepping stone towards the more general case. 
For example, local stars around a simplex within a larger 2D or 3D triangulation are shellable.
Poincar\'e--Friedrichs inequalities over local stars are our main goal in this exposition. 
\\


Before we outline the main results of this exposition in more detail, 
we first review the state of the literature on Poincar\'e--Friedrichs inequalities on specific classes of domains. 


Most research on optimal constants has addressed the case of convex domains. 
Notably, if the constants are required to depend on the convex domain only via its diameter, 
then the optimal constants $C_{\grad,\Omega,p}$ for any $1 \leq p < \infty$ are known explicitly~\cite{bebendorf2003note,acosta2004optimal,esposito2013poincare,ferone2012remark}.
When the domain is not convex but star-shaped with respect to a ball,
then polynomial interpolation estimates already imply the Poincar\'e inequality~\cite{brenner2008mathematical,ern2021finite}. 
Poincar\'e inequalities hold over star-shaped open sets~\cite[Theorem~3.1]{hurri1988poincare}. % DONE: strange sentence 

We can access some estimates from the literature in the formalism of exterior calculus. 
We are aware of work by Guerini and Savo~\cite{guerini2004eigenvalue},
who provide lower bounds for the spectrum of the Hodge--Laplace operator on convex domains with smooth boundary, % DONE: Theo : smoothly bounded -> with smooth boundary
thus giving upper bounds for the Poincar\'e--Friedrichs constant of the exterior derivative, and hence for the curl and divergence operators, when $p=2$.
By duality, that implies lower bounds for the case of full boundary conditions.




Actually, in the present gradient setting and since $\Omega$ is connected, already the gradient constraint determines $\Phi_{\grad}( \bff )$ up to a constant\cye{, so that~\eqref{math:pot_grad_def} is a one-dimensional minimization problem}. 

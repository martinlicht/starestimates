\documentclass[a4paper]{article}
\usepackage[utf8]{inputenc}
\usepackage{fullpage}
\usepackage{hyperref}         % For hypertext links


\usepackage[dvipsnames]{xcolor}
\usepackage{amsmath}
\usepackage{amssymb}
\usepackage{amsfonts}
\usepackage{amsthm}
\usepackage{graphicx}
\usepackage{tikz}
\usepackage{tikz-cd}
\usetikzlibrary{calc}
\usepackage{amscd}


\newtheorem{definition}{Definition}[section]
\newtheorem{theorem}{Theorem}[section]
\newtheorem{lemma}[theorem]{Lemma}
\newtheorem{remark}[theorem]{Remark}
\newtheorem{example}[theorem]{Example}
\newtheorem{proposition}[theorem]{Proposition}

\makeatletter
\newcommand\suchthat{%
 \@ifstar
  {\mathrel{}\middle|\mathrel{}}
  {\mid}%
}
\makeatother

\newcommand{\Ceins}[1]{C_{1,#1}}
\newcommand{\Czwei}[1]{C_{2,#1}}
\newcommand{\Cdrei}[1]{C_{3,#1}}
\newcommand{\Cvier}[2]{C_{4,#1,#2}}
\newcommand{\Cfive}[2]{C_{5,#1,#2}}
% \newcommand{\Ceins}{\sqrt{n}}
% \newcommand{\Czwei}{\sqrt{2n}}



\newcommand{\Perm}{{\operatorname{Perm}}}

\newcommand{\linhull}{{\operatorname{span}}}
\newcommand{\convexhull}{{\operatorname{convex}}}

\newcommand{\Poinc}{\frakP}
\newcommand{\Bogov}{\frakB}

\newcommand{\mollifier}{\frakm}

\newcommand{\eps}{\epsilon}

\newcommand{\Id}{\rm{Id}}

\newcommand{\Lip}{\rm{Lip}}

\newcommand{\diff}{\mathop{}\!\mathrm{d}}

\DeclareMathOperator{\adj}{adj}
%\newcommand{\adj}{{\rm adj}}
\newcommand{\inv}{{-1}}
\newcommand{\invt}{{-t}}
\newcommand{\tinv}{{-t}}
\newcommand{\determinant}{{\operatorname{det}}}
\DeclareMathOperator{\Jacobian}{{\rm Jac}}
% \newcommand{\Jacobian}{{\rm Jac}}
\newcommand{\dimension}{\operatorname{dim}}
\newcommand{\sgn}{\operatorname{sgn}}
\newcommand{\adjugate}{\operatorname{adj}}
\newcommand{\sym}{\operatorname{sym}}
\newcommand{\signum}{\operatorname{sgn}}
\newcommand{\kronecker}{{\hat\delta}}
\newcommand{\dom}{\operatorname{dom}}
\newcommand{\codom}{\operatorname{codom}}
\newcommand{\rng}{\operatorname{ran}}
\newcommand{\convex}{\operatorname{convex}}
\newcommand{\coker}{\operatorname{coker}}
\newcommand{\coran}{\operatorname{coran}}

\newcommand{\dif}{{\mathrm d}}
\newcommand{\Dif}{{\mathrm D}}
\newcommand{\grad}{\operatorname{grad}}
\newcommand{\Grad}{\operatorname{Grad}}
\newcommand{\curl}{\operatorname{curl}}
\newcommand{\Curl}{\operatorname{Curl}}
\newcommand{\rot}{\curl}
\newcommand{\divergence}{\operatorname{div}}
\newcommand{\diver}{\operatorname{div}}
\newcommand{\svcurl}{\operatorname{sv-curl}}
\newcommand{\vscurl}{\operatorname{vs-curl}}
\newcommand{\sdiver}{\operatorname{sdiv}}
\newcommand{\Sdiver}{\operatorname{Sdiv}}
\newcommand{\sgrad}{\operatorname{sgrad}}
\newcommand{\Sgrad}{\operatorname{Sgrad}}
\newcommand{\Laplace}{\bigtriangleup}
\newcommand{\laplace}{\bigtriangleup}
\newcommand{\laplacian}{\laplace}
\newcommand{\Laplacian}{\laplace}
% \newcommand{\cartan}{{\mathsf d}}
% \newcommand{\cartanx}{{\mathsf d}x}
\newcommand{\cartan}{d}
\newcommand{\cartanx}{dx}

%\newcommand{\argmin}{\operatorname{argmin}}
\DeclareMathOperator*{\argmin}{{\rm argmin}}

% \DeclareMathOperator*{\carapace}{{\rm corona}}
\DeclareMathOperator*{\carapace}{{\partial \rm st}}

\newcommand{\supp}{\operatorname{supp}}
\newcommand{\esssup}{{\operatorname{esssup}}}
\newcommand{\essinf}{{\operatorname{essinf}}}

\newcommand{\equivalent}{ \Longleftrightarrow }
\newcommand{\vol}{\operatorname{vol}}
\newcommand{\st}{ \mid }
\newcommand{\diam}{{\operatorname{diam}}}
\newcommand{\height}{\operatorname{height}}
\newcommand{\dist}{\operatorname{dist}}
\newcommand{\patch}{\operatorname{st}}

\newcommand{\Trace}{\operatorname{Tr}}
\newcommand{\Tr}{\operatorname{Tr}}
\newcommand{\trace}{\operatorname{tr}}
\newcommand{\normaltrace}{\operatorname{nm}}
\newcommand{\tr}{\operatorname{tr}}
\newcommand{\Ext}{\operatorname{Ext}}
\newcommand{\Ex}{\operatorname{Ex}}
\newcommand{\ext}{\operatorname{ext}}

\newcommand{\Poincare}{\sfP}
\newcommand*{\volsphere}[1]{\color{red}{S_{#1}}}
\newcommand*{\volball}[1]{B_{#1}}

\newcommand{\subsimplex}{\calS^{\downarrow}}
\newcommand{\supsimplex}{\calS^{\uparrow}}
\newcommand{\supersimplex}{\supsimplex}
\newcommand{\orientation}{\mathscr{O}}
\newcommand{\restrict}{R}

\newcommand{\Mesh}{\calT}
\newcommand{\Vertices}{\calV}
\newcommand{\Edges}{\calE}
\newcommand{\Faces}{\calF}
\newcommand{\Ball}{\calB}
\newcommand{\Sphere}{S}
\newcommand{\underlying}[1]{\left| #1 \right|}

\newcommand{\Distr}{\calD}
\newcommand{\Cont}{\calC}
\newcommand{\Lebesgue}{L}
\newcommand{\Sobolev}{W}
\newcommand{\SOBOLEV}{\bfW}
\newcommand{\SobolevLambda}{W\Lambda}
\newcommand{\Alt}{\Lambda}
\newcommand{\loc}{\rm{loc}}

\newcommand{\Ned}{{\calN d}}
\newcommand{\RT}{{\calR T}}
\newcommand{\BDM}{{\calB \calD \calM}}
  

\newcommand*{\ConstantPF}{C_{\rm{PF}}}







\newcommand{\bbA}{{\mathbb A}}
\newcommand{\bbB}{{\mathbb B}}
\newcommand{\bbC}{{\mathbb C}}
\newcommand{\bbD}{{\mathbb D}}
\newcommand{\bbE}{{\mathbb E}}
\newcommand{\bbF}{{\mathbb F}}
\newcommand{\bbG}{{\mathbb G}}
\newcommand{\bbH}{{\mathbb H}}
\newcommand{\bbI}{{\mathbb I}}
\newcommand{\bbJ}{{\mathbb J}}
\newcommand{\bbK}{{\mathbb K}}
\newcommand{\bbL}{{\mathbb L}}
\newcommand{\bbM}{{\mathbb M}}
\newcommand{\bbN}{{\mathbb N}}
\newcommand{\bbO}{{\mathbb O}}
\newcommand{\bbP}{{\mathbb P}}
\newcommand{\bbQ}{{\mathbb Q}}
\newcommand{\bbR}{{\mathbb R}}
\newcommand{\bbS}{{\mathbb S}}
\newcommand{\bbT}{{\mathbb T}}
\newcommand{\bbU}{{\mathbb U}}
\newcommand{\bbV}{{\mathbb V}}
\newcommand{\bbW}{{\mathbb W}}
\newcommand{\bbX}{{\mathbb X}}
\newcommand{\bbY}{{\mathbb Y}}
\newcommand{\bbZ}{{\mathbb Z}}

\newcommand{\bfA}{{\mathbf A}}
\newcommand{\bfB}{{\mathbf B}}
\newcommand{\bfC}{{\mathbf C}}
\newcommand{\bfD}{{\mathbf D}}
\newcommand{\bfE}{{\mathbf E}}
\newcommand{\bfF}{{\mathbf F}}
\newcommand{\bfG}{{\mathbf G}}
\newcommand{\bfH}{{\mathbf H}}
\newcommand{\bfI}{{\mathbf I}}
\newcommand{\bfJ}{{\mathbf J}}
\newcommand{\bfK}{{\mathbf K}}
\newcommand{\bfL}{{\mathbf L}}
\newcommand{\bfM}{{\mathbf M}}
\newcommand{\bfN}{{\mathbf N}}
\newcommand{\bfO}{{\mathbf O}}
\newcommand{\bfP}{{\mathbf P}}
\newcommand{\bfQ}{{\mathbf Q}}
\newcommand{\bfR}{{\mathbf R}}
\newcommand{\bfS}{{\mathbf S}}
\newcommand{\bfT}{{\mathbf T}}
\newcommand{\bfU}{{\mathbf U}}
\newcommand{\bfV}{{\mathbf V}}
\newcommand{\bfW}{{\mathbf W}}
\newcommand{\bfX}{{\mathbf X}}
\newcommand{\bfY}{{\mathbf Y}}
\newcommand{\bfZ}{{\mathbf Z}}

\newcommand{\bfa}{{\mathbf a}}
\newcommand{\bfb}{{\mathbf b}}
\newcommand{\bfc}{{\mathbf c}}
\newcommand{\bfd}{{\mathbf d}}
\newcommand{\bfe}{{\mathbf e}}
\newcommand{\bff}{{\mathbf f}}
\newcommand{\bfg}{{\mathbf g}}
\newcommand{\bfh}{{\mathbf h}}
\newcommand{\bfi}{{\mathbf i}}
\newcommand{\bfj}{{\mathbf j}}
\newcommand{\bfk}{{\mathbf k}}
\newcommand{\bfl}{{\mathbf l}}
\newcommand{\bfm}{{\mathbf m}}
\newcommand{\bfn}{{\mathbf n}}
\newcommand{\bfo}{{\mathbf o}}
\newcommand{\bfp}{{\mathbf p}}
\newcommand{\bfq}{{\mathbf q}}
\newcommand{\bfr}{{\mathbf r}}
\newcommand{\bfs}{{\mathbf s}}
\newcommand{\bft}{{\mathbf t}}
\newcommand{\bfu}{{\mathbf u}}
\newcommand{\bfv}{{\mathbf v}}
\newcommand{\bfw}{{\mathbf w}}
\newcommand{\bfx}{{\mathbf x}}
\newcommand{\bfy}{{\mathbf y}}
\newcommand{\bfz}{{\mathbf z}}


\newcommand{\calA}{{\mathcal A}}
\newcommand{\calB}{{\mathcal B}}
\newcommand{\calC}{{\mathcal C}}
\newcommand{\calD}{{\mathcal D}}
\newcommand{\calE}{{\mathcal E}}
\newcommand{\calF}{{\mathcal F}}
\newcommand{\calG}{{\mathcal G}}
\newcommand{\calH}{{\mathcal H}}
\newcommand{\calI}{{\mathcal I}}
\newcommand{\calJ}{{\mathcal J}}
\newcommand{\calK}{{\mathcal K}}
\newcommand{\calL}{{\mathcal L}}
\newcommand{\calM}{{\mathcal M}}
\newcommand{\calN}{{\mathcal N}}
\newcommand{\calO}{{\mathcal O}}
\newcommand{\calP}{{\mathcal P}}
\newcommand{\calQ}{{\mathcal Q}}
\newcommand{\calR}{{\mathcal R}}
\newcommand{\calS}{{\mathcal S}}
\newcommand{\calT}{{\mathcal T}}
\newcommand{\calU}{{\mathcal U}}
\newcommand{\calV}{{\mathcal V}}
\newcommand{\calW}{{\mathcal W}}
\newcommand{\calX}{{\mathcal X}}
\newcommand{\calY}{{\mathcal Y}}
\newcommand{\calZ}{{\mathcal Z}}

\newcommand{\fraka}{{\mathfrak a}}
\newcommand{\frakb}{{\mathfrak b}}
\newcommand{\frakc}{{\mathfrak c}}
\newcommand{\frakd}{{\mathfrak d}}
\newcommand{\frake}{{\mathfrak e}}
\newcommand{\frakf}{{\mathfrak f}}
\newcommand{\frakg}{{\mathfrak g}}
\newcommand{\frakh}{{\mathfrak h}}
\newcommand{\fraki}{{\mathfrak i}}
\newcommand{\frakj}{{\mathfrak j}}
\newcommand{\frakk}{{\mathfrak k}}
\newcommand{\frakl}{{\mathfrak l}}
\newcommand{\frakm}{{\mathfrak m}}
\newcommand{\frakn}{{\mathfrak n}}
\newcommand{\frako}{{\mathfrak o}}
\newcommand{\frakp}{{\mathfrak p}}
\newcommand{\frakq}{{\mathfrak q}}
\newcommand{\frakr}{{\mathfrak r}}
\newcommand{\fraks}{{\mathfrak s}}
\newcommand{\frakt}{{\mathfrak t}}
\newcommand{\fraku}{{\mathfrak u}}
\newcommand{\frakv}{{\mathfrak v}}
\newcommand{\frakw}{{\mathfrak w}}
\newcommand{\frakx}{{\mathfrak x}}
\newcommand{\fraky}{{\mathfrak y}}
\newcommand{\frakz}{{\mathfrak z}}
\newcommand{\frakA}{{\mathfrak A}}
\newcommand{\frakB}{{\mathfrak B}}
\newcommand{\frakC}{{\mathfrak C}}
\newcommand{\frakD}{{\mathfrak D}}
\newcommand{\frakE}{{\mathfrak E}}
\newcommand{\frakF}{{\mathfrak F}}
\newcommand{\frakG}{{\mathfrak G}}
\newcommand{\frakH}{{\mathfrak H}}
\newcommand{\frakI}{{\mathfrak I}}
\newcommand{\frakJ}{{\mathfrak J}}
\newcommand{\frakK}{{\mathfrak K}}
\newcommand{\frakL}{{\mathfrak L}}
\newcommand{\frakM}{{\mathfrak M}}
\newcommand{\frakN}{{\mathfrak N}}
\newcommand{\frakO}{{\mathfrak O}}
\newcommand{\frakP}{{\mathfrak P}}
\newcommand{\frakQ}{{\mathfrak Q}}
\newcommand{\frakR}{{\mathfrak R}}
\newcommand{\frakS}{{\mathfrak S}}
\newcommand{\frakT}{{\mathfrak T}}
\newcommand{\frakU}{{\mathfrak U}}
\newcommand{\frakV}{{\mathfrak V}}
\newcommand{\frakW}{{\mathfrak W}}
\newcommand{\frakX}{{\mathfrak X}}
\newcommand{\frakY}{{\mathfrak Y}}
\newcommand{\frakZ}{{\mathfrak Z}}







\newcommand{\rma}{{\mathrm a}}
\newcommand{\rmb}{{\mathrm b}}
\newcommand{\rmc}{{\mathrm c}}
\newcommand{\rmd}{{\mathrm d}}
\newcommand{\rme}{{\mathrm e}}
\newcommand{\rmf}{{\mathrm f}}
\newcommand{\rmg}{{\mathrm g}}
\newcommand{\rmh}{{\mathrm h}}
\newcommand{\rmi}{{\mathrm i}}
\newcommand{\rmj}{{\mathrm j}}
\newcommand{\rmk}{{\mathrm k}}
\newcommand{\rml}{{\mathrm l}}
\newcommand{\rmm}{{\mathrm m}}
\newcommand{\rmn}{{\mathrm n}}
\newcommand{\rmo}{{\mathrm o}}
\newcommand{\rmp}{{\mathrm p}}
\newcommand{\rmq}{{\mathrm q}}
\newcommand{\rmr}{{\mathrm r}}
\newcommand{\rms}{{\mathrm s}}
\newcommand{\rmt}{{\mathrm t}}
\newcommand{\rmu}{{\mathrm u}}
\newcommand{\rmv}{{\mathrm v}}
\newcommand{\rmw}{{\mathrm w}}
\newcommand{\rmx}{{\mathrm x}}
\newcommand{\rmy}{{\mathrm y}}
\newcommand{\rmz}{{\mathrm z}}
\newcommand{\rmA}{{\mathrm A}}
\newcommand{\rmB}{{\mathrm B}}
\newcommand{\rmC}{{\mathrm C}}
\newcommand{\rmD}{{\mathrm D}}
\newcommand{\rmE}{{\mathrm E}}
\newcommand{\rmF}{{\mathrm F}}
\newcommand{\rmG}{{\mathrm G}}
\newcommand{\rmH}{{\mathrm H}}
\newcommand{\rmI}{{\mathrm I}}
\newcommand{\rmJ}{{\mathrm J}}
\newcommand{\rmK}{{\mathrm K}}
\newcommand{\rmL}{{\mathrm L}}
\newcommand{\rmM}{{\mathrm M}}
\newcommand{\rmN}{{\mathrm N}}
\newcommand{\rmO}{{\mathrm O}}
\newcommand{\rmP}{{\mathrm P}}
\newcommand{\rmQ}{{\mathrm Q}}
\newcommand{\rmR}{{\mathrm R}}
\newcommand{\rmS}{{\mathrm S}}
\newcommand{\rmT}{{\mathrm T}}
\newcommand{\rmU}{{\mathrm U}}
\newcommand{\rmV}{{\mathrm V}}
\newcommand{\rmW}{{\mathrm W}}
\newcommand{\rmX}{{\mathrm X}}
\newcommand{\rmY}{{\mathrm Y}}
\newcommand{\rmZ}{{\mathrm Z}}





\newcommand{\scrA}{{\mathscr A}}
\newcommand{\scrB}{{\mathscr B}}
\newcommand{\scrC}{{\mathscr C}}
\newcommand{\scrD}{{\mathscr D}}
\newcommand{\scrE}{{\mathscr E}}
\newcommand{\scrF}{{\mathscr F}}
\newcommand{\scrG}{{\mathscr G}}
\newcommand{\scrH}{{\mathscr H}}
\newcommand{\scrI}{{\mathscr I}}
\newcommand{\scrJ}{{\mathscr J}}
\newcommand{\scrK}{{\mathscr K}}
\newcommand{\scrL}{{\mathscr L}}
\newcommand{\scrM}{{\mathscr M}}
\newcommand{\scrN}{{\mathscr N}}
\newcommand{\scrO}{{\mathscr O}}
\newcommand{\scrP}{{\mathscr P}}
\newcommand{\scrQ}{{\mathscr Q}}
\newcommand{\scrR}{{\mathscr R}}
\newcommand{\scrS}{{\mathscr S}}
\newcommand{\scrT}{{\mathscr T}}
\newcommand{\scrU}{{\mathscr U}}
\newcommand{\scrV}{{\mathscr V}}
\newcommand{\scrW}{{\mathscr W}}
\newcommand{\scrX}{{\mathscr X}}
\newcommand{\scrY}{{\mathscr Y}}
\newcommand{\scrZ}{{\mathscr Z}}


\newcommand{\veca}{{\vec a}}
\newcommand{\vecb}{{\vec b}}
\newcommand{\vecc}{{\vec c}}
\newcommand{\vecd}{{\vec d}}
\newcommand{\vece}{{\vec e}}
\newcommand{\vecf}{{\vec f}}
\newcommand{\vecg}{{\vec g}}
\newcommand{\vech}{{\vec h}}
\newcommand{\veci}{{\vec i}}
\newcommand{\vecj}{{\vec j}}
\newcommand{\veck}{{\vec k}}
\newcommand{\vecl}{{\vec l}}
\newcommand{\vecm}{{\vec m}}
\newcommand{\vecn}{{\vec n}}
\newcommand{\veco}{{\vec o}}
\newcommand{\vecp}{{\vec p}}
\newcommand{\vecq}{{\vec q}}
\newcommand{\vecr}{{\vec r}}
\newcommand{\vecs}{{\vec s}}
\newcommand{\vect}{{\vec t}}
\newcommand{\vecu}{{\vec u}}
\newcommand{\vecv}{{\vec v}}
\newcommand{\vecw}{{\vec w}}
\newcommand{\vecx}{{\vec x}}
\newcommand{\vecy}{{\vec y}}
\newcommand{\vecz}{{\vec z}}
\newcommand{\vecA}{{\vec A}}
\newcommand{\vecB}{{\vec B}}
\newcommand{\vecC}{{\vec C}}
\newcommand{\vecD}{{\vec D}}
\newcommand{\vecE}{{\vec E}}
\newcommand{\vecF}{{\vec F}}
\newcommand{\vecG}{{\vec G}}
\newcommand{\vecH}{{\vec H}}
\newcommand{\vecI}{{\vec I}}
\newcommand{\vecJ}{{\vec J}}
\newcommand{\vecK}{{\vec K}}
\newcommand{\vecL}{{\vec L}}
\newcommand{\vecM}{{\vec M}}
\newcommand{\vecN}{{\vec N}}
\newcommand{\vecO}{{\vec O}}
\newcommand{\vecP}{{\vec P}}
\newcommand{\vecQ}{{\vec Q}}
\newcommand{\vecR}{{\vec R}}
\newcommand{\vecS}{{\vec S}}
\newcommand{\vecT}{{\vec T}}
\newcommand{\vecU}{{\vec U}}
\newcommand{\vecV}{{\vec V}}
\newcommand{\vecW}{{\vec W}}
\newcommand{\vecX}{{\vec X}}
\newcommand{\vecY}{{\vec Y}}
\newcommand{\vecZ}{{\vec Z}}


\newcommand{\boldalpha}{{\boldsymbol\alpha}}
\newcommand{\boldbeta}{{\boldsymbol\beta}}
\newcommand{\boldgamma}{{\boldsymbol\gamma}}
\newcommand{\bolddelta}{{\boldsymbol\delta}}
\newcommand{\boldepsilon}{{\boldsymbol\epsilon}}
\newcommand{\boldzeta}{{\boldsymbol\zeta}}
\newcommand{\boldeta}{{\boldsymbol\eta}}
\newcommand{\boldtheta}{{\boldsymbol\theta}}
\newcommand{\boldiota}{{\boldsymbol\iota}}
\newcommand{\boldkappa}{{\boldsymbol\kappa}}
\newcommand{\boldlambda}{{\boldsymbol\lambda}}
\newcommand{\boldmu}{{\boldsymbol\mu}}
\newcommand{\boldnu}{{\boldsymbol\nu}}
\newcommand{\boldxi}{{\boldsymbol\xi}}
\newcommand{\boldomicron}{{\boldsymbol o}}
\newcommand{\boldpi}{{\boldsymbol\pi}}
\newcommand{\boldrho}{{\boldsymbol\rho}}
\newcommand{\boldsigma}{{\boldsymbol\sigma}}
\newcommand{\boldtau}{{\boldsymbol\tau}}
\newcommand{\boldupsilon}{{\boldsymbol\upsilon}}
\newcommand{\boldphi}{{\boldsymbol\phi}}
\newcommand{\boldchi}{{\boldsymbol\chi}}
\newcommand{\boldpsi}{{\boldsymbol\psi}}
\newcommand{\boldomega}{{\boldsymbol\omega}}




\begin{document}






        \item 
        % \todo{Combine all this to get the final transformation.}
        The last simplicial complex that we introduce is called $\calK$,
        and it is the simplicial complex obtained from $\patch_{\calT}(S)$ via barycentric refinement of $S$. 
        
        We let $\calK$ the simplicial complex obtained from $\patch_{\calT}(S)$ via barycentric refinement of $S$. 
        Using our initial observations, we obtain homeomorphisms 
        \begin{align*}
            \Psi_{\calK } : |\calK| \rightarrow \calB,
            \qquad 
            \Psi_{\calR'} : |\calR'| \rightarrow \calB.
        \end{align*}
        Both $\Psi_{\calK}$ and $\Psi_{\calR'}$ map $T$ onto the same subset of the unit ball.
        Consequently, the image under $\Psi_{\calK}$ of $\overline{|\calK| \setminus T}$
        onto the image under $\Psi_{\calR'}$ of $|\calR^{c}|$.
        It follows that 
        \begin{align*}
            \Xi_{S} := \Psi_{\calK}^{-1} \Psi_{\calR'} \Theta
        \end{align*}
        is a mapping from $T$ onto $\overline{\patch_{\calT}(S) \setminus T}$.
        We estimate the Jacobians: 
        \begin{align*}
            \| \Jacobian \Xi_{S} \| 
            \leq 
            \sup_{ K \in \calK  }
            \left( 1 + n \kappa(A_{K})^{2} \right)^{\frac 1 2}
            \sigma_{\max}(A_{K}) 
            %
            \sup_{ K \in \calR' } 
            \sqrt{2n} 
            \sigma_{\max}(A_{K}^{-1}) 
            %
            \sup_{ K \in \calR  } \kappa(A_{K})
            \\
            \| \Jacobian \Xi_{S}^{-1} \| 
            \leq 
            \sup_{ K \in \calK  }
            \sqrt{2n} 
            \sigma_{\max}(A_{K}^{-1}) 
            %
            \sup_{ K \in \calR' } 
            \left( 1 + n \kappa(A_{K})^{2} \right)^{\frac 1 2}
            \sigma_{\max}(A_{K}) 
            %
            \sup_{ K \in \calR  } \kappa(A_{K})
            .
        \end{align*}
        \color{red}
        Each bisection step reduces the diameter by at most a factor of one half,
        and it reduces the volume always by a factor of one half. 
        Via the estimates in Lemma~\ref{lemma:affinetransform}, we thus find 
        \begin{align*}
            \| \Jacobian \Xi_{S} \| 
            &\leq 
            2 \left( \sqrt{n} + 1 \right) 
            \sup_{ K \in \calK  }
            \Ceins{n} \diam(K) 
            \sup_{ K \in \calR' } 
            \Czwei{n} \kappa(K) \diam(K)^{-1}
            \sup_{ K \in \calR  } \kappa(A_{K})
            \\&\leq 
            32 \left( \sqrt{n} + 1 \right) 
            \theta(\calT)
            \Ceins{n} 
            \Czwei{n} 
            \kappa(\calT) 
            \alpha(\calT)
            \\
            \| \Jacobian \Xi_{S}^{-1} \| 
            &\leq 
            2 \left( \sqrt{n} + 1 \right) 
            \sup_{ K \in \calK  }
            \Czwei{n} \kappa(K) \diam(K)^{-1}
            \sup_{ K \in \calR' } 
            \Ceins{n} \diam(K)
            \sup_{ K \in \calR  } \kappa(A_{K})
            \\
            &\leq 
            32 \left( \sqrt{n} + 1 \right) 
            \theta(\calT)
            \Ceins{n} 
            \Czwei{n} 
            \kappa(\calT) 
            \alpha(\calT)
            .
        \end{align*}
        Together with the definition of $\theta(\calT)$, the desired result follows. 
%         \color{red}
%         Each bisection halves the volume of the triangles. Thus,
%         \begin{align*}
%             \| \Jacobian \Xi_{S} \| 
%             &\leq 
%             2 
%             \sup_{ K \in \patch_{\calT}(S) }
%             2 \sigma_{\max}(A_{K}) 
%             4 
%             \left( \sqrt{n} + 1 \right) \sigma_{\max}(A_{T}^{-1}) 
%             4 
%             \kappa(A_{T})
%             \\&\leq 
%             64
%             \left( \sqrt{n} + 1 \right) 
%             \kappa(A_{T})
%             \sigma_{\max}(A_{T}^{-1}) 
%             \sup_{ K \in \patch_{\calT}(S) }
%             \sigma_{\max}(A_{K}     ) 
%             \\
%             \| \Jacobian \Xi_{S}^{-1} \| 
%             &\leq 
%             2 \left( \sqrt{n} + 1 \right) 
%             \sup_{ K \in \patch_{\calT}(S) }
%             \sigma_{\max}(A_{K}) 
%             \sigma_{\max}(A_{T}^{-1}) 
%             4 \kappa(A_{T})
%             \\&\leq 
%             64
%             \left( \sqrt{n} + 1 \right) 
%             \kappa(A_{T})
%             \sigma_{\max}(A_{T}^{-1}) 
%             \sup_{ K \in \patch_{\calT}(S) }
%             \sigma_{\max}(A_{K}     ) 
%             .
%         \end{align*}
%         We use 
%         \begin{gather*}
%             \sup_{ K \in \patch_{\calT}(S) }
%             \sigma_{\max}(A_{K}) 
%             \leq 
%             \Ceins{n} 
%             \sup_{ K \in \patch_{\calT}(S) }
%             \diam(K)
%             \leq 
%             \Ceins{n} 
%             \theta(\calT)
%             \sigma_{\max}(A_{T}) 
%             .
%         \end{gather*}
%         \color{black}
        % That establishes the final estimate. 
%         \item 
%         Suppose that $S = \{v_0\}$ is a single vertex. 
%         We construct a transformation from the vertex star around $S$ onto the Euclidean unit ball. 
%         Without loss of generality, $S$ is at the origin. 
%         For any $T \in \patch_{\calT}(S)$, we let $F_T$ be the transformation. 
%         We define $F$ over $|\patch_{\calT}(S)|$ piecewise. 
%         \todo{Complete}
%         
        
%         \item 
%         Let $\calU$ be the simplicial complex of dimension $n-1$ that contains precisely those facets of $T$ that do not contain $S$.
%         We write $Y := F( |\calU| )$. 
%         \todo{Shape of the outer patch $Y$, star-shaped geometry}
%         
%         \item 
%         We let $\sphere{n}_{\star} := \sphere{n} \setminus \{e_{n+1}\} \subseteq \bbR^{n+1}$ be the unit sphere without the North pole. 
%         Suppose that $\bar x \in \sphere{n}$ is a point on the unit sphere and $\bar x \neq e_{n+1}$. 
%         We define its stereographic projection as 
%         \begin{align*}
%             \pi( \bar x ) 
%             = 
%             \sum_{i=1}^{n} \frac{x_i}{1 - x_{n+1}} e_{i}
%             .
%         \end{align*}
%         This is a mapping $\pi : \sphere{n}_{\star} \rightarrow \bbR^{n}$. 
%         Its inverse is defined as follows: given $y \in \bbR^{n}$, we have 
%         \begin{align*}
%             \pi^{-1}( y )
%             = 
%             \sum_{i=1}^{n} \frac{2 y_i}{1 + \| y \|^2} e_{i}
%             +
%             \frac{1 - \| y \|^2}{1 + \| y \|^2} e_{n+1}
%             .
%         \end{align*}
%         \item 
%         Mirroring the unit sphere along the equator is expressed in stereographic coordinates by the involution 
%         \begin{align*}
%             \mu : \bbR^{n} \rightarrow \bbR^{n},
%             \qquad 
%             y \mapsto \frac{y}{\| y \|}.
%         \end{align*}
%         \item 
%         \begin{align*}
%             G(y) := \| y \| \cdot \phi\left( \frac{y}{\|y\|} \right)^{-1} \frac{y}{\|y\|}
%         \end{align*}
%         \item 
%         We combine these mappings:
%         \begin{align*}
%             Q : \sphere{n} \rightarrow \sphere{n}, \qquad x \mapsto \pi^{-1} G^{-1} \mu G \pi(x)
%         \end{align*}
%         We want to analyze its Lipschitz properties. \todo{Stuff}
%         \item
%         We have constructed a bi-Lipschitz mapping from the Euclidean unit sphere into itself.
%         We use the following result~\cite[Theorem~3]{alestalo2018radial}:
%         if $f : \sphere{n} \rightarrow \sphere{n}$ is any $L$-Lipschitz mapping from the Euclidean unit sphere into itself, 
%         then the radial extension $f : \ball{n} \rightarrow \ball{n}$ is an $L$-Lipschitz mapping from the Euclidean unit ball into itself.
%         Consequently, \todo{write out}
%         \item 
%         In total, we thus have 
%         \begin{align*}
%             Q' := F^{-1} \pi^{-1} G^{-1} \mu G \pi F
%             .
%         \end{align*}




        \item 
        We begin with the following preparatory observations. 
        Consider any $n$-simplex $T$ with vertices $v_0, v_1, \dots, v_n$, where $v_0$ is the origin. 
        We let $A_{T} \in \bbR^{n \times n}$ be the matrix that maps the $i$-th unit vector to $v_i$.
        That matrix maps the reference simplex $\Delta^{n}$ onto $T$. 
        Then the mapping 
        \begin{align}
            F_T(x) 
            = 
            \frac{ \| A_{T}^{-1} x \|_{1} }{ \|x\|_2 } x
        \end{align}
        maps $T$ onto a ``simplex with a curved face'' inside the Euclidean unit ball. Its inverse is
        \begin{align}
            F_T^{-1}(x) 
            = 
            \frac{ \|x\|_2 }{ \| A_{T}^{-1} x \|_{1} } x
            .
        \end{align}
        We analyze its Lipschitz properties.
        % 
        In what follows, $y \in \bbR^{n}$. 
        On the one hand, 
        we compute the Jacobian at any $x \in \mathring T$:
        \begin{align*}
            \Jacobian F_T(x) \cdot y
            &= 
            \frac{ \|x\| y - x \|x\|^{-1} x^t y }{\|x\|^2} 
            \left( \vecone \cdot A_{T}^{-1} x \right)
            + 
            \frac{x}{\|x\|} \left( \vecone \cdot A_{T}^{-1} y \right)
            \\&= 
            \left( \Id - \|x\|^{-2} x \otimes x^t \right)
            \frac{ 1 }{ \|x\| } 
            \left( \vecone \cdot A_{T}^{-1} x \right)
            + 
            \frac{x}{\|x\|} \left( \vecone \cdot A_{T}^{-1} x \right)
            \\&= 
            \left( A_{T}^{-t} \vecone \cdot \hat x \right)
            \left( \Id - \hat x \otimes \hat x^t \right) y
            + 
            \left( A_{T}^{-t} \vecone \cdot y \right) \hat x 
            .
        \end{align*}
        This is an orthogonal sum of two vectors. Its norm satisfies:
        \begin{align*}
            \| \Jacobian F_T(x) \cdot y \|^{2}
            &= 
            \left| A_{T}^{-t} \vecone \cdot \hat x \right|^{2}
            \| \left( \Id - \hat x \otimes \hat x^t \right) y \|^{2}
            + 
            \left| A_{T}^{-t} \vecone \cdot y \right|^{2} \| \hat x \|^{2}
            .
        \end{align*}
        The Euclidean operator norm satisfies 
        \begin{align*}
            \| \Jacobian F_T(x) \| 
            \leq 
            \sqrt 2 \left\| A_{T}^{-t} \vecone \right\|_{2}
            \leq 
            \sqrt{2n} \cdot \sigma_{\max}(A_{T}^{-1}) 
            .
        \end{align*}
        On the other hand, we can rewrite the Jacobian as:
        \begin{align*}
            \Jacobian F_T(x) \cdot y
            &= 
            \left( A_{T}^{-t} \vecone \cdot \hat x \right)
            y
            - 
            \left( A_{T}^{-t} \vecone \cdot \hat x \right)
            \left( \hat x \otimes \hat x^t \right) y
            + 
            \left( A_{T}^{-t} \vecone \cdot y \right) \hat x 
            \\&= 
            \left( A_{T}^{-t} \vecone \cdot \hat x \right)
            y
            - 
            \hat x 
            \left( 
                \left( A_{T}^{-t} \vecone \cdot \hat x \right) \hat x^{t}
                -
                \vecone^{t} A_{T}^{-1} 
            \right) y
            \\&= 
            \left( A_{T}^{-t} \vecone \cdot \hat x \right)
            y
            - 
            \hat x 
            \left( 
                \left( A_{T}^{-t} \vecone \cdot \hat x \right) \hat x
                -
                A_{T}^{-t} \vecone
            \right)^{t} y
            .
        \end{align*}
        Its inverse is found via the Woodbury formula: 
        % Its inverse is easily verified to be:
        \begin{align*}
            \Jacobian F_T(x)^{-1}
            &= 
            \left( A_{T}^{-t} \vecone \cdot \hat x \right)^{-1}
            \Id
            - 
            \left( A_{T}^{-t} \vecone \cdot \hat x \right)^{-2}
            \hat x 
            \left( 
                \left( A_{T}^{-t} \vecone \cdot \hat x \right) \hat x
                -
                A_{T}^{-t} \vecone
            \right)^{t} 
            \\&= 
            \left( A_{T}^{-t} \vecone \cdot \hat x \right)^{-1}
            \left( 
                \Id
                - 
                \hat x 
                \left( 
                    % \left( A_{T}^{-t} \vecone \cdot \hat x \right)^{-1}
                    % \left( A_{T}^{-t} \vecone \cdot \hat x \right) 
                    \hat x
                    -
                    \left( A_{T}^{-t} \vecone \cdot \hat x \right)^{-1}
                    A_{T}^{-t} \vecone
                \right)^{t}
            \right)
            \\&= 
            \left( \vecone \cdot A_{T}^{-1} \hat x \right)^{-1}
            \left( 
                \Id
                - 
                \hat x 
                \left( 
                    % \left( A_{T}^{-t} \vecone \cdot \hat x \right)^{-1}
                    % \left( A_{T}^{-t} \vecone \cdot \hat x \right) 
                    \hat x
                    -
                    \left( \vecone \cdot A_{T}^{-1} \hat x \right)^{-1}
                    A_{T}^{-t} \vecone
                \right)^{t}
            \right)
            \\&= 
            \left( \vecone \cdot A_{T}^{-1} \hat x \right)^{-1}
            \left( 
                \Id - \hat x \otimes \hat x^{t}
                + 
                \left( \vecone \cdot A_{T}^{-1} \hat x \right)^{-1}
                \hat x \otimes \vecone^{t} A_{T}^{-1}
            \right)
            .
        \end{align*}
        The inverse acts via multiplication with $\left( \vecone \cdot A_{T}^{-1} \hat x \right)^{-1}$ on the span of $\hat x$.
        We notice the upper bound 
        \begin{align*}
            \left( \vecone \cdot A_{T}^{-1} \hat x \right)^{-1}
            &= 
            \| A_{T}^{-1} \hat x \|^{-1}_{1}
            \\&\leq  
            \| A_{T}^{-1} \hat x \|^{-1}_{2}
            \leq 
            \sigma_{\min}( A_{T}^{-1} )^{-1}
            = 
            \sigma_{\max}( A_{T} )
            .
        \end{align*}
        This holds because $A_{T}^{-1} \hat x$ is in the positive quadrant.
        %, and its scalar product with $\vecone$ is minimized when it is colinear with a coordinate axis.
        If $y \in \bbR^{n}$ is orthogonal to $\hat x$, then 
        \begin{align*}
            \| \Jacobian F_T(x)^{-1} \cdot y \|^{2}
            &\leq 
%             \left( \vecone \cdot A_{T}^{-1} \hat x \right)^{-2}
%             \left\| 
%                 y
%             \right\|^{2}
%             +
%             \left( \vecone \cdot A_{T}^{-1} \hat x \right)^{-2}
%             \left| 
%                 \left( \vecone \cdot A_{T}^{-1} \hat x \right)^{-1}
%                 \vecone^{t} A_{T}^{-1} y 
%             \right|^{2}
%             \\&\leq 
            \left( \vecone \cdot A_{T}^{-1} \hat x \right)^{-2}
            \left\| 
                y
            \right\|^{2}
            +
            \frac{ 
            \left| 
                \left( \vecone \cdot A_{T}^{-1} \hat x \right)^{-1}
                \vecone^{t} A_{T}^{-1} y 
            \right|^{2}
            }{
            \left( \vecone \cdot A_{T}^{-1} \hat x \right)^{2}
            }
            .
        \end{align*}
        The operator norm of that inverse is:
        \begin{align*}
            \| \Jacobian F_T(x)^{-1} \|
            \leq 
            \sigma_{\max}( A_{T} )
            \left( 1 + n \frac{ \sigma_{\max}(A_{T})^{2} }{ \sigma_{\min}(A_{T})^{2} } \right)^{\frac 1 2}
            .
        \end{align*}
        Since the domain and codomain of $F_{T}$ are convex, 
        this also implies bounds on the Lipschitz constants of $F_{T}$ and its inverse. 










        
\color{red}
        On the one hand, given $x, y \in T$, 
        where $\| y \|_{1} \leq \| x \|_{1}$ without loss of generality, we find 
        \begin{align*}
            &
            \left\| F_T(x) - F_T(y) \right\|_{2}
            \\&\qquad 
            \leq 
            \left\| 
                \frac{ \| A_{T}^{-1} x \|_{1} }{ \|x\|_2 } x - \frac{ \| A_{T}^{-1} y \|_{1} }{ \|y\|_2 } y
            \right\|_{2}
            \\&\qquad 
            \leq 
            \left\| 
                \frac{ \| A_{T}^{-1} x \|_{1} }{ \|x\|_2 } x - \frac{ \| A_{T}^{-1} x \|_{1} }{ \|x\|_2 } y
            \right\|_{2}
            +
            \left\| 
                \frac{ \| A_{T}^{-1} x \|_{1} }{ \|x\|_2 } y - \frac{ \| A_{T}^{-1} y \|_{1} }{ \|y\|_2 } y
            \right\|_{2}
            \\&\qquad 
            \leq 
            \frac{ \| A_{T}^{-1} x \|_{1} }{ \|x\|_2 } 
            \left\| 
                x - y
            \right\|_{2}
            +
            \left| \|A_{T}^{-1} x\|_1 - \|A_{T}^{-1} y\|_1 \right|
            \\&\qquad 
            \leq 
            2\sqrt{n} \sigma_{\max}(A_{T}^{-1}) 
            \left\| x - y \right\|_{2}
            .
        \end{align*}
        On the one hand, given $x, y \in T$, 
        where $\| x \|_{2} \leq \| y \|_{2}$ without loss of generality, we find 
        \begin{align*}
            &
            \left\| F_T(x) - F_T(y) \right\|_{2}
            \\&\qquad 
            \leq 
            \left\| 
                \frac{ \| A_{T}^{-1} x \|_{1} }{ \|x\|_2 } x - \frac{ \| A_{T}^{-1} y \|_{1} }{ \|y\|_2 } y
            \right\|_{2}
            \\&\qquad 
            \leq 
            \left\| 
                \frac{ \| A_{T}^{-1} x \|_{1} }{ \|x\|_2 } x - \frac{ \| A_{T}^{-1} x \|_{1} }{ \|x\|_2 } y
            \right\|_{2}
            +
            \left\| 
                \frac{ \| A_{T}^{-1} x \|_{1} }{ \|x\|_2 } y - \frac{ \| A_{T}^{-1} y \|_{1} }{ \|x\|_2 } y
            \right\|_{2}
            +
            \left\| 
                \frac{ \| A_{T}^{-1} y \|_{1} }{ \|x\|_2 } y - \frac{ \| A_{T}^{-1} y \|_{1} }{ \|y\|_2 } y
            \right\|_{2}
            \\&\qquad 
            \leq 
            \frac{ \| A_{T}^{-1} x \|_{1} }{ \|x\|_2 } 
            \left\| 
                x - y
            \right\|_{2}
            +
            \left| \|A_{T}^{-1} x\|_1 - \|A_{T}^{-1} y\|_1 \right|
            \frac{\|y\|}{\|x\|}
            +
            \|A_{T}^{-1} y\|_1 \|y\|_{2}
            \left| \frac{1}{\|x\|} - \frac{1}{\|y\|} \right|
            \\&\qquad 
            \leq 
            \frac{ \| A_{T}^{-1} x \|_{1} }{ \|x\|_2 } 
            \left\| 
                x - y
            \right\|_{2}
            +
            \left| \|A_{T}^{-1} x\|_1 - \|A_{T}^{-1} y\|_1 \right|
            \frac{\|y\|}{\|x\|}
            +
            \|A_{T}^{-1} y\|_1 \|y\|_{2}
            \left| \frac{ \|y\|_{2} - \|x\|_{2} }{\|x\|_{2} \|y\|_{2}} \right|
            \\&\qquad 
            \leq 
            3 \sqrt{n} \sigma_{\max}(A_{T}^{-1}) 
            \left\| x - y \right\|_{2}
            .
        \end{align*}
        On the other hand, given $x, y \in F_T(T)$, 
        where $\| A_{T}^{-1} y \|_{1} \leq \| A_{T}^{-1} x \|_{1}$ without loss of generality, we find 
        \begin{align*}
            &
            \left\| F_T^{-1}(x) - F_T^{-1}(y) \right\|_{2}
            \\&\qquad 
            \leq 
            \left\| 
                \frac{ \|x\|_2 }{ \| A_{T}^{-1} x \|_{1} } x - \frac{ \|y\|_2 }{ \| A_{T}^{-1} y \|_{1} } y
            \right\|_{2}
            \\&\qquad 
            \leq 
            \left\| 
                \frac{ \|x\|_2 }{ \| A_{T}^{-1} x \|_{1} } x - \frac{ \|x\|_2 }{ \| A_{T}^{-1} x \|_{1} } y
            \right\|_{2}
            +
            \left\| 
                \frac{ \|x\|_2 }{ \| A_{T}^{-1} x \|_{1} } y - \frac{ \|y\|_2 }{ \| A_{T}^{-1} x \|_{1} } y
            \right\|_{2}
            +
            \left\| 
                \frac{ \|y\|_2 }{ \| A_{T}^{-1} x \|_{1} } y - \frac{ \|y\|_2 }{ \| A_{T}^{-1} y \|_{1} } y
            \right\|_{2}
            \\&\qquad 
            \leq 
            \frac{ \|x\|_2 }{ \| A_{T}^{-1} x \|_{1} } 
            \left\| 
                x - y
            \right\|_{2}
            +
            \left| \|x\|_2 - \|y\|_2 \right|
            \frac{ \|y\|_2 }{ \| A_{T}^{-1} x \|_{1} }
            +
            \|y\|_2^{2}
            \left| 
            \frac{ \| A_{T}^{-1} y \|_{1} - \| A_{T}^{-1} x \|_{1} }{ \| A_{T}^{-1} x \|_{1} \| A_{T}^{-1} y \|_{1} }
            \right|
            \\&\qquad 
            \leq 
            \frac{ \|x\|_2 }{ \| A_{T}^{-1} x \|_{2} } 
            \left\| 
                x - y
            \right\|_{2}
            +
            \left| \|x\|_2 - \|y\|_2 \right|
            \frac{ \|y\|_2 }{ \| A_{T}^{-1} x \|_{2} }
            +
            \|y\|_2^{2}
            \left| 
            \frac{ \| A_{T}^{-1} y \|_{1} - \| A_{T}^{-1} x \|_{1} }{ \| A_{T}^{-1} x \|_{2} \| A_{T}^{-1} y \|_{2} }
            \right|
            \\&\qquad 
            \leq 
            \left( 2 + \sqrt{n} \frac{\sigma_{\max}(A_{T}^{-1}) }{ \sigma_{\min}(A_{T}^{-1}) } \right) \sigma_{\min}(A_{T}^{-1})^{-1} 
            \left\| x - y \right\|_{2}
            \\&\qquad 
            \leq 
            \left( 2 + \sqrt{n} \kappa(A_{T}) \right) \sigma_{\max}(A_{T})
            \left\| x - y \right\|_{2}
            .
        \end{align*}
        By convexity of the range of $F_T(T)$, 
        we obtain a bound on the operator norm of the Jacobian of $F_{T}^{-1}$. 
        We recall that $\sigma_{\max}(A_{T}^{-1}) = \sigma_{\min}(A_{T})^{-1}$ and $\sigma_{\min}(A_{T}^{-1})^{-1} = \sigma_{\max}(A_{T})$.
        \color{black}
        











\begin{proof}
    We let $\hat F_0$ be the convex closure of the first $n$ unit vectors. 
    We let $\hat F_1, \dots, \hat F_n$ be the remaining faces of the reference simplex 
    such that the $i$-th coordinate equals zero along $\hat F_i$.

    Let $\varphi : \Delta^{n} \rightarrow T$ be an affine diffeomorphism 
    that maps the faces $\hat F_{0}, \hat F_{1}, \dots, \hat F_{l}$ onto the boundary patch $\Gamma$. 
    We let $\Upsilon^{n}$ be constructed by reflecting $\Delta^{n}$ across the coordinate axes $l+1$ through $n$.
    The boundary $\partial\Upsilon^{n}$ is the reflection of the faces $\hat F_{0}, \hat F_{1}, \dots, \hat F_{l}$ under that process. 
    Obviously, $\Upsilon^{n}$ is convex with diameter $2$. 


    Let $f = \cartan u$ with $u \in \bfW^{p}\Lambda^{k}(T)$.
    We set $\hat f := \varphi^{\ast} f$ and $\hat u = \varphi^{\ast} u$, 
    so that $\hat u \in \cartan \bfW^{p}\Lambda^{k}(\Delta^{n})$ and $\cartan \hat u = \hat f$.

    We construct $\tilde u \in L^{p}(\Upsilon^{n})$ as the reflection $\hat u$ across the coordinate axes $l+1$ through $n$, and we build $\tilde f \in L^{p}(\Upsilon^{n})$ analogously.
    Then $\tilde u \in \bfW^{p}(\Upsilon^{n})$ with $\cartan \tilde u = \tilde f$. 

    The operator norm bound of the potential operator shows 
    \begin{align*}
        \| \tilde u \|_{L^{p}(\Upsilon^{n})} 
        &\leq 
        n A(n,\ell) \frac{ \vol_{n}(B_2(0)) }{|\Upsilon^{n}|} 2 
        \| \tilde f \|_{L^{p}(\Upsilon^{n})}
        \\&
        \leq 
        2 n A(n,\ell) \vol_{n}(B_1(0)) \frac{ n! \cdot 2^{n} }{2^{n-l}} 
        \| \tilde f \|_{L^{p}(\Upsilon^{n})}
        .
    \end{align*}
    We use the pullback estimates 
    \begin{gather*}
        \| \tilde u \|_{L^{p}(\Upsilon^{n})} 
        \leq 
        \| \Jacobian\varphi \|_{\infty}^{k}
        |\det\varphi^{-\frac{1}{p}}|
        \cdot 
        \| \tilde u \|_{L^{p}(\Upsilon^{n})} 
        ,
        \\
        \| \tilde f \|_{L^{p}(\Upsilon^{n})} 
        \leq 
        2^{n-l}
        \| \hat f \|_{L^{p}(\hat T)} 
        \leq 
        2^{n-l}
        \| \Jacobian\varphi^{-1} \|_{\infty}^{k+1}
        |\det\varphi^{\frac{1}{p}}|
        \cdot 
        \| f \|_{L^{p}(T)} 
        .
    \end{gather*}
\end{proof}





\begin{remark}
    For reference, how the extension should work:
    suppose that $u \in H(\Omega)$ and let $f = du$.
    Suppose we have $u_1 \in H(T_1)$ with $du_1 = du$ over $T_1$. 
    We built $u'_2 \in H(T_2)$ such that 
    \begin{align*}
        \trace_F u_1 = \trace_F u'_2
    \end{align*}
    We let $u''_2 \in H(T_2)$ such that 
    \begin{align*}
        d u''_2 = f - d u'_2, \quad \trace_F u''_2 = 0
    \end{align*}
    Such $u''_2$ exists because 
    \begin{align*}
        \trace_F( f_2 - d u'_2 ) 
        &= 
        \trace_F f_1 - \trace_F d u'_2
        \\&= 
        \trace_F f_1 - \trace_F d u'_2
        \\&= 
        \trace_F d u_1 - \trace_F d u'_2
        \\&= 
        d \trace_F u_1 - d \trace_F u'_2
        \\&= 
        d \trace_F u_1 - d \trace_F u_1
        = 0
        .
    \end{align*}
    Setting $u_2 = u'_2 + u''_2$, we find 
    \begin{align*}
        d u_2 = d u'_2 + d u''_2 = d u'_2 + f - d u'_2 = f,
        \\
        \trace_F u_2 = \trace_F u'_2 = \trace_F u_1.
    \end{align*}
    That requires an extension from given partial boundary data. 
    Possible approaches:
    \begin{itemize}
        \item Reflect a neighborhood and multiply with cut-off function.
        \item Take traces from patch and extend onto new simplex. 
    \end{itemize}
\end{remark}








\begin{lemma}\label{lemma:mixedbconsimplex}
    Let $T$ be an $n$-simplex and with a face $F$. 
    If $u \in W^{1,p}(T)$ such that $\trace_{F} u = 0$, then 
    \begin{align*}
        \| u \|_{L^{p}(T)}
        &
        \leq 
        C_{{\rm PF},T,F,p} \| \nabla u \|_{L^{p}(T)}
    \end{align*}
    where $C_{{\rm PF},T,F,p} > 0$ is a constant such that 
    \begin{align*}
        C_{{\rm PF},T,F,p}
        \leq 
        \left( C_{{\rm PF},T,p} + \left( C_{{\rm PF},T,p}^{p} + p \cdot \diam(T) C_{{\rm PF},T,p}^{p-1} \right)^{\frac 1 p} \right),  
        \quad 
        C_{{\rm PF},T,F,p}
        \leq 
        \diam(T)
        .
    \end{align*}
\end{lemma}
\begin{proof}
    There exists $w \in W^{1,p}(T)$ with $\nabla w = \nabla u$ and 
    \[
        \| w \|_{L^{p}(T)}
        \leq 
        C_{{\rm PF},T,p} 
        \| \nabla w \|_{\bfL^{p}(T)}
        .
    \]
    Then $w-u$ is constant, and thus $\gamma := \trace_{F} (w-u) = \trace_{F} w$. 
    We use that 
    \[
        \| \gamma \|_{L^{p}(T)}
        =
        \gamma \vol(T)^\frac{1}{p}
        =
        \left( \frac{ \vol(T) }{ \vol(F) } \right)^\frac{1}{p}
        \cdot 
        \gamma 
        \vol(F)^\frac{1}{p}
        \leq 
        \left( \frac{ \vol(T) }{ \vol(F) } \right)^\frac{1}{p}
        \| \gamma \|_{L^{p}(F)}
        .
    \]
    Using a trace inequality~\cite[Lemma~2.8]{veeser2012poincare} when $1 \leq p < \infty$, we find 
    \begin{align*}
        \| \gamma \|_{L^{p}(F)}^{p}
        &
        \leq 
        \frac{ \vol(F) }{ \vol(T) }
        \| w \|_{L^{p}(T)}^{p}
        +
        p
        \cdot 
        \diam(T)
        \frac{ \vol(F) }{ \vol(T) }
        \| w \|_{L^{p}(T)}^{p-1}
        \| \nabla w \|_{L^{p}(T)}
        \\&
        \leq 
        \left( \frac{ \vol(F) }{ \vol(T) } \right)
        \left( C_{{\rm PF},T,p}^{p} + p \cdot C_{{\rm PF},T,p}^{p-1} \right) 
        \diam(T)^{p}
        \| \nabla w \|_{L^{p}(T)}^{p}
        \\&
        \leq 
        \left( \frac{ \vol(F) }{ \vol(T) } \right)
        \left( C_{{\rm PF},T,p}^{p} + p \cdot \diam(T) C_{{\rm PF},T,p}^{p-1} \right) 
        %\diam(T)^{p}
        \| \nabla w \|_{L^{p}(T)}^{p}
        .
    \end{align*}
    Recall that $u = w - \gamma$. The first inequality follows. 
    
    The second inequality follows from Rademachers theorem when $p = \infty$.
    Let assume that $1 \leq p < \infty$. 
    Suppose that $u \in C^{\infty}(T)$ with support disjoint from $F$.
    Without loss of generality, $F$ lies in the span of the first $n-1$ coordinates and the $n$-th coordinate is non-negative over $T$. 
    We write $\bfg \in \bfL^{\infty}(T)$ be the trivial extension of $u$ outside of $T$.
    With the fundamental theorem of calculus and H\"olders inequality: 
    \begin{align*}
        &
        \int_{\Omega} |u(x_{1},\dots,x_{n-1},x_{n})|^{p} dx_1 \dots dx_{n-1} dx_{n}
        \\&
        \leq
        \int_{\Omega} \left| \int_{0}^{x_n} \partial_{n} u(x_{1},\dots,x_{n-1},y) dy \right|^{p} dx_1 \dots dx_{n-1} dx_{n}
        \\&
        \leq
        \int_{\Omega} \int_{0}^{x_n} x_{n}^{p-1} \left| \partial_{n} u(x_{1},\dots,x_{n-1},y) \right|^{p} dy dx_1 \dots dx_{n-1} dx_{n}
        \\&
        \leq
        \int_{\Omega} \int_{0}^{D} D^{p-1} \left| \bfg(x_{1},\dots,x_{n-1},y) \right|^{p} dy dx_1 \dots dx_{n-1} dx_{n}
        \leq 
        D^{p} 
        \int_{\Omega} |\nabla u|^{p} dx
        .
    \end{align*}
    That $\|u\|_{L^{p}(\Omega)} \leq \diam(T) \|\nabla u\|_{L^{p}(\Omega)}$ for all $W^{1,p}(T)$ whose members have vanishing trace along $F$ follows from approximation via members of $u \in C^{\infty}(T)$ whose support is disjoint from $F$. 
    
    Very briefly verify that density argument: 
    There exists $\varphi : \hat T \rightarrow T_{m+1}$ mapping the convex closure of the $n$ unit vectors onto the face $F$.
    We let $\hat u := u \circ \varphi$ and define $\hat g \in \bfL^p(\hat T)$ via $\hat g := \nabla ( u \circ \varphi )$. 
    Then $\nabla \hat u = \hat g$. 
    % 
    Let $\hat U$ be the unit ball of the $\ell^1$ metric, which contains $\hat T$.
    We let $\tilde u$ be the extension of $\hat u$ onto $\hat U$ by reflection across the coordinate axes.
    We define $\tilde g = \nabla \tilde u$
    and see that $\tilde g$ is the extension of $\hat g$ onto $\hat U$ by reflection across the coordinate axes. 
    By construction $\tilde u$ vanishes at the boundary of $\hat U$,
    and we approximate $\tilde u$ in $W^{1,p}(\hat U)$ via a sequence in $C^{\infty}_{c}(\hat U)$. 
    
    
    There exists $\varphi : \hat T \rightarrow T_{m+1}$ mapping the convex closure of the $n$ unit vectors onto the face $F$.
    We let $\hat u := u \circ \varphi$ and define $\hat g \in \bfL^p(\hat T)$ via $\hat g := \nabla ( u \circ \varphi )$. 
    Then $\nabla \hat u = \hat g$. 
    % 
    Let $\hat U$ be the unit ball of the $\ell^1$ metric, which contains $\hat T$.
    We let $\tilde u$ be the extension of $\hat u$ onto $\hat U$ by reflection across the coordinate axes.
    We define $\tilde g = \nabla \tilde u$
    and see that $\tilde g$ is the extension of $\hat g$ onto $\hat U$ by reflection across the coordinate axes. 
    By construction $\tilde u$ vanishes at the boundary of $\hat U$, and thus Friedrichs inequality satisfies 
    \begin{gather*}
        \| \tilde u \|_{L^{p}(\hat U)} 
        \leq 
        2 
        \| \tilde g \|_{\bfL^{p}(\hat U)}
        .
    \end{gather*}
    In addition to that, 
    \begin{gather*}
        \| u \|_{L^{p}(T_{m+1})}
        \leq 
        |\det(\Dif \varphi)|^{\frac 1 p} 
        \| \hat u \|_{L^{p}(\hat T)}
        \leq 
        |\det(\Dif \varphi)|^{\frac 1 p} 
        \| \tilde u \|_{L^{p}(\hat U)}
        ,
        \\
        \| \tilde g \|_{\bfL^{p}(\hat U)}
        \leq 
        2^{\frac n p}
        \| \hat g \|_{\bfL^{p}(\hat T)}
        \leq 
        2^{\frac n p}
        |\det(\Dif \varphi_1)|^{-\frac 1 p} \| \Dif \varphi \| \cdot 
        \| \nabla u \|_{\bfL^{p}(T)}
        .
    \end{gather*}
    The second inequality follows.
\end{proof}







Given $a \in \Omega$, we introduce the potential operator 
\begin{align*}
    \Poinc_{\ell,a} u(x) 
    &:= 
    (x-a)\lrcorner \int_{0}^{1} t^{\ell-1} u\left(a+t(x-a)\right)\,dt\;
    .
\end{align*}
By definition,
\begin{align*}
    \Poinc_{\ell} u(x) 
    =
    \int_{\Omega} \Poinc_{\ell,a} u(x) \,da\;
    .
\end{align*}
We study the interaction of the potential operator with the exterior derivative in more detail.
Suppose that $u \in C^{\infty}(\bbR^n,\Alt^{\ell})$ with $\supp u \subseteq \overline\Omega$.
We rewrite the potential,
\begin{align*}
    \Poinc_{\ell,a} u(x) 
    &= 
    (x-a)\lrcorner \int_{0}^{1} t^{\ell-1} u\left(a+t(x-a)\right)\,dt\,da\;
    \\&
    = 
    (x-a)\lrcorner 
    \sum_{\sigma \in \Sigma(k,n)}
    \int_{0}^{1} 
    t^{\ell-1} u_{\sigma}\left(a+t(x-a)\right) \cartanx^{\sigma} dt 
    \\&
    = 
    \sum_{\sigma \in \Sigma(k,n)} \sum_{i=1}^{k}
    \int_{0}^{1} 
    t^{\ell-1} u_{\sigma}\left(a+t(x-a)\right) (-1)^{i-1} (x-a)_{\sigma(i) } \cartanx^{\sigma-\sigma(i)} dt 
    ,
\end{align*}
and compute its exterior derivative:
\begin{align*}
    \cartan \Poinc_{\ell,a} u(x) 
    &= 
    \ell
    \sum_{\sigma \in \Sigma(k,n)} \sum_{i=1}^{k}
    \int_{0}^{1} 
    t^{\ell-1} u_{\sigma}\left(a+t(x-a)\right) \cartanx^{\sigma-\sigma(i)} dt 
    \\&\qquad
    + 
    \sum_{\sigma \in \Sigma(k,n)} \sum_{i=1}^{k} \sum_{j=1}^{n}
    \int_{0}^{1} 
    t^{\ell} \frac{ \partial u_{\sigma} }{\partial x_{j}}\left(a+t(x-a)\right) (-1)^{i-1} (x-a)_{\sigma(i) } \cartanx^{j} \wedge \cartanx^{\sigma-\sigma(i)} dt 
    .
\end{align*}
We write the exterior derivative of $u$ as 
\begin{align*}
    \cartan u(x)
    =
    \sum_{\sigma \in \Sigma(k,n)} \sum_{j=1}^{n}
    \frac{ \partial u }{\partial x_{j}}(x) \cartanx^{j} \wedge \cartanx^{\sigma} dt 
    ,
\end{align*}
and apply the potential operator to this result, which gives 
\begin{align*}
    \Poinc_{\ell+1,a} \cartan u(x)
    &=
    (x-a)\lrcorner 
    \sum_{\sigma \in \Sigma(k,n)} \sum_{j=1}^{n}
    \int_{0}^{1} t^{\ell} \frac{ \partial u_{\sigma} }{\partial x_{j}}\left(a+t(x-a)\right) dt 
    \;\cartanx^{j} \wedge \cartanx^{\sigma}
    \\&
    = 
    \sum_{\sigma \in \Sigma(k,n)} \sum_{j=1}^{n}
    \int_{0}^{1} t^{\ell} \frac{ \partial u_{\sigma} }{\partial x_{j}}\left(a+t(x-a)\right) dt (x-a)_{j}
    \;\cartanx^{\sigma}
    \\&\qquad 
    - 
    \sum_{\sigma \in \Sigma(k,n)} \sum_{i=1}^{k} \sum_{j=1}^{n}
    (-1)^{i-1}
    \int_{0}^{1} t^{\ell} \frac{ \partial u_{\sigma} }{\partial x_{j}}\left(a+t(x-a)\right) dt 
    (x-a)_{\sigma(i)} \cartanx^{j} \wedge \cartanx^{\sigma-\sigma(i)}
    .
\end{align*}
We add the exterior derivative of the potential and the potential of the exterior derivative.
Taking into account cancellations, this gives the identity 
\begin{align*}
    &
    \cartan \Poinc_{\ell,a} u(x)
    +
    \Poinc_{\ell+1,a} \cartan u(x)
    \\&
    =
    \ell
    \sum_{\sigma \in \Sigma(k,n)} \sum_{i=1}^{k}
    \int_{0}^{1} 
    t^{\ell-1} u_{\sigma}\left(a+t(x-a)\right) \cartanx^{\sigma-\sigma(i)} dt 
    \\&\qquad
    + 
    \sum_{\sigma \in \Sigma(k,n)} \sum_{i=1}^{k} \sum_{j=1}^{n}
    \int_{0}^{1} 
    t^{\ell} \frac{ \partial u_{\sigma} }{\partial x_{j}}\left(a+t(x-a)\right) (-1)^{i-1} (x-a)_{\sigma(i) } \cartanx^{j} \cartanx^{\sigma-\sigma(i)} dt 
    \\&\qquad
    +
    \sum_{\sigma \in \Sigma(k,n)} \sum_{j=1}^{n}
    \int_{0}^{1} t^{\ell} \frac{ \partial u_{\sigma} }{\partial x_{j}}\left(a+t(x-a)\right) dt (x-a)_{j}
    \cartanx^{\sigma}
    \\&\qquad
    - 
    \sum_{\sigma \in \Sigma(k,n)} \sum_{i=1}^{k} \sum_{j=1}^{n}
    (-1)^{i-1}
    \int_{0}^{1} t^{\ell} \frac{ \partial u_{\sigma} }{\partial x_{j}}\left(a+t(x-a)\right) dt 
    (x-a)_{\sigma(i)} \cartanx^{j} \wedge \cartanx^{\sigma-\sigma(i)}
    \\&
    =
    \sum_{\sigma \in \Sigma(k,n)} \sum_{i=1}^{k}
    \int_{0}^{1} 
    \ell t^{\ell-1} u_{\sigma}\left(a+t(x-a)\right) \cartanx^{\sigma-\sigma(i)} dt 
    \\&\qquad
    +
    \sum_{\sigma \in \Sigma(k,n)} \sum_{j=1}^{n}
    \int_{0}^{1} t^{\ell} \frac{ \partial u_{\sigma} }{\partial x_{j}}\left(a+t(x-a)\right) (x-a)_{j} dt
    \cartanx^{\sigma}
    \\&
    =
    \sum_{\sigma \in \Sigma(k,n)} 
    \int_{0}^{1} \frac{\partial}{\partial t} \left( t^{\ell} u_{\sigma}\left(a+t(x-a)\right) \right) dt \cartanx^{\sigma}
    \\&
    =
    \sum_{\sigma \in \Sigma(k,n)} 
    u_{\sigma}(x) \cartanx^{\sigma}
    =
    u(x)
    .
\end{align*}
In particular, if $\cartan u = 0$, then $\cartan \Poinc_{\ell,a} u = u$.






% Develop construction of complexes via adding patches 
\subsection{Patch coverings}

We are interested in exhausting a simplicial complex in a controlled manner. 
Suppose that $\calT$ is a manifold-like $n$-dimensional simplicial complex. 
We call $n$-simplices $S,T \in \calT$ \emph{face-connected} if there exists a sequence $S_0=S,S_1,\dots,T=S_m$ such that $S_{i} \cap S_{i-1}$ is a face for all $1 \leq i \leq m$. Clearly, face-connected is an equivalence relation. A \emph{connected component} of $\calT$ is an equivalence class under the face-connected relation, and we call $\calT$ connected if all its $n$-simplices are face-connected. 
In a manifold-like simplicial complex, two $n$-simplices have non-empty intersection if and only if they are face-connected. 

It is a consequence of Lemma~\ref{lemma:characterizationofmanifoldcomplexes} that,
if $\calT$ is a manifold-like simplicial complex, then $\patch_{\calT}(S)$ is connected for all $S \in \calT$.
Moreover, a manifold-like simplicial complex $\calT$ is connected if and only if $\bigcup\calT$ is a connected topological space. 

We let \emph{$k$-patching} refer to enumerations $S_1, S_2, \dots$ of $\subsimplex_{k}(\calT)$
such that for any $0 \leq m$, the union 
$\calU_{m} := \patch_{\calT}(S_0) \cup \patch_{\calT}(S_1) \cup \patch_{\calT}(S_2) \cup \dots \cup \patch_{\calT}(S_m)$
shares an $n$-simplex with the patch $\patch_{\calT}(S_{m+1})$.
We call the $k$-patching \emph{manifold-like} if $\calU_{m}$ is manifold-like for all $0 \leq m$.
Clearly, a $k$-patching only exists if $k < n$ and if $\calT$ is connected. 
We refer the reader to Figure~\ref{figure:illustrationpatching} for an illustration.

\begin{lemma}
    A connected manifold-like $n$-dimensional simplicial complex $\calT$ admits a $k$-patching for any $k < n$.
\end{lemma}
\begin{proof}
    We define an undirected graph by letting $\subsimplex_{k}(\calT)$ be the set of nodes 
    and connecting any $S, S' \in \subsimplex_{k}(\calT)$ if there exists $T \in \subsimplex_{k+1}(\calT)$ with $S, S' \subseteq T$. 
    Note that in this case $\patch_{\calT}(S) \cap \patch_{\calT}(S') = \patch_{\calT}(T)$. 
    Since $\calT$ is connected, the graph $\calG$ is connected. 
    
    We let $S_0, S_1, S_2, \dots$ be an enumeration of the $k$-simplices generated by a breadth-first traversal of the graph $\calG$,
    starting at some arbitrary but fixed $S_0 \in \calT$. 
    Since $\calG$ is locally finite and connected, 
    this breadth-first traversal includes all $k$-simplices of $\calT$.
    
    Let $m \in \bbZ$ with $m \geq 1$. If $S_m$ has distance $d \in \bbN$ from $S_1$ in the graph $\calG$,
    then there exists $0 \leq l \leq m$ such that $S_l$ has distance $d-1$ from $S_1$ and is connected to $S_m$. 
    It follows that the sequence $S_0, S_1, S_2, \dots$ is a $k$-patching of $\calT$. 
\end{proof}

The preceding lemma is generally not true when the $k$-patching is required to be manifold-like,
as can be verified from some simple examples;~see Figure~\ref{figure:annuluscounterexample}.

\color{Emerald}
\begin{align}
    \min\limits_{ c \in \bbR }
    \| u - c \|_{L^{p}(\Omega)}
    \leq 
    % 2^{ - \frac{p-1}{p}}
    %2^{ - 1 + \frac{1}{p}}
    \frac{1}{ 2^{ 1 - \frac{1}{p} } }
    \diam(\Omega)
    \| \nabla u \|_{L^{p}(\Omega)}
    ,
    \quad 
    u \in W^{1,p}(\Omega)
    ,
\end{align}

\begin{proof}
    Given $z \in \Omega$, we define 
    \[
        w(x,z) 
        := 
        \int_0^1 \bff( z + t(x-z) ) \cdot (x-z)
        =
        u(x) - u(z)
        .
    \]
    Subsequently,
    \begin{align*}
        w(x) = |\Omega|^{-1} \int_{\Omega} w(x,z) dz.
    \end{align*}
    Clearly, $\nabla_{x} u(x) = \nabla_{x} w(x,z) = \nabla_{x} w(x)$.
    Next, 
    \begin{align*}
        \int_{\Omega} | w(x) |^{p} dx
        &\leq 
        \int_{\Omega} |\Omega|^{-1} \int_{\Omega} | w(x,z) |^{p} dz dx
        \\&\leq 
        \int_{\Omega} |\Omega|^{-1} \int_{\Omega} \int_0^1 | \bff( z + t(x-z) ) \cdot (x-z) |^{p} dz dx
        \\&= 
        |\Omega|^{-1} \int_{\Omega} \int_{\Omega} \int_0^{D/|x-z|} | \bff( z + t(x-z) ) \cdot (x-z) |^{p} dz dx
        .
    \end{align*}

\end{proof}
\color{black}

\begin{lemma}
    Suppose that $T, T'$ are two $n$-simplices whose intersection is a common face $F := T_1 \cap T_2$. Then 
    \begin{align*}
        \| u - u_{\Omega} \|_{L^p(\Omega)} 
        &
        \leq 
        \frac{2}{((n-1)!)^n }
        \left( \frac{\diam(T)^n}{\vol(T)} \right)^n
        \left( \frac{\diam(T')^n}{n! \vol(T')} \right)^n
        \diam(\Omega)
        \| \nabla u \|_{L^p(\Omega)} 
    \end{align*}
    for any $u \in W^{1,p}(\Omega)$ with $p \in [1,\infty)$.
\end{lemma}
\begin{proof}
    Suppose a tetrahedron $T$ has vertices $v_0,v_1,\dots,v_n$ and let $F_0,F_1\dots,F_n$ be the faces opposite to those vertices. 
    The height $h(T,k)$ of the vertex $v_k$ in $T$ equals 
    \[
        h(T,k) := n \frac{\vol(T)}{\vol(F_k)}
    \]
    A lower bound for the height in terms of the volume and diameter of $T$ is 
    \[
        h(T,k) 
        = 
        n \frac{\vol(T)}{\diam(T)^n}
        \cdot 
        \frac{\diam(T)^n}{\vol(F_k)}
        \geq 
        n \frac{\vol(T)}{\diam(T)^n}
        \cdot 
        (n-1)! \frac{\diam(T)^n}{\diam(T)^{n-1}}
        \geq 
        \frac{\vol(T)}{\diam(T)^n}
        \cdot 
        n! \diam(T) 
        .
    \]
    Any subsimplex $S$ of $T$ adjacent to $v_k$ thus satisfies 
    \[
        \diam(S)
        \geq 
        \frac{\vol(T)}{\diam(T)^n}
        \cdot 
        n! \diam(T) 
        .
    \]
    We let $w \in F_0$ be the barycenter in the face opposite to $v_0$. 
    Then the convex combination of $w$ and any of the faces $F_1,\dots,F_n$ defines simplices $S_1,\dots,S_n$ which satisfy 
    \[
        \vol(S_1) = \dots = \vol(S_n) = \frac{\vol(T)}{n}
        .
    \]
    Consequently, 
    the height $h_k := $ of the vertex $w$ in $S_k$ equals 
    \[
        h_k := n \frac{\vol(S_k)}{\vol(F_k)} = \frac{\vol(T)}{\vol(F_k)}
    \]
    It follows that $T$ is star-shaped with respect to the intersection of $T$
    with the ball around $w$ of radius $h$, where $h := \min_{1 \leq k \leq n} h_k$.
    A lower bound for height in terms of the volume and diameter of $T$ is 
    \[
        h_k 
        = 
        \frac{\vol(T)}{\diam(T)^n}
        \cdot 
        \frac{\diam(T)^n}{\vol(F_k)}
        \geq 
        \frac{\vol(T)}{\diam(T)^n}
        \cdot 
        (n-1)! \frac{\diam(T)^n}{\diam(T)^{n-1}}
        \geq 
        \frac{\vol(T)}{\diam(T)^n}
        \cdot 
        (n-1)! \diam(T) 
        .
    \]
    Let now $T'$ be another $n$-simplex with vertices $v_0',v_1,\dots,v_n$.
    We construct $h'$ analogous to above and $h'' := \min(h,h')$. 
    Hence $\Omega := T \cup T'$ is star-shaped with respect to a ball around $w$ of radius $h''$.
    Using a result by Farweg and Rosteck,
    \begin{align*}
        \| u - u_{\Omega} \|_{L^p(\Omega)} 
        &
        \leq 
        2 \diam(\Omega)^n
        \omega_n^{ 1 - \frac 1 n }
        \frac{|\Omega|^{\frac{1}{n}}}{|\Ball(w,h'')|}
        \| \nabla u \|_{L^p(\Omega)} 
        \\&
        \leq 
        2 \diam(\Omega)^n
        \omega_n^{ - \frac 1 n }
        \frac{|\Omega|^{\frac{1}{n}}}{(h'')^n}
        \| \nabla u \|_{L^p(\Omega)} 
        .
    \end{align*}
    Without loss of generality, $\diam(T) \leq \diam(T')$. Now, 
    \begin{align*}
        \| u - u_{\Omega} \|_{L^p(\Omega)} 
        &
        \leq 
        \left( \frac{\diam(T)^n}{\vol(T)} \right)^n
        2 \diam(\Omega)^n
        \omega_n^{ - \frac 1 n }
        \frac{|\Omega|^{\frac{1}{n}}}{((n-1)!)^n \diam(T)^n }
        \| \nabla u \|_{L^p(\Omega)} 
        \\&
        \leq 
        \frac{2}{((n-1)!)^n }
        \left( \frac{\diam(T)^n}{\vol(T)} \right)^n
        \frac{ \diam(T')^n }{ \diam(T)^n }
        \diam(T')
        \| \nabla u \|_{L^p(\Omega)} 
        .
    \end{align*}
    We know that 
    \begin{align*}
        \diam(T')
        \leq 
        \frac{\diam(T')^n}{n! \vol(T')}
        \diam(F)
        \leq 
        \frac{\diam(T')^n}{n! \vol(T')}
        \diam(T)
        .
    \end{align*}
    The final estimate is
    \begin{align*}
        \| u - u_{\Omega} \|_{L^p(\Omega)} 
        &
        \leq 
        \frac{2}{((n-1)!)^n }
        \left( \frac{\diam(T)^n}{\vol(T)} \right)^n
        \left( \frac{\diam(T')^n}{n! \vol(T')} \right)^n
        \diam(\Omega)
        \| \nabla u \|_{L^p(\Omega)} 
        .
    \end{align*}
    This finishes the proof. 
\end{proof}


We use the following trace inequality, which can be found in the literature. 

\begin{lemma}\label{lemma:traceinequality}
    Let $p \in [1,\infty)$ and let $T$ be an $n$-dimensional simplex with a face $F$. Then 
    \begin{align*}
        \| u \|_{L^{p}(F)}^{p}
        \leq 
        \frac{|F|}{|T|} 
        \| u \|_{L^{p}(T)}^{p}
        +
        p \cdot \diam(T) 
        \frac{|F|}{|T|} 
        \| u \|_{L^{p}(T)}^{p-1}
        \| \nabla u \|_{\bfL^{p}(T)}
        .
    \end{align*}
    for all $u \in W^{1,p}(T)$. Moreover, 
    \begin{align*}
        \| u \|_{L^{\infty}(F)}
        \leq 
        \| u \|_{L^{\infty}(T)}
        .
    \end{align*}
    for all $u \in W^{1,\infty}(T)$. \qed
\end{lemma}

\begin{remark}
    Lemma~\cite{veeser2012poincare} has been stated by Veeser and Verfürth~\cite[Lemma 2.8]{veeser2012poincare} when $1 \leq p < \infty$. 
    The case $p = \infty$ is obvious because functions in $W^{1,\infty}(T)$ are already Lipschitz. 
    \color{red}\textbf{The following is probably wrong} 
    However, note that for any $p \in [1,\infty)$ and $u \in W^{1,p}(T)$ we have 
    \begin{align*}
        \| u \|_{L^{p}(F)}
        &\leq 
        \| u \|_{L^{p}(T)}^{\frac{p-1}{p}}
        \left(
            \frac{|F|}{|T|}
            \| u \|_{L^{p}(T)}
            +
            p \cdot \diam(T)
            \frac{|F|}{|T|}
            \| \nabla u \|_{\bfL^{p}(T)}
        \right)^{\frac{1}{p}}
        \\&\leq 
        \| u \|_{L^{p}(T)}^{\frac{p-1}{p}}
        \left( \frac{|F|}{|T|} \right)^{\frac 1 p}
        \left(
            \| u \|_{L^{p}(T)}^{\frac{1}{p}}
            +
            \sqrt[p]{p} \cdot \diam(T)^{\frac 1 p} 
            \| \nabla u \|_{\bfL^{p}(T)}^{\frac 1 p} 
        \right)
        .
    \end{align*}
    The limit as $p$ goes to infinity leads to the inequality as stated in the preceding lemma,
    which is therefore the natural limit case of the aforementioned trace inequality.
\end{remark}


\begin{lemma}
    Let $T_1$ and $T_{2}$ be two $n$-simplices such that $F = T_1 \cap T_2$ is a common subsimplex of dimension $n-1$. Then 
    \color{red} Here we estimate the Poincar\'e constant over a face patch.
\end{lemma}

\color{red}
Possible approaches: 
\begin{itemize}
 \item Take the trace jump (requires trace inequality with explicit constant) and extend the resulting constant function
 \item The union is star-shaped with respect to some (hopefully large) convex set, then use inequality for that situation (available in literature)
 \item Reflection trick
 \item Paper by Veeser and Verfuerth?
\end{itemize}
Most importantly, all constants must be explicitly computable!
We can do all these methods and compare the results. 
\color{black}
    
\begin{proof}
    We write $U := T_1 \cup T_2$. Let $u \in W^{1,p}(U)$.
    Suppose $u_1 \in W^{1,p}(T_1)$ and $u_2 \in W^{1,p}(T_2)$
    satisfy $\nabla u_1 = \nabla u_{|T_1}$ and $\nabla u_2 = \nabla u_{|T_2}$.
    Then there exists a constant $c \in \bbR$ such that $c = \trace_{T_1,F} u_1 - \trace_{T_1,F} u_2$.
    Consequently, if we define $u_2' = u_2 + c$, then we obtain a function $u = u_1 \cup u_2' \in W^{1,p}(T)$. 
    Notice that 
    \begin{align*}
        \| u \|_{L^{p}(U)}^{p}
        &\leq 
        \| u_1 \|_{L^{p}(T_1)}^{p}
        +
        \| u'_2 \|_{L^{p}(T_2)}^{p}
        \\
        &\leq 
        \| u_1 \|_{L^{p}(T_1)}^{p}
        +
        \left( 
            \| u_2 \|_{L^{p}(T_2)}
            +
            \left(\frac{|T_2|}{|F|}\right)^{\frac 1 p}
            \| c \|_{L^{p}(F)}
        \right)^{p}
    \end{align*}
    \begin{align*}
        \color{red}
        \| u'_2 \|_{L^{p}(T_2)}
        \leq 
        \| u_2 \|_{L^{p}(T_2)}
        +
        \| c \|_{L^{p}(T_2)}
        = 
        \| u_2 \|_{L^{p}(T_2)}
        +
        \left(\frac{|T_2|}{|F|}\right)^{\frac 1 p}
        \| c \|_{L^{p}(F)}
        .
    \end{align*}
    Recall that 
    \begin{align*}
        \| v \|_{L^{p}(F)}^{p}
        \leq 
        \frac{|F|}{|T|} 
        \| v \|_{L^{p}(T)}^{p}
        +
        p \cdot \diam(T) 
        \frac{|F|}{|T|} 
        \| v \|_{L^{p}(T)}^{p-1}
        \| \nabla v \|_{\bfL^{p}(T)}
        ,
        \quad 
        v \in W^{1,p}(T)
        .
    \end{align*}
    Now 
    \begin{align*}
        \| c \|_{L^{p}(F)}^{p}
        &\leq 
        2^{p-1}
        \| u_1 \|_{L^{p}(F)}^{p}
        +
        2^{p-1}
        \| u_2 \|_{L^{p}(F)}^{p}
    \end{align*}
    We thus proceed with 
    \begin{align*}
        \| u_1 \|_{L^{p}(F)}^{p}
        +
        \| u_2 \|_{L^{p}(F)}^{p}
        &
        \leq 
        \frac{|F|}{|T_1|} 
        \| u_1 \|_{L^{p}(T_1)}^{p}
        +
        p \cdot \diam(T_1) 
        \frac{|F|}{|T_1|} 
        \| u_1 \|_{L^{p}(T_1)}^{p-1}
        \| \nabla u_1 \|_{\bfL^{p}(T_1)}
        \\&\quad\quad
        +
        \frac{|F|}{|T_2|} 
        \| u_2 \|_{L^{p}(T_2)}^{p}
        +
        p \cdot \diam(T_2) 
        \frac{|F|}{|T_2|} 
        \| u_2 \|_{L^{p}(T_2)}^{p-1}
        \| \nabla u_2 \|_{\bfL^{p}(T_2)}
    \end{align*}
    If $p = 1$, then we summarize this as 
    \begin{align*}
        \| u \|_{L^{1}(U)} 
        &\leq 
        \| u_1 \|_{L^{1}(T_1)} 
        +
        \| u'_2 \|_{L^{1}(T_2)} 
        \\
        &\leq 
        \| u_1 \|_{L^{1}(T_1)} 
        +
        \| u_2 \|_{L^{1}(T_2)}
        +
        \left(\frac{|T_2|}{|F|}\right)
        \| c \|_{L^{1}(F)}
        \\
        &\leq 
        \| u_1 \|_{L^{1}(T_1)} 
        +
        \| u_2 \|_{L^{1}(T_2)}
        +
        \left(\frac{|T_2|}{|F|}\right)
        \| u_1 \|_{L^{1}(F)} 
        +
        \left(\frac{|T_2|}{|F|}\right)
        \| u_2 \|_{L^{1}(F)} 
        \\
        &\leq 
        \| u_1 \|_{L^{1}(T_1)} 
        +
        \| u_2 \|_{L^{1}(T_2)}
        \\&\qquad 
        +
        \left(\frac{|T_2|}{|F|}\right)
        \frac{|F|}{|T_1|} 
        \| u_1 \|_{L^{1}(T_1)} 
        +
        \diam(T_1) 
        \left(\frac{|T_2|}{|F|}\right)
        \frac{|F|}{|T_1|} 
        \| \nabla u_1 \|_{\bfL^{1}(T_1)}
        \\&\quad\quad
        +
        \left(\frac{|T_2|}{|F|}\right)
        \frac{|F|}{|T_2|} 
        \| u_2 \|_{L^{1}(T_2)} 
        +
        \diam(T_2) 
        \left(\frac{|T_2|}{|F|}\right)
        \frac{|F|}{|T_2|} 
        \| \nabla u_2 \|_{\bfL^{1}(T_2)}
        \\
        &\leq 
        \| u_1 \|_{L^{1}(T_1)} 
        +
        \| u_2 \|_{L^{1}(T_2)}
        \\&\qquad 
        +
        \frac{|T_2|}{|T_1|} 
        \| u_1 \|_{L^{1}(T_1)} 
        +
        \diam(T_1) 
        \frac{|T_2|}{|T_1|} 
        \| \nabla u_1 \|_{\bfL^{1}(T_1)}
        \\&\quad\quad
        +
        \| u_2 \|_{L^{1}(T_2)} 
        +
        \diam(T_2) 
        \| \nabla u_2 \|_{\bfL^{1}(T_2)}
        \\
        &\leq 
        \left( 1 + \frac{|T_2|}{|T_1|} \right)
        \| u_1 \|_{L^{1}(T_1)} 
        +
        2
        \| u_2 \|_{L^{1}(T_2)}
        +
        \diam(T_1) 
        \frac{|T_2|}{|T_1|} 
        \| \nabla u_1 \|_{\bfL^{1}(T_1)}
        +
        \diam(T_2) 
        \| \nabla u_2 \|_{\bfL^{1}(T_2)}
        .
    \end{align*}
    When $1 < p < \infty$, then Young's inequality implies 
    \begin{align*}
        \| u_1 \|_{L^{p}(F)}^{p}
        +
        \| u_2 \|_{L^{p}(F)}^{p}
        &
        \leq 
        \frac{|F|}{|T_1|} 
        \| u_1 \|_{L^{p}(T_1)}^{p}
        +
        \frac{|F|}{|T_1|} 
        \| u_1 \|_{L^{p}(T_1)}^{q(p-1)}
        +
        \frac p q \cdot \diam(T_1)^{p}
        \frac{|F|}{|T_1|} 
        \| \nabla u_1 \|_{\bfL^{p}(T_1)}^{p}
        \\&\quad\quad
        +
        \frac{|F|}{|T_2|} 
        \| u_2 \|_{L^{p}(T_2)}^{p}
        +
        \frac{|F|}{|T_2|} 
        \| u_2 \|_{L^{p}(T_2)}^{q(p-1)}
        +
        \frac p q \cdot \diam(T_2)^{p} 
        \frac{|F|}{|T_2|} 
        \| \nabla u_2 \|_{\bfL^{p}(T_2)}^{p}
        \\
        &
        \leq 
        \frac{|F|}{|T_1|} 
        \| u_1 \|_{L^{p}(T_1)}^{p}
        +
        \frac{|F|}{|T_1|} 
        \| u_1 \|_{L^{p}(T_1)}^{p}
        +
        (p-1) \cdot \diam(T_1)^{p} 
        \frac{|F|}{|T_1|} 
        \| \nabla u_1 \|_{\bfL^{p}(T_1)}^{p}
        \\&\quad\quad
        +
        \frac{|F|}{|T_2|} 
        \| u_2 \|_{L^{p}(T_2)}^{p}
        +
        \frac{|F|}{|T_2|} 
        \| u_2 \|_{L^{p}(T_2)}^{p}
        +
        (p-1) \cdot \diam(T_2)^{p} 
        \frac{|F|}{|T_2|} 
        \| \nabla u_2 \|_{\bfL^{p}(T_2)}^{p}
        .
    \end{align*}
    \color{red}The last terms vanish in the limit $p \rightarrow 1$. 
    It is not clear whether $p=1$ emerges as a limit case: 
    we have gained an additional $p$-norm of $u$ but applying the Poincare inequality introduces the Poincare constant, 
    whereas there is no such dependence on the Poincare constant in the case $p=1$ above. 
    \color{black}
    In summary, 
    \begin{align*}
        2^{1-p}
        \left(\frac{|T_2|}{|F|}\right)
        \| c \|_{L^{p}(F)}^{p}
        &
        \leq 
        2
        \frac{|T_2|}{|T_1|} 
        \| u_1 \|_{L^{p}(T_1)}^{p}
        +
        (p-1) \cdot \diam(T_1)^{p} 
        \frac{|T_2|}{|T_1|} 
        \| \nabla u_1 \|_{\bfL^{p}(T_1)}^{p}
        \\&\quad\quad\quad\quad
        +
        2
        \| u_2 \|_{L^{p}(T_2)}^{p}
        +
        (p-1) \cdot \diam(T_2)^{p} 
        \| \nabla u_2 \|_{\bfL^{p}(T_2)}^{p}
        .
    \end{align*}
    We thus find 
    %{\tiny
    \begin{align*}
        \| u \|_{L^{p}(U)}^{p}
        &\leq 
        \| u_1 \|_{L^{p}(T_1)}^{p}
        +
        \| u'_2 \|_{L^{p}(T_2)}^{p}
        \\
        &\leq 
        \| u_1 \|_{L^{p}(T_1)}^{p}
        +
        \left( 
            \| u_2 \|_{L^{p}(T_2)}
            +
            \left(\frac{|T_2|}{|F|}\right)^{\frac 1 p}
            \| c \|_{L^{p}(F)}
        \right)^{p}
        \\
        &\leq 
        \| u_1 \|_{L^{p}(T_1)}^{p}
        +
        2^{p-1}
        \| u_2 \|_{L^{p}(T_2)}^{p}
        +
        2^{p-1}
        \left(\frac{|T_2|}{|F|}\right)
        \| c \|_{L^{p}(F)}^{p}
        \\
        % &\leq 
        % \| u_1 \|_{L^{p}(T_1)}^{p}
        % +
        % 2^{p-1}
        % \| u_2 \|_{L^{p}(T_2)}^{p}
        % +
        % 4^{p-1}
        % \left( 
        %     2 \max\left( 
        %         \frac{|T_2|}{|T_1|} 
        %         ,
        %         1
        %     \right)
        %     \| u_1 \cup u_2 \|_{L^{p}(U)}^{p}
        %     +
        %     (p-1) \cdot 
        %     \max\left( 
        %         \diam(T_1)^p \frac{|T_2|}{|T_1|} 
        %         ,
        %         \diam(T_2)^p 
        %         %
        %     \right) 
        %     \| \nabla u_1 \cup \nabla u_2 \|_{\bfL^{p}(U)}^{p}
        % \right)
        % \\
        &\leq 
        \| u_1 \|_{L^{p}(T_1)}^{p}
        +
        2^{p-1}
        \| u_2 \|_{L^{p}(T_2)}^{p}
        \\&\quad\quad\quad\quad
        +
        4^{p-1}
        2
        \frac{|T_2|}{|T_1|} 
        \| u_1 \|_{L^{p}(T_1)}^{p}
        +
        4^{p-1}
        (p-1) \cdot \diam(T_1)^{p} 
        \frac{|T_2|}{|T_1|} 
        \| \nabla u_1 \|_{\bfL^{p}(T_1)}^{p}
        \\&\quad\quad\quad\quad
        +
        4^{p-1}
        2
        \| u_2 \|_{L^{p}(T_2)}^{p}
        +
        4^{p-1}
        (p-1) \cdot \diam(T_2)^{p} 
        \| \nabla u_2 \|_{\bfL^{p}(T_2)}^{p}
    \end{align*}
    %}
    We can now apply the Poincar\'e--Friedrichs inequality over each simplex. Hence,
    \begin{align*}
        \| u \|_{L^{p}(U)}^{p}
        &\leq 
        C_1 D^{p} \| \nabla u_1 \|_{L^{p}(T_1)}^{p}
        +
        2^{p-1}
        C_2 D^{p} \| \nabla u_2 \|_{L^{p}(T_2)}^{p}
        \\&\quad\quad\quad\quad
        +
        4^{p-1}
        2
        Q
        C_1 D^{p} \| \nabla u_1 \|_{L^{p}(T_1)}^{p}
        +
        4^{p-1}
        (p-1) \cdot D^{p} 
        Q
        \| \nabla u_1 \|_{\bfL^{p}(T_1)}^{p}
        \\&\quad\quad\quad\quad
        +
        4^{p-1}
        2
        C_2 D^{p} \| \nabla u_2 \|_{L^{p}(T_2)}^{p}
        +
        4^{p-1}
        (p-1) \cdot D^{p} 
        \| \nabla u_2 \|_{\bfL^{p}(T_2)}^{p}
        \\
        &\leq 
        \left( 
            C_1 
            +
            4^{p-1}
            2
            Q
            C_1 
            +
            4^{p-1}
            (p-1)
            Q
        \right) 
        D^{p}
        \| \nabla u_1 \|_{L^{p}(T_1)}^{p}
        \\&\quad\quad\quad\quad
        +
        \left( 
        2^{p-1}
        C_2 
        +
        4^{p-1}
        2
        C_2 
        +
        4^{p-1}
        (p-1)
        \right) 
        D^{p}
        \| \nabla u_2 \|_{L^{p}(T_2)}^{p}
        \\
        &\leq 
        \left( 
            2^{p-1}
            C  
            +
            4^{p-1}
            2
            Q'
            C_1 
            +
            4^{p-1}
            (p-1)
            Q'
        \right) 
        D^{p} 
        \| \nabla u \|_{L^{p}(U)}^{p}
        .
    \end{align*}
    
    \color{red} Possibly the estimate can be tweaked to obtain nicer constants.
\end{proof}



%  
 % \color{red}We want to summarize the Lp norms so that the Poincare inequality over $U_F$ is applied only once. In the special cases $p=1$ and $p=\infty$, this works as above. Do we have a formula for general $p$ that contains these as limit cases? \color{black}
 % In the special case $p < \infty$, we use the finite-dimensional H\"older inequality to find 
 % \begin{align*}
 %    \| w_{m} \|_{L^{p}(T_m)}^{p}
 %    &\leq 
 %    3^{p-1}
 %    \| u_{F_m} \|_{L^{p}(T_m)}^{p}
 %    +
 %    3^{p-1}
 %    \frac{ \vol(T_m) }{ \vol(T_{j(m)}) }
 %    \| u_{F_{m}} \|_{L^{p}(T_{j(m)})}^{p}
 %    +
 %    3^{p-1}
 %    \frac{ \vol(T_m) }{ \vol(T_{j(m)}) }
 %    \| w_{m-1}\|_{L^{p}(T_{j(m)})}^{p}
 %    \\&\leq 
 %    3^{p-1}
 %    \max\left( 1, \frac{ \vol(T_m) }{ \vol(T_{j(m)}) } \right)
 %    \| u_{F_m} \|_{L^{p}(U_{F_m})}^{p}
 %    +
 %    3^{p-1}
 %    \frac{ \vol(T_m) }{ \vol(T_{j(m)}) }
 %    \| w_{m-1}\|_{L^{p}(T_{j(m)})}^{p}
 %    ,
 % \end{align*}
 % and therefore
 % \begin{align*}
 %    \| w_{m} \|_{L^{p}(T_m)}
 %    &\leq 
 %    3^{\frac{p-1}{p}}
 %    \max\left( 1, \frac{ \vol(T_m) }{ \vol(T_{j(m)}) } \right)^{\frac 1 p}
 %    \| u_{F_m} \|_{L^{p}(U_{F_m})} 
 %    +
 %    3^{\frac{p-1}{p}}
 %    \frac{ \vol(T_m)^{\frac 1 p} }{ \vol(T_{j(m)})^{\frac 1 p} }
 %    \| w_{m-1}\|_{L^{p}(T_{j(m)})} 
 %    .
 % \end{align*}
 % \color{blue}
 % In the special case $p < \infty$, we use the finite-dimensional H\"older inequality to find 
 % \begin{align*}
 %    \| w_{m} \|_{L^{p}(T_m)}^{p}
 %    &\leq 
 %    \left( 
 %        1 + 2 \frac{ \vol(T_m)^{\frac{1}{p-1}} }{ \vol(T_{j(m)})^{\frac{1}{p-1}} }
 %    \right)^{ p-1 }
 %    \left( 
 %        \| u_{F_m} \|_{L^{p}(T_m)}^{p}
 %        +
 %        \| u_{F_{m}} \|_{L^{p}(T_{j(m)})}^{p}
 %        +
 %        \| w_{m-1}\|_{L^{p}(T_{j(m)})}^{p}
 %    \right)
 %    ,
 % \end{align*}
 % and therefore
 % \begin{align*}
 %    \| w_{m} \|_{L^{p}(T_m)}
 %    &\leq 
 %    \left( 
 %        1 + 2 \frac{ \vol(T_m)^{\frac{1}{p-1}} }{ \vol(T_{j(m)})^{\frac{1}{p-1}} }
 %    \right)^{ \frac{p-1}{p} }
 %    \left( 
 %        \| u_{F_m} \|_{L^{p}(T_m)} 
 %        +
 %        \| u_{F_{m}} \|_{L^{p}(T_{j(m)})} 
 %        +
 %        \| w_{m-1}\|_{L^{p}(T_{j(m)})} 
 %    \right)
 %    .
 % \end{align*}
 % \color{black}
 
 
 



we have any of the following two conditions:
\begin{itemize}
 \item 
 there exists $\epsilon > 0$ 
 and a homeomorphism 
 \begin{align*}
    \phi : \Ball_\epsilon(x) \cap \underlying{\calT}  
    \rightarrow 
    \left\{ x \in \bbR^{n} \suchthat |x| \leq 1 \right\}
 \end{align*}
 such that $\phi(x) = 0$,
 \item 
 there exists $\epsilon > 0$ and a homeomorphism 
 \begin{align*}
    \phi : \Ball_\epsilon(x) \cap \underlying{\calT} 
    \rightarrow 
    \left\{ x \in \bbR^{n} \suchthat |x| \leq 1, x_n \leq 0 \right\}
 \end{align*}
 such that $\phi(x) = 0$.
\end{itemize}
This characterizes the union of the simplicial complex $\calT$ as a topological manifold with boundary. 















\subsection{On the topology of shellable triangulations}
    
A shellable simplicial complex that triangulates a manifold has a very restricted topology. 
The Mayer-Vietoris theorem implies restrictions on the homology groups of simplicial complexes iteratively constructed via shelling. 

    Suppose furthermore that $T$ is an $n$-simplex, $T \notin \calT$, 
    whose boundary complex intersects with $\partial\calT$ at a non-empty simplicial complex $\calS$ of dimension $n-1$. 
    In particular, $\calS$ is either the whole boundary complex of $T$ or a patch around a proper subsimplex of $T$. 
    We write $\calT'$ for the  simplicial complex that contains $\calT \cup T$.


Suppose that $\calT$ is a strongly connected $n$-dimensional simplicial complex that triangulates a manifold, and suppose that its boundary complex $\partial\calT$ is either empty or strongly connected.  
Suppose furthermore that $T$ is an $n$-simplex, $T \notin \calT$, whose boundary complex intersects with $\partial\calT$ at a non-empty simplicial complex $\calS$ of dimension $n-1$. In particular, $\calS$ is either the whole boundary complex of $T$ or a patch around a proper subsimplex of $T$. We write $\calT'$ be the  simplicial complex that contains $\calT \cup T$.
We exclude the case $n \leq 1$ since otherwise the discussion is trivial.

According to the simplicial Mayer-Vietoris theorem, there exists an exact sequence 
\begin{align*}
    \begin{CD}
        \dots \longrightarrow H_{k}(\calS) \longrightarrow H_{k}(\calT \oplus T) \longrightarrow H_{k}(\calT') \longrightarrow H_{k-1}(\calS) \longrightarrow\dots 
    \end{CD}
\end{align*}
We notice that $\calS$, $\calT$, $T$ and $\calT'$ are each connected, and hence their zeroth homology groups are isomorphic to ${G}$. 
By the nature of the mappings in the Mayer-Vietoris sequence, 
\color{red}which uses some specific knowledge about the arrows\color{black}, 
we see that 
\begin{align*}
    \begin{CD}
        0 \longrightarrow H_{0}(\calS) \longrightarrow H_{0}(\calT) \oplus H_{0}(T) \longrightarrow H_{0}(\calT') \longrightarrow 0 
    \end{CD}
\end{align*}
is an exact sequence. 
Moreover, $H_{k}(\calS) = H_{k}(T) = 0$ for $0 < k < n-1$. Hence, for $1 < k < n-1$,
the Mayer-Vietoris sequence includes the exact sequence 
\begin{align*}
    \begin{CD}
        0 \longrightarrow H_{k}(\calT \oplus T) \longrightarrow H_{k}(\calT') \longrightarrow 0 
    \end{CD}
\end{align*}
and we conclude that $H_{k}(\calT) = 0$ for $0 < k < n-1$.
Finally, the Mayer-Vietoris sequence includes the exact sequence 
\begin{align*}
    \begin{CD}
        0 \longrightarrow H_{n}(\calT \oplus T) \longrightarrow H_{n}(\calT') \longrightarrow H_{n-1}(\calS) \longrightarrow H_{n-1}(\calT \oplus T) \longrightarrow H_{n-1}(\calT') \longrightarrow 0 
    \end{CD}
\end{align*}

% Here, if $n=1$, it already follows that $H_{n}(\calT \oplus T) \simeq H_{n}(\calT')$.
% % We know that  $H_{0}(\calT') \simeq H_{0}(\calT) \simeq H_{0}(T) \simeq H_{0}(\calS) \simeq \bbZ$ because the respective simplicial complexes are path-connected. 
% Hence $H_{1}(\calT) \oplus H_{1}(T) \simeq H_{1}(\calT')$. 
% one easily sees that $H_{0}(\calT') \simeq \bbZ$ and $H_{1}(\calT) \oplus H_{1}(T) \simeq H_{1}(\calT')$. 
Since $\calT$ and $T$ are triangulations of manifolds with non-empty boundary, their $n$-th simplicial homology groups vanish. 
then $H_{n}(\calT') \simeq H_{n}(\calT \oplus T)$ are trivial and $H_{n-1}(\calT') \simeq H_{n-1}(\calT \oplus T)$ follows.
Consider now the case that $\calS$ is isomorphic to a sphere.
Since the boundary complex of $\calT$ is strongly connected and non-branching, it must coincide with $\calS$. Hence $\calT'$ triangulates a manifold without boundary, and thus $H_{n}(\calT') \simeq \bbZ$. 
\color{red}By some hand-waving argument, $H_{n}(\calT') \longrightarrow H_{n-1}(\calS)$ is an isomorphism.\color{black}
Hence $H_{n-1}(\calT) \rightarrow H_{n-1}(\calT')$ is an isomorphism.

In the case that $\calS$ is isomorphic to a disk, we have $H_{n-1}(\calS) = 0$,
and then $H_{n}(\calT') \simeq H_{n}(\calT \oplus T) = 0$ are trivial 
and $H_{n-1}(\calT') \simeq H_{n-1}(\calT \oplus T)$ follows.
Consider now the case that $\calS$ is isomorphic to a sphere.
Since the boundary complex of $\calT$ is strongly connected and non-branching, it must coincide with $\calS$. Hence $\calT'$ triangulates a manifold without boundary, and thus $H_{n}(\calT') \simeq {G}$. 
\color{red}By some hand-waving argument, $H_{n}(\calT') \longrightarrow H_{n-1}(\calS)$ is an isomorphism.\color{black}
Hence $H_{n-1}(\calT) \rightarrow H_{n-1}(\calT')$ is an isomorphism.


We conclude that the simplicial homology groups of a shellable $n$-dimensional simplicial complex $\calT$ that triangulates a manifold satisfy 
\begin{align*}
    H_0(\calT) \simeq \bbZ, 
    \quad 
    H_1(\calT) \simeq \dots \simeq H_{n-1}(\calT) \simeq 0.
\end{align*}
Moreover, $H_n(\calT) \simeq {G}$ if $\calT$ has no boundary and $H_n(\calT) \simeq 0$ otherwise. 
Moreover, $H_n(\calT) \simeq \bbZ$ if $\calT$ has no boundary and $H_n(\calT) \simeq 0$ otherwise. 

% then $H_{n}(\calT') \simeq H_{n}(\calT \oplus T)$ and $H_{n-1}(\calT') \simeq H_{n-1}(\calT \oplus T)$ follow immediately.
% Moreover, Thus $H_{n}(\calT') = H_{n}(\calT) \oplus H_{n}(\calS)$.
% We conclude that any shellable simplicial complex has the homology groups of a disk or a sphere. 
% NB: the zeroth homology is always free, and the n-th homology group, being the kernel of a mapping free abelian group, is free as well.
% \begin{align*}
%     \begin{CD}
%         0 \longrightarrow H_{1}(\calT) \oplus H_{1}(T) \longrightarrow H_{1}(\calT') \longrightarrow H_{0}(\calS) \longrightarrow H_{0}(\calT) \oplus H_{0}(T) \longrightarrow H_{0}(\calT') \longrightarrow 0 
%     \end{CD}
% \end{align*}

\begin{remark}    
    % % We know that  $H_{0}(\calT') \simeq H_{0}(\calT) \simeq H_{0}(T) \simeq H_{0}(\calS) \simeq {G}$ because the respective simplicial complexes are path-connected. 
    % Hence $H_{1}(\calT) \oplus H_{1}(T) \simeq H_{1}(\calT')$. 
    % one easily sees that $H_{0}(\calT') \simeq {G}$ and $H_{1}(\calT) \oplus H_{1}(T) \simeq H_{1}(\calT')$. 
     
    % then $H_{n}(\calT') \simeq H_{n}(\calT \oplus T)$ and $H_{n-1}(\calT') \simeq H_{n-1}(\calT \oplus T)$ follow immediately.
    % Moreover, Thus $H_{n}(\calT') = H_{n}(\calT) \oplus H_{n}(\calS)$.
    % We conclude that any shellable simplicial complex has the homology groups of a disk or a sphere. 
    % NB: the zeroth homology is always free, and the n-th homology group, being the kernel of a mapping free abelian group, is free as well.
    % \begin{align*}
    %     \begin{CD}
    %         0 \longrightarrow H_{1}(\calT) \oplus H_{1}(T) \longrightarrow H_{1}(\calT') \longrightarrow H_{0}(\calS) \longrightarrow H_{0}(\calT) \oplus H_{0}(T) \longrightarrow H_{0}(\calT') \longrightarrow 0 
    %     \end{CD}
    % \end{align*}
\end{remark}













\newpage\color{red}
 We want 
 \begin{align*}
    \| w_{m} \|_{L^{p}(T_m)}^{p}
    \leq
    \left( 
        1 
        + 
        2
        \frac{ \vol(T_m)^{\frac q p} }{ \vol(T_{j(m)})^{\frac q p} }
    \right)^{p/q}
    \left( 
        \| u_{F_m} \|_{L^{p}(T_m)}^{p}
        +
        \| u_{F_{m}} \|_{L^{p}(T_{j(m)})}^{p}
        +
        \| w_{m-1}\|_{L^{p}(T_{j(m)})}^{p}
    \right)
    .
 \end{align*}
 In the special case $p=1$ and $p=\infty$,
 \begin{align*}
    \| w_{m} \|_{L^{1}(T_m)}
    &\leq 
    \max\left(
        1, \frac{ \vol(T_m) }{ \vol(T_{j(m)}) } 
    \right)
    \| u_{F_m} \|_{L^{1}(U_{F_m})}
    +
    \max\left(
        1, \frac{ \vol(T_m) }{ \vol(T_{j(m)}) } 
    \right)
    \| w_{m-1}\|_{L^{1}(T_{j(m)})}
    .
    \\
    \| w_{m} \|_{L^{2}(T_m)}^{2}
    &\leq
    \left( 
        1 
        + 
        2
        \frac{ \vol(T_m) }{ \vol(T_{j(m)}) }
    \right)
    \left( 
        \| u_{F_m} \|_{L^{2}(T_m)}^{2}
        +
        \| u_{F_{m}} \|_{L^{2}(T_{j(m)})}^{2}
        +
        \| w_{m-1}\|_{L^{2}(T_{j(m)})}^{2}
    \right)
    .
    \\
    \| w_{m} \|_{L^{\infty}(T_m)}
    &\leq 
    3
    \max\left( 
        \| u_{F_m} \|_{L^{\infty}(U_{F_m})}
        ,
        \| w_{m-1}\|_{L^{\infty}(T_{j(m)})}
    \right)
    .
 \end{align*}
 \color{orange}
 \begin{align*}
    \| w_{m} \|_{L^{p}(T_m)}^{p}
    \leq
    \left( 
        2 
        + 
        \frac{ \vol(T_m)^{\frac q p} }{ \vol(T_{j(m)})^{\frac q p} }
    \right)^{p/q}
    \left( 
        \| u_{F_m} \|_{L^{p}(T_m)}^{p}
        +
        \| u_{F_{m}} \|_{L^{p}(T_{j(m)})}^{p}
        +
        \frac{ \vol(T_m) }{ \vol(T_{j(m)}) }
        \| w_{m-1}\|_{L^{p}(T_{j(m)})}^{p}
    \right)
    .
 \end{align*}
 In the special case $p=1$ and $p=\infty$, 
 \begin{align*}
    \| w_{m} \|_{L^{1}(T_m)}
    &\leq 
    \max\left(
        1, \frac{ \vol(T_m) }{ \vol(T_{j(m)}) } 
    \right)
    \| u_{F_m} \|_{L^{1}(U_{F_m})}
    +
    \max\left(
        1, \frac{ \vol(T_m) }{ \vol(T_{j(m)}) } 
    \right)
    \frac{ \vol(T_m) }{ \vol(T_{j(m)}) }
    \| w_{m-1}\|_{L^{1}(T_{j(m)})}
    .
    \\
    \| w_{m} \|_{L^{2}(T_m)}^{2}
    &\leq
    \left( 
        2 
        + 
        \frac{ \vol(T_m) }{ \vol(T_{j(m)}) }
    \right) 
    \left( 
        \| u_{F_m} \|_{L^{2}(T_m)}^{2}
        +
        \| u_{F_{m}} \|_{L^{2}(T_{j(m)})}^{2}
        +
        \frac{ \vol(T_m) }{ \vol(T_{j(m)}) }
        \| w_{m-1}\|_{L^{2}(T_{j(m)})}^{2}
    \right)
    \\
    \| w_{m} \|_{L^{\infty}(T_m)}
    &\leq 
    3
    \max\left( 
        \| u_{F_m} \|_{L^{\infty}(U_{F_m})}
        ,
        \| w_{m-1}\|_{L^{\infty}(T_{j(m)})}
    \right)
    .
 \end{align*}
 \color{JungleGreen}
 \begin{align*}
    \| w_{m} \|_{L^{p}(T_m)}^{p}
    &\leq
    \left( 
        3
    \right)^{p/q}
    \left( 
        \| u_{F_m} \|_{L^{p}(T_m)}^{p}
        +
        \frac{ \vol(T_m) }{ \vol(T_{j(m)}) }
        \| u_{F_{m}} \|_{L^{p}(T_{j(m)})}^{p}
        +
        \frac{ \vol(T_m) }{ \vol(T_{j(m)}) }
        \| w_{m-1}\|_{L^{p}(T_{j(m)})}^{p}
    \right)
    \\&
    \leq
    \left( 
        3
    \right)^{p/q}
    \left( 
        \max\left( 1, \frac{ \vol(T_m) }{ \vol(T_{j(m)}) } \right)
        \| u_{F_{m}} \|_{L^{p}(T_m \cup T_{j(m)})}^{p}
        +
        \frac{ \vol(T_m) }{ \vol(T_{j(m)}) }
        \| w_{m-1}\|_{L^{p}(T_{j(m)})}^{p}
    \right)
    .
 \end{align*}
 In the special case $p=1$ and $p=\infty$, ????
 \begin{align*}
    \| w_{m} \|_{L^{1}(T_m)}
    &\leq 
    \left( 
        \max\left( 1, \frac{ \vol(T_m) }{ \vol(T_{j(m)}) } \right)
        \| u_{F_{m}} \|_{L^{1}(T_m \cup T_{j(m)})}^{1}
        +
        \frac{ \vol(T_m) }{ \vol(T_{j(m)}) }
        \| w_{m-1}\|_{L^{1}(T_{j(m)})}^{1}
    \right)
    .
    \\
    \| w_{m} \|_{L^{2}(T_m)}^{2}
    &\leq
    3
    \left( 
        \max\left( 1, \frac{ \vol(T_m) }{ \vol(T_{j(m)}) } \right)
        \| u_{F_{m}} \|_{L^{2}(T_m \cup T_{j(m)})}^{2}
        +
        \frac{ \vol(T_m) }{ \vol(T_{j(m)}) }
        \| w_{m-1}\|_{L^{2}(T_{j(m)})}^{2}
    \right)
    \\
    \| w_{m} \|_{L^{\infty}(T_m)}
    &\leq 
    3
    \max\left( 
        \| u_{F_m} \|_{L^{\infty}(U_{F_m})}
        ,
        \| w_{m-1}\|_{L^{\infty}(T_{j(m)})}
    \right)
    .
 \end{align*}
 \color{MidnightBlue}
 \begin{align*}
    \| w_{m} \|_{L^{p}(T_m)}^{p}
    &\leq
    \left( 
        2 
        + 
        \frac{ \vol(T_m)^{\frac q p} }{ \vol(T_{j(m)})^{\frac q p} }
    \right)^{p/q}
    \left( 
        \| u_{F_m} \|_{L^{p}(T_m)}^{p}
        +
        \frac{ \vol(T_m) }{ \vol(T_{j(m)}) }
        \| u_{F_{m}} \|_{L^{p}(T_{j(m)})}^{p}
        +
        \| w_{m-1}\|_{L^{p}(T_{j(m)})}^{p}
    \right)
    \\&
    \leq
    \left( 
        2 
        + 
        \frac{ \vol(T_m)^{\frac q p} }{ \vol(T_{j(m)})^{\frac q p} }
    \right)^{p/q}
    \left( 
        \max\left( 1, \frac{ \vol(T_m) }{ \vol(T_{j(m)}) } \right)
        \| u_{F_{m}} \|_{L^{p}(T_m \cup T_{j(m)})}^{p}
        +
        \| w_{m-1}\|_{L^{p}(T_{j(m)})}^{p}
    \right)
    .
 \end{align*}
 In the special case $p=1$ and $p=\infty$, ????
 \begin{align*}
    \| w_{m} \|_{L^{1}(T_m)}
    &\leq 
    \max\left( 1, \frac{ \vol(T_m) }{ \vol(T_{j(m)}) } \right)
    \left( 
        \max\left( 1, \frac{ \vol(T_m) }{ \vol(T_{j(m)}) } \right)
        \| u_{F_{m}} \|_{L^{1}(T_m \cup T_{j(m)})} 
        +
        \| w_{m-1}\|_{L^{1}(T_{j(m)})} 
    \right)
    .
    \\
    \| w_{m} \|_{L^{2}(T_m)}^{2}
    &\leq 
    \left( 
        2 
        + 
        \frac{ \vol(T_m) }{ \vol(T_{j(m)}) }
    \right) 
    \left( 
        \max\left( 1, \frac{ \vol(T_m) }{ \vol(T_{j(m)}) } \right)
        \| u_{F_{m}} \|_{L^{2}(T_m \cup T_{j(m)})}^{2}
        +
        \| w_{m-1}\|_{L^{2}(T_{j(m)})}^{2}
    \right)
    \\
    \| w_{m} \|_{L^{\infty}(T_m)}
    &\leq 
    3
    \max\left( 
        \| u_{F_m} \|_{L^{\infty}(U_{F_m})}
        ,
        \| w_{m-1}\|_{L^{\infty}(T_{j(m)})}
    \right)
    .
 \end{align*}
 \color{black}
 
 
 
 
 
 
 
 We introduce the operator $B^{k}$, which can be written in several alternative forms
\begin{align*}
    B^{k} u(x) 
    &= 
    - |\Omega|^{-1}
    \int_{\Omega} \,(x-a) \lrcorner \int_1^\infty t^{{k}-1}\,u\left( a+t(x-a) \right) \,dt\,da
    \\&= 
    |\Omega|^{-1}
    \int_{\Omega} \,(x-a) \lrcorner \int_0^1 t^{-{k}-1}\,u\left( a+(x-a)/s \right) \,ds\,da
    .
\end{align*}
If $u \in C^{\infty}_{c}(\bbR^{n},\Alt^{k})$, 
then $B^{k} u$ is smooth with support in $\bbR^{n}\setminus\overline\Omega$ and $B^{k} u(x) = 0$ when $x \notin \Omega$. 
Since $\Omega$ is convex, $u \in C^{\infty}_{c}(\Omega,\Alt^{k})$ implies $\supp B^{k} u \subset\Omega$, 
Since $\Omega$ is bounded, $\supp u \subseteq \overline\Omega$ implies $\supp B^{k} u \subset\overline\Omega$. 
% The fact that $B^{k}$ indeed maps 
% $C^{\infty}_{c}(\bbR^{n},\Alt^{k})$ to $C^{\infty}_{c}(\bbR^{n},\Alt^{{k}-1})$ will be a the following theorem.

\begin{align*}
    \int_{\Omega}
    \left| B^{k} u(x) \right|^{p}
    dx
    &
    = 
    |\Omega|^{-1}
    \int_{\Omega}
    \int_{\Omega} 
    \int_0^1
    s^{(-{k}-1)p}\, |x-a|^{p} \left| u\left( a+(x-a)/s \right) \right|^{p} \,ds\,da
    dx
    \\&
    = 
    |\Omega|^{-1}
    \int_{\Omega}
    \int_{\Omega-a} 
    \int_{|z|/D}^{1}
    s^{(-{k}-1)p}\, |z|^{p} \left| u\left( a+z/s \right) \right|^{p} \,ds\,da
    dz
    \\&
    = 
    |\Omega|^{-1}
    \int_{\Omega}
    \int_{2\Omega} 
    \int_{|z|/D}^{1}
    s^{(-{k}-1)p}\, |z|^{p} \left| u\left( a+z/s \right) \right|^{p} \,ds\,da
    dz
    \\&
    = 
    |\Omega|^{-1}
    \int_{\Omega + z/s}
    \int_{2\Omega} 
    \int_{|z|/D}^{1} % y = a + z/s
    s^{(-{k}-1)p}\, |z|^{p} \left| u\left( y \right) \right|^{p} \,ds\,dy
    dz
    \\&
    = 
    |\Omega|^{-1}
    \int_{2\Omega}
    \int_{2\Omega} 
    \int_{|z|/D}^{1} % y = a + z/s
    s^{(-{k}-1)p}\, |z|^{p} \left| u\left( y \right) \right|^{p} \,ds\,dy
    dz
    .
\end{align*}



\begin{align*}
    G_\ell(x,y) 
    &\leq 
    \int_{0}^{\frac{D}{|x-y|}} \tau^{n-\ell} (\tau+1)^{\ell-1} \,d\tau
    \\
    &\leq 
    \frac{|x-y|}{D}
    \int_{0}^{\frac{D}{|x-y|}} \frac{D}{|x-y|} \tau^{n-\ell} (\tau+1)^{\ell-1} \,d\tau
    .
\end{align*}
\begin{align*}
    G_\ell(x,y)^{p} 
    &\leq 
    \frac{|x-y|}{D}
    \int_{0}^{\frac{D}{|x-y|}} \frac{D^{p}}{|x-y|^{p}} \tau^{pn-p\ell} (\tau+1)^{p\ell-p} \,d\tau
    \\&\leq 
    \frac{|x-y|}{D}
    \frac{D^{p}}{|x-y|^{p}} \int_{0}^{\frac{D}{|x-y|}} (\tau+1)^{pn-p} \,d\tau
    \\&\leq 
    \frac{D^{p-1}}{|x-y|^{p-1}} \left( \left( \frac{D}{|x-y|} + 1 \right)^{pn-p+1} - 1 \right)
    \\&\leq 
    \frac{D^{p-1}}{|x-y|^{p-1}} \left( 2^{pn-p+1} \left( \frac{D}{|x-y|} \right)^{pn-p+1} - 1 \right)
    \\&\leq 
    2^{pn-p+1} \frac{D^{pn}}{|x-y|^{pn}} - \frac{D^{p-1}}{|x-y|^{p-1}}
    .
\end{align*}





Moreover,  
\begin{align*}
    \int_{\Omega} \left| B_{\ell} u(x) \right|^{p} dx
    &=
    |\Omega|^{-1} 
    \int_{\Omega} 
    \int_{\Omega} 
    \left( 2^{pn-p+1} \frac{D^{pn}}{|x-y|^{pn}} - \frac{D^{p-1}}{|x-y|^{p-1}} \right)
    \left| (x-y)\lrcorner u(y) \right|^{p} 
    dy dx
    \\&=
    |\Omega|^{-1} 
    \int_{\Omega} 
    \int_{\Omega} \left( 2^{pn-p+1} D^{pn} |x-y|^{-pn+p} - D^{p-1} |x-y| \right) dx
    \left| u(y) \right|^{p} 
    dy
    .
\end{align*}






We use the radial integrals
\begin{align*}
    \int_{B_D(0)} |z|^{m} dz
    =
    \vol_{n-1}(S_1) \int_0^D r^{m} r^{n-1} dr
    =
    \vol_{n-1}(S_1) \int_0^D r^{m+n-1} dr
    =
    \vol_{n-1}(S_1) \frac{D^{m+n}}{m+n}
    .
\end{align*}







In order to see other properties of the operators, we apply a different change of variables. 
Let us write this in detail for the operator $R_{\ell}$. 
We substitute $y$ by the new variable $a=y+t(x-y)$.
Whence $da = (1-t)^n dy$ and $t(x-y) = a-y$ and $x-y = (a-y)/t$ and $x - (a-x)/t = y$
\begin{align*}
    \Poinc_{\ell} u(x) 
    &= 
    \int \int_{0}^{1} (t-1)^{n-\ell}t^{\ell-1} 
    \chi_{\Omega}\left(y+t(y-x)\right) 
    (x-y)\lrcorner u(y) \,dt\,dy 
    \\&= 
    \int \int_{0}^{1} (t-1)^{n-\ell}t^{\ell-1} 
    \chi_{\Omega}\left(y+(x-y)-(x-y)+t(y-x)\right) 
    (x-y)\lrcorner u(y) \,dt\,dy 
    \\&= 
    \int \int_{0}^{1} (t-1)^{n-\ell}t^{\ell-1} 
    \chi_{\Omega}\left(x+(t+1)(y-x)\right) 
    (x-y)\lrcorner u(y) \,dt\,dy 
    \\
    &= 
    \int \int_{0}^{1} (t-1)^{n-\ell}t^{\ell-1} (t+1)^{n}
    \chi_{\Omega}\left(a\right) 
    \frac{ x-a }{t+1}\lrcorner u( x + (a-x)/(t+1) ) \,dt\,dy 
\end{align*}
Then we substitute $1-s = 1/(t+1)$, that is, $s = t/(t+1)$:
Whence $t = \left( 1/s-1 \right)^{-1}$.
\begin{align*}
    \Poinc_{\ell} u(x) 
    &= 
    \int \int_{0}^{1} (t-1)^{n-\ell} \left( 1/s-1 \right)^{\ell-1} (t+1)^{n}
    \chi_{\Omega}\left(a\right) 
    \frac{ x-a }{t+1}\lrcorner u( x + (a-x)/(t+1) ) \,dt\,dy 
    \\
    &= 
    .
\end{align*}


Then we substitute $s = (t-1)/t$:
\begin{align*}
    &= 
    \int \int_{0}^{1} (t-1)^{n-\ell}t^{-n+\ell-1} 
    \chi_{\Omega}\left(a\right)
    \frac{(a-y)}{t} \lrcorner u(x+(a-x)/t) \,dt\,da 
    \\
    &= 
    \int \int_{0}^{1} (t-1)^{n-\ell}t^{\ell-1-n-1} \chi_{\Omega}\left(a\right)\, (x-a)\lrcorner u(x+(a-x)/t) \,dt\,da
    .
\end{align*}
Then we substitute $s = (t-1)/t$:
\begin{align*}
    \Poinc_{\ell} u(x) 
    &= 
    \int \int_{0}^{1} s^{n-\ell} (s-1)^{2} \chi_{\Omega}\left(a\right)\, (x-a)\lrcorner u( x+(s-1)(a-x) ) \frac{-1}{(s-1)^2}\,ds\,da
    \\
    &= 
    - \int \int_{0}^{1} s^{n-\ell} \chi_{\Omega}\left(a\right)\, (x-a)\lrcorner u( s(a-x) - a ) \,ds\,da
    \\
    &=
    - \int \chi_{\Omega}(a) \,(x-a)\lrcorner \int_{0}^{1} t^{\ell-1} u\left(a+t(x-a)\right)\,dt\,da\;
    .
\end{align*}
We copy a statement by McIntosh and Costabel:
\begin{align*}
    \Poinc_{\ell} u(x) 
    = 
    - \int \chi_{\Omega}(a) \,(x-a)\lrcorner \int_{0}^{1} t^{\ell-1} u\left(a+t(x-a)\right)\,dt\,da\;
    .
\end{align*}
$a = y + t(x-y)$
$s = \frac{t}{t-1}$
From that representation we can tell that whenever $u \in C^{\infty}(\bbR^{n})$ has support contained in $\overline\Omega$, we immediately have that $\Poinc_{\ell} u(x)$ is smooth with support contained in $\overline\Omega$ as well. 











From this form of $\Poinc_{\ell}$, because of the unbounded interval of integration in $t$, one cannot immediately conclude that $\Poinc_{\ell}$ maps ${C^\infty}$ functions to ${C^\infty}$ functions.
But if $u\in{C^\infty_c}(\bbR^n,\Alt^\ell)$, one sees that $\Poinc_{\ell} u$ is ${C^\infty}$ on 
$\bbR^n\setminus\supp\chi_{\Omega}$, and that 
$\Poinc_{\ell} u(x)=0$ unless $x$ lies in the starlike hull of $\supp u$ with respect to $B$. 
Thus if $\Omega$ is open and starlike with respect to $B$, then   
$u\in{C^\infty_c}(\Omega,\Alt^\ell)$ implies $\supp \Poinc_{\ell} u \subset\Omega$, and, if $\Omega$ is bounded, then
$u\in{C^\infty}_{\overline\Omega}(\bbR^n,\Alt^\ell)$ implies $\supp \Poinc_{\ell} u \subset\overline\Omega$. 
The fact that $\Poinc_{\ell}$ indeed maps 
${C^\infty_c}(\bbR^n,\Alt^\ell)$ to ${C^\infty_c}(\bbR^n,\Alt^{\ell-1})$ will be a consequence of Theorem~\ref{T:pseudo} below.




\subsection{Homotopy relations}

Cartan's formula for the Lie derivative of a differential form with respect to a vector field can be written as
$$
  \frac{d}{dt}F^*_t u = F^*_t \left(d(X_t\lrcorner u) + X_t\lrcorner du\right)\;,
$$
where $F^*_t$ denotes the pull-back by the flow $F_t$ associated with the vector field $X_t$. Here we consider the special case of the dilation flow with center $a$
$$
  F_t(x) = a+t(x-a) \quad\text{ with vector field } X_t=x-a\;,
$$
which gives a pull-back of
$$
  F^*_t u(x) = t^\ell \,u\left(a+t(x-a)\right) \quad \text{for an $\ell$-form } u \;.
$$
This leads to the formula
\begin{equation}
\label{eq:Cartan}
 \frac{d}{dt}(t^\ell u\left(a+t(x-a)\right) =
    d\Bigl(t^{\ell-1}(x-a)\lrcorner u\left(a+t(x-a)\right)\Bigr) +
    t^{\ell}(x-a)\lrcorner du\left(a+t(x-a)\right)
\end{equation}
which can also be verified elementarily from the formulas we gave in Section~\ref{S:notation}.

Integrating \eqref{eq:Cartan} from $0$ to $1$ and comparing with \eqref{eq:regPoincare}, 
we find the homotopy relations, valid for all $u\in{C^\infty_c}(\bbR^n,\Alt^\ell)$ 
\begin{equation}
\begin{aligned}
\label{eq:dR+Rd=1}
 dR_{\ell} u + R_{\ell+1} du &= u\; &&(1\le\ell\le n-1)\;;\\
 R_1 du &= u - \left(\chi_{\Omega},u\right)\; &&(\ell=0)\;;\\
 dR_nu &=u &&(\ell=n)\;.
\end{aligned}
\end{equation}
One could be tempted to integrate Cartan's formula from $1$ to $\infty$ and compare with \eqref{eq:Bogo}, thus formally obtaining a similar homotopy relation for $\Poinc_{\ell}$ directly. The result is indeed true except for $\ell=n$, but for a rigorous proof we prefer to use the duality relation \eqref{eq:dualRT} to deduce corresponding anticommutativity relations for $\Poinc_{\ell}$ from the relations \eqref{eq:dR+Rd=1} which are already proved. Here is what one obtains for $u\in{C^\infty_c}(\bbR^n,\Alt^\ell)$:
\begin{equation}
\begin{aligned}
\label{eq:dT+Td=1}
  d\Poinc_{\ell} u + \Poinc_{\ell+1} du &= u\; &&(1\le\ell\le n-1)\;;\\
  T_1 du &= u \; &&(\ell=0)\;;\\
 dT_n u &=u - (\int\! u)\star\chi_{\Omega}&&(\ell=n)\;.
\end{aligned}
\end{equation}
Here we consider $\chi_{\Omega}$ as an element of ${C^\infty_c}(\bbR^n,\Alt^0)$, so that for another $0$-form $u$ we have the $L^2$ scalar product $\left(\chi_{\Omega},u\right)=\int\chi_{\Omega}(a)u(a)da$, and $\star\chi_{\Omega}$ is the $n$-form $\chi_{\Omega}(x)dx_1\wedge\dots\wedge dx_n$. 

The formulas for the endpoints $\ell=0$ and $\ell=n$ correspond to the two extended de Rham complexes without boundary conditions and with compact support, see \eqref{eq:edRwobc} and \eqref{eq:edRwcs}. To see this, let us extend the definition of the exterior derivative by writing $\overline d$ for all the mappings of the complex
$$
 0 \to\bbR\to^\iota{C^\infty}(\overline\Omega,\Alt^0) \to^d {C^\infty}(\overline\Omega,\Alt^1)
 \to^d\cdots \to^d {C^\infty}(\overline\Omega,\Alt^n) \to 0
$$ 
and $\underline d$ for all the mappings of the complex
$$
 0 \to {C^\infty}_{\overline\Omega}(\bbR^n,\Alt^0) \to^d {C^\infty}_{\overline\Omega}(\bbR^n,\Alt^1) \to^d\cdots
 \to^d {C^\infty}_{\overline\Omega}(\bbR^n,\Alt^n) \to^{\iota^*} \bbR \to 0
$$
where $\iota$ is the inclusion mapping for constant functions and 
$\iota^*=(\star\iota)'$ denotes the integral $u\mapsto\int\!u$ for $n$-forms.

If we correspondingly extend the definitions of $R_{\ell}$ and $\Poinc_{\ell}$ by
$$
\begin{aligned}
  R_0u &:=\left(\chi_{\Omega},u\right)\,\text{ for $0$-forms }u\,, &
  R_{n+1} &:=0\,,\\
  T_{n+1}u &:= \star(u\chi_{\Omega}) \,\text{ for }u\in\bbR\,,  &
  T_0 &:= 0\,,
\end{aligned}
$$
then we can write the relations \eqref{eq:dR+Rd=1} and \eqref{eq:dT+Td=1} simply as
\begin{equation}\label{eq:R&T&d}
 \overline d\,R_{\ell} u + R_{\ell+1} \,\overline d u  = u
 \quad \text{ and }\quad
 \underline d\,\Poinc_{\ell} u + \Poinc_{\ell+1} \,\underline du = u\quad \text{ for all }\;
  0\le\ell\le n.
\end{equation}

 
 
 


We introduce the operator $B^{k}$, which is
\begin{equation}\label{eq:Bogo} % TODO: rewrite 
    B^{k} u(x) = - \int_{\Omega} \,(x-a) \lrcorner \int_1^\infty t^{{k}-1}\,u\left( a+t(x-a) \right) \,dt\,da.
\end{equation}
If $u \in C^{\infty}_{c}(\bbR^{n},\Alt^{k})$, 
then $B^{k} u$ is smooth with support in $\bbR^{n}\setminus\overline\Omega$ and $B^{k} u(x) = 0$ when $x \notin \Omega$. 
Since $\Omega$ is convex, $u \in C^{\infty}_{c}(\Omega,\Alt^{k})$ implies $\supp B^{k} u \subset\Omega$, 
Since $\Omega$ is bounded, $u \in C^{\infty}_{\overline\Omega}(\bbR^{n},\Alt^{k})$ implies $\supp B^{k} u \subset\overline\Omega$. 
% The fact that $B^{k}$ indeed maps C^{\infty}_{c}(\bbR^{n},\Alt^{k})$ to $C^{\infty}_{c}(\bbR^{n},\Alt^{{k}-1})$ will be a the following theorem.

 
 
 


























\begin{lemma}
    Let $\Omega \subseteq \bbR^{n}$. There exists a bounded mapping 
    \begin{align*}
     \Bogov_{\ell} : L^{p}(\Omega,\Alt^{\ell}) \rightarrow \bfW^{p}_{0}(\Omega,\Alt^{\ell-1})
    \end{align*}
    such that 
    \begin{align*}
        \| u \|_{L^{p}} \leq n A(n,\ell) \frac{ \vol_{n}(B_D(0)) }{|\Omega|} D \| u \|_{L^{p}(\Omega)}
    \end{align*}
    and such that $\cartan \Bogov_{\ell} u = u$ for all $u \in \cartan \bfW^{p}_{0}(\Omega,\Alt^{\ell-1})$.
\end{lemma}

\begin{lemma}
    Let $\Omega \subseteq \bbR^{n}$. There exists a bounded mapping 
    \begin{align*}
     \Bogov_{\ell} : L^{p}(\Omega,\Alt^{\ell}) \rightarrow \bfW^{p}_{0}(\Omega,\Alt^{\ell-1})
    \end{align*}
    such that 
    \begin{align*}
        \| u \|_{L^{p}} \leq n A(n,\ell) \frac{ \vol_{n}(B_D(0)) }{|\Omega|} D \| u \|_{L^{p}(\Omega)}
    \end{align*}
    and such that $\cartan \Bogov_{\ell} u = u$ for all $u \in \cartan \bfW^{p}_{0}(\Omega,\Alt^{\ell-1})$.
\end{lemma}





There exists a sequence $u_{i} \in C^{\infty}(\overline\Omega,\Alt^{\ell})$ that converges to $u$ in $W^{p}(\Omega,\Alt^{\ell})$
and which satisfies $\cartan u_{i} = 0$. Let $w \in L^{p}(\Omega,\Alt^{\ell-1})$ be the limit of $\Poinc_{\ell} u_{i}$.
We know that $\Poinc_{\ell} u_{i} \in C^{\infty}(\overline\Omega,\Alt^{\ell-1})$. 
Given any $v \in C^{\infty}_{c}(\Omega,\Alt^{n-\ell-1})$, we verify 
\begin{align*}
    \int_{\Omega} w \wedge \cartan v
    &=
    \lim_{i} \int_{\Omega} \Poinc_{\ell} u_{i} \wedge \cartan v
    \\&=
    \lim_{i} \int_{\Omega} \cartan \Poinc_{\ell} u_{i} \wedge v
    \\&=
    \lim_{i} \int_{\Omega} u_{i} \wedge v
    =
    \int_{\Omega} u \wedge v
    .
\end{align*}
That means $w \in W^{p}(\Omega,\Alt^{\ell-1})$ with $\cartan w = u$. 








\section*{Theorem 8.1}

\noindent \textbf{Theorem 8.1 (Discrete Poincaré inequality).} There holds
\begin{align*}
    \|g\|^2_{0,\Omega} 
    \leq 
    C_P |g|^2_{1,T} 
    + 
    \frac{4}{|\Omega|} \left( \int_{\Omega} g(x) \, dx \right)^2
        \quad 
    \forall g \in W(T_h), \forall h > 0
\end{align*}
with
\begin{align*}
C_P = 4C_d C_\Omega \frac{C_{d,T}}{\kappa_T} [\text{diam}(\Omega)]^2 + 8c_d h^2,
\end{align*}
where \(C_\Omega\) is given by (27) when \(\Omega\) is convex and by (28) otherwise, \(C_{d,T}\) is given by (12) when Assumption (B) is satisfied and by (15) in the general case, \(c_d\) is given by (22), and \(C_d\) is given by (26).
\\

\noindent Obviously
\begin{align*}
    \|g\|_{0,\Omega} 
    \leq 
    \|g - I(g)\|_{0,\Omega} + \|I(g)\|^2_{0,\Omega}
    ,
\end{align*}
\begin{align*}
    \|g\|^2_{0,\Omega} 
    &\leq 
    2\|g - I(g)\|^2_{0,\Omega} + 2\|I(g)\|^2_{0,\Omega} 
    ,
    \|g\|_{0,\Omega} 
    &\leq
    \sqrt{2}
    \sqrt{ \|g - I(g)\|^2_{0,\Omega} + 2\|I(g)\|^2_{0,\Omega} }
    .
\end{align*}
We apply the cellwise local approximation estimate, which coincides with the cellwise Poincare inequality:
\begin{align*}
    \|g - I(g)\|_{0,\Omega}
    &\leq 
    \frac{ h_T }{\pi}
    \| \nabla_\calT g \|_{0,\Omega}
    ,
    \|g - I(g)\|_{0,\Omega}
    &\leq 
    \frac{ h_T }{\pi}
    \| \nabla_\calT g \|_{0,\Omega}
    .
\end{align*}



\begin{align}
        v_0 + \langle v_k - v_0, v_{k-1} - v_0, \dots, v_1 - v_0 \rangle,
        \\
        \langle v_n - v_{k+1}, \dots, v_n - v_{n-1} \rangle + v_{0},
        \\
        v_0 + \sum_{i=1}^{k} \lambda_{i} ( v_i - v_0 ),
        \\
        v_0 + \sum_{i=k+1}^{n} \lambda_{i} ( v_n - v_i ),
        \\
        x = v_0 + \sum_{i=1}^{k} \lambda_{i} ( v_i - v_0 ) + \sum_{i=k+1}^{n} \lambda_{i} ( v_n - v_i ),
    \end{align}
    \begin{align}
        v_0 \left( 1 - \sum_{i=1}^{k} \lambda_{i} \right)
        +
        \sum_{i=1}^{k} \lambda_{i} v_i 
        +
        \sum_{i=k+1}^{n-1} \lambda_{i} v_n
        -
        \sum_{i=k+1}^{n-1} \lambda_{i} v_i
    \end{align}
    \begin{align}
        v_0 \left( 1 - \sum_{i=1}^{k} \lambda_{i} \right)
        +
        \sum_{i=1}^{k} \lambda_{i} v_i 
        +
        \sum_{i=k+1}^{n-1} (-\lambda_{i}) v_i
        +
        \sum_{i=k+1}^{n-1} \lambda_{i} v_n
    \end{align}


    
    
    
    
    
    
    
    
    
    
    
    
    
    
    
    
    
    
    
    
    
    
    
    
    Bogovskii was the first establish the 
    on whose work the following presentation is based. 
    In contrast to the aforementioned reference, the goal of our discussion is more modest:
    we merely need them over convex sets and are only interested in their operator norms between Lebesgue spaces.
    However, we derive explicit bounds for the operator norms, which are of independent interest. 
    
    It can be identified as a formal adjoint to the Poincar\'e operator in exterior calculus. 
    The Bogovski\u{\i}-type potential operator first appeared for the mathematical discussion of hydrodynamical models~\cite{bogovskii1979solution} and has since been extended to exterior calculus. 
    The regularized Bogovski\u{\i} potential operator was studied extensively by Costabel and McIntosh.
    These operators are defined over sets star-shaped with respect to a ball. 
    
    The generalization to differential forms has been discussed extensively by Costabel and McIntosh~\cite{costabel2010bogovskiui},
    on whose work the following presentation is based. 
    In contrast to the aforementioned reference, the goal of our discussion is more modest:
    we merely need them over convex sets and are only interested in their operator norms between Lebesgue spaces.
    However, we derive explicit bounds for the operator norms, which are of independent interest. 

    
    Their analysis identifies the potential as a pseudodifferential operator of negative order and utilizes tools of harmonic analysis. However, no operator norm estimates are given in the literature, to the best of our best knowledge.
    We study a variation of this potential operator over convex sets.
    We simply use a constant averaging over the entire domain. 
    While the resulting Bogovski\u{\i}-type potential has more moderate regularity properties, 
    this simplification suits our purposes and enables some simple estimates for the continuity constant of the potential operator in Lebesgue norms. 


For jede Matrix $M \in \mathbb R^{n \times n}$ schreiben wir die Reihen als $r_1, r_2, \dots, r_n \in \bbR^{n}$.
Die maximale Euklidische Norm der Reihenvektoren
\begin{align*}
    \| M \|_{\star} := \max\limits_{1 \leq i \leq n} \| r_i \|_{2}
\end{align*}
defininiert eine Norm auf den $n \times n$ Matrizen.

I am looking for the smallest constant $C > 0$ such that 
\begin{align*}
    \| M \|_{2} \leq C \| M \|_{\star}.
\end{align*}
Es ist bekannt dass die Frobenius norm eine obere Schranke ist:
\begin{align*}
    \| M \|_{2} = \sigma_{1}(M) \leq \sum_{i=1}^{n} \sigma_{i}(M) = \| M \|_{F}
\end{align*}
wo $\sigma_{1}(M) \geq \dots \geq \sigma_{n}(M)$ die Singularwerte darstellen. Folglich,
\begin{align*}
    \| M \|_{F} = \sqrt{\sum_{i=1}^{n} \| r_i \|_{2} } \leq \sqrt{n} \max\limits_{1 \leq i \leq n} \| r_i \|_{2}
\end{align*}
Kann die Konstante $\sqrt{n}$ noch verbessert werden?























On the other hand, by the reverse triangle inequality,
    \begin{align}
        \left| F(x) - F(y) \right|
        &\geq 
        \left\| 
            \frac{x}{\|x\|}
            \sum_{i=1}^{n} \lambda_{i}(x)
            -
            \frac{y}{\|y\|}
            \sum_{i=1}^{n} \lambda_{i}(y)
        \right\|
        \\&
        \geq 
        \left| 
            \frac{\|x\|}{\|x\|}
            \sum_{i=1}^{n} \lambda_{i}(x)
            -
            \frac{\|y\|}{\|y\|}
            \sum_{i=1}^{n} \lambda_{i}(y)
        \right|
        \\&
        = 
        \left| 
            \sum_{i=1}^{n} \lambda_{i}(x)
            -
            \sum_{i=1}^{n} \lambda_{i}(y)
        \right|
        \\&
        = 
        \left| 
            \vec{1}
            \cdot 
            A^{-1}( x - y )
        \right|
    \end{align}
    Note that $A^{-1}( x - y ) \in \Delta^{n} - \Delta^{n}$. 
    \begin{align}
        \left| F(x) - F(y) \right|
        &\geq 
        \left\| 
            \frac{x}{\|x\|}
            \left( 1 - \lambda_{0}(x) \right) 
            -
            \frac{y}{\|y\|}
            \left( 1 - \lambda_{0}(y) \right) 
        \right\|
        \\&
        \geq 
        \left\| 
            \frac{x}{\|x\|}
            \left( 1 - \lambda_{0}(x) \right) 
            -
            \frac{y}{\|y\|}
            \left( 1 - \lambda_{0}(y) \right) 
        \right\|
    \end{align}
    %% \todo{finish this estimate somehow}
    
    
    
    
    
    
    
    
    
    
    \begin{proof}
\begin{itemize}
    \item 
    % \todo{Choose hyperplane that puts T on one side}
    We use Lemma~\ref{lemma:oppositesubsimplex}
    \item 
    % \todo{Align barycenter of opposite subsimplex with coordinate axis}
    \item 
    % \todo{opposite simplex cast to disk}
    \item 
    % \todo{roll out rest of sphere onto disk}
    \item 
    % \todo{estimate reflection}
\end{itemize}
\end{proof}


















    We use the following auxiliary result. 
    Suppose that $\|\cdot\|_{\alpha}$ and $\|\cdot\|_{\alpha}$ are two norms on a Banach space.
    Let 
    $$
        F_{\alpha,\beta}(x) = \| x \|_{\alpha} \frac{ x }{ \| x \|_{\beta} }.
    $$
    Then 
    \begin{align*}
        &
        \left\| 
            \| x \|_{\alpha} \frac{ x }{ \| x \|_{\beta} } 
            - 
            \| y \|_{\alpha} \frac{ y }{ \| y \|_{\beta} } 
        \right\|_{\beta}
        \\&\qquad 
        \leq 
        C_{\beta}^{\alpha} 
        \left\| 
            \| x \|_{\alpha} \frac{ x }{ \| x \|_{\beta} } 
            - 
            \| y \|_{\alpha} \frac{ y }{ \| y \|_{\beta} } 
        \right\|_{\alpha}
        \\&\qquad 
        \leq 
        C_{\beta}^{\alpha} 
        \left\| 
            \| x \|_{\alpha} \frac{ x }{ \| x \|_{\beta} } 
            - 
            \| y \|_{\alpha} \frac{ x }{ \| y \|_{\beta} } 
        \right\|_{\alpha}
        +
        C_{\beta}^{\alpha} 
        \left\| 
            \| y \|_{\alpha} \frac{ x }{ \| y \|_{\beta} } 
            - 
            \| y \|_{\alpha} \frac{ y }{ \| y \|_{\beta} } 
        \right\|_{\alpha}
        \\&\qquad 
        \leq 
        C_{\beta}^{\alpha} 
        \frac{ \| x \|_{\alpha} }{ \| x \|_{\beta} } 
        \left| 
            \| x \|_{\alpha}
            - 
            \| y \|_{\alpha}
        \right|
        +
        C_{\beta}^{\alpha} 
        \frac{ \| y \|_{\alpha} }{ \| y \|_{\beta} } 
        \left\| x - y \right\|_{\alpha}
        \\&\qquad 
        \leq 
        C_{\beta}^{\alpha} 
        C_{\alpha}^{\beta} 
        \left\| x - y \right\|_{\alpha}
        +
        C_{\beta}^{\alpha} 
        C_{\alpha}^{\beta} 
        \left\| x - y \right\|_{\alpha}
        .
    \end{align*}
    \begin{align*}
        &
        \| 
            F(x) - F(y) 
        \|_{\beta}
        \\&\leq 
        \left\| 
            \| x \|_{\alpha} \frac{ x }{ \| x \|_{\beta} } 
            - 
            \| y \|_{\alpha} \frac{ y }{ \| y \|_{\beta} } 
        \right\|_{\beta}
        \\&\leq 
        \left\| 
            \| x \|_{\alpha} \frac{ x }{ \| x \|_{\beta} } 
            - 
            \| x \|_{\alpha} \frac{ y }{ \| x \|_{\beta} } 
            + 
            \| x \|_{\alpha} \frac{ y }{ \| x \|_{\beta} } 
            - 
            \| y \|_{\alpha} \frac{ y }{ \| x \|_{\beta} } 
            + 
            \| y \|_{\alpha} \frac{ y }{ \| x \|_{\beta} } 
            - 
            \| y \|_{\alpha} \frac{ y }{ \| y \|_{\beta} } 
        \right\|_{\beta}
        \\&\leq 
        \left\| 
            \| x \|_{\alpha} \frac{ x }{ \| x \|_{\beta} } 
            - 
            \| x \|_{\alpha} \frac{ y }{ \| x \|_{\beta} } 
        \right\|_{\beta}
            + 
        \left\| 
            \| x \|_{\alpha} \frac{ y }{ \| x \|_{\beta} } 
            - 
            \| y \|_{\alpha} \frac{ y }{ \| x \|_{\beta} } 
        \right\|_{\beta}
            + 
        \left\| 
            \| y \|_{\alpha} \frac{ y }{ \| x \|_{\beta} } 
            - 
            \| y \|_{\alpha} \frac{ y }{ \| y \|_{\beta} } 
        \right\|_{\beta}
        \\&\leq 
        \frac{ \| x \|_{\alpha} }{ \| x \|_{\beta} } 
        \left\| 
            x - y
        \right\|_{\beta}
            + 
        \frac{ \| y \|_{\beta} }{ \| x \|_{\beta} } 
        \left| \| x \|_{\alpha} - \| y \|_{\alpha} \right|
        + 
        \| y \|_{\alpha} \| y \|_{\beta} 
        \left| 
            \frac{ 1 }{ \| x \|_{\beta} } 
            - 
            \frac{ 1 }{ \| y \|_{\beta} }  
        \right|
        \\&\leq 
        \frac{ \| x \|_{\alpha} }{ \| x \|_{\beta} } 
        \left\| 
            x - y
        \right\|_{\beta}
        + 
        % BEcause w l o g 
        \left| \| x \|_{\alpha} - \| y \|_{\alpha} \right|
        + 
        \| y \|_{\alpha} \| y \|_{\beta} 
        \frac{ \left| \| x \|_{\beta} - \| y \|_{\beta} \right| }{ \| x \|_{\beta} \| y \|_{\beta} }
        \\&\leq 
        \frac{ \| x \|_{\alpha} }{ \| x \|_{\beta} } 
        \left\| 
            x - y
        \right\|_{\beta}
            + 
        % BEcause w l o g 
        \left| \| x \|_{\alpha} - \| y \|_{\alpha} \right|
            + 
        \| y \|_{\alpha} \| y \|_{\beta} 
        \frac{ \left| \| x \|_{\beta} - \| y \|_{\beta} \right| }{ \| x \|_{\beta} \| y \|_{\beta} }
        \\&\leq 
        \frac{ \| x \|_{\alpha} }{ \| x \|_{\beta} } 
        \left\| x - y \right\|_{\beta}
            + 
        % BEcause w l o g 
        \left\| x - y \right\|_{\alpha}
            + 
        \| y \|_{\alpha} \| y \|_{\beta} 
        \frac{ \left\| x - y \right\|_{\beta} }{ \| y \|_{\beta} \| y \|_{\beta} }
        \\&\leq 
        C^{\alpha}_{\beta} 
        C^{\beta}_{\alpha}
        \left\| x - y \right\|_{\alpha}
            + 
        % BEcause w l o g 
        \left\| x - y \right\|_{\alpha}
            + 
        C^{\alpha}_{\beta} 
        C^{\beta}_{\alpha}
        \left\| x - y \right\|_{\alpha}
        .
    \end{align*}
    % \item 
    Consider any simplex $T$ with one vertex $v_0$ at the origin. 
    Let $\lambda_{0}, \lambda_{1}, \dots, \lambda_{n}$ be the barycentric coordinates. 
    The mapping 
    \begin{align}
        F(x) = \frac{x}{\|x\|} \left( \sum_{i=1}^{n} \lambda_{i}(x) \right) = \frac{x}{\|x\|} \left( 1 - \lambda_{0}(x) \right)
    \end{align}
    maps $T$ onto a ``simplex with a curved face'' inside the unit ball. We analyze its Lipschitz properties.
    On the one hand, 
    \begin{align}
        &
        F(x) - F(y)
        \\&
        =
        \frac{x}{\|x\|} \left( \sum_{i=1}^{n} \lambda_{i}(x) \right)
        -
        \frac{y}{\|x\|} \left( \sum_{i=1}^{n} \lambda_{i}(x) \right)
        \\&\qquad 
        +
        \frac{y}{\|x\|} \left( \sum_{i=1}^{n} \lambda_{i}(x) \right)
        -
        \frac{y}{\|x\|} \left( \sum_{i=1}^{n} \lambda_{i}(y) \right)
        \\&\qquad 
        +
        \frac{y}{\|x\|} \left( \sum_{i=1}^{n} \lambda_{i}(y) \right)
        -
        \frac{y}{\|y\|} \left( \sum_{i=1}^{n} \lambda_{i}(y) \right)
        .    
    \end{align}
    Without loss of generality, $\|y\| \leq \|x\|$. 
    \begin{align}
        &
        \left\| F(x) - F(y) \right\|
        \\&
        \leq
        \left\| x - y \right\|
        \frac{\sum_{i=1}^{n} \lambda_{i}(x)}{\|x\|}
        \\&\qquad 
        +
        \left| \sum_{i=1}^{n} \lambda_{i}(x) - \sum_{i=1}^{n} \lambda_{i}(y) \right|
        +
        \|y\| \left( \sum_{i=1}^{n} \lambda_{i}(y) \right)
        \frac{ \left| \|y\| - \|x\| \right| }{\|x\| \|y\|}
        \\&\leq
        \left\| x - y \right\|
        \sqrt{n} \sigma_{\min}(A^{-1})
        +
        \sqrt{n} \sigma_{\min}(A^{-1}) \left\| x - y \right\|
        +
        \left\| x - y \right\|
        \sqrt{n} \sigma_{\min}(A^{-1})
        \\&\leq
        3\sqrt{n} \sigma_{\min}(A^{-1})
        \left\| x - y \right\|
        .    
    \end{align}
    \begin{align}
        &
        \left\| F(x) - F(y) \right\|
        \\&
        \leq
        \left\| x - y \right\|
        \frac{\sum_{i=1}^{n} \lambda_{i}(x)}{\|x\|}
        \\&\qquad 
        +
        \left| \sum_{i=1}^{n} \lambda_{i}(x) - \sum_{i=1}^{n} \lambda_{i}(y) \right|
        +
        \|y\| \left( \sum_{i=1}^{n} \lambda_{i}(y) \right)
        \frac{ \left| \|y\| - \|x\| \right| }{\|x\| \|y\|}
        \\&
        \leq
        \left\| x - y \right\|
        \frac{\sum_{i=1}^{n} \lambda_{i}(x)}{\|x\|}
        \\&\qquad 
        +
        \left| \lambda_{0}(x) - \lambda_{0}(y) \right|
        +
        \left( \sum_{i=1}^{n} \lambda_{i}(y) \right)
        \frac{ \left| \|y\| - \|x\| \right| }{\|x\|}
        \\&\leq
        \leq
        \left\| x - y \right\|
        \frac{\sum_{i=1}^{n} \lambda_{i}(x)}{\|x\|}
        \\&\qquad 
        +
        \left| \lambda_{0}(x) - \lambda_{0}(y) \right|
        +
        \left( \sum_{i=1}^{n} \lambda_{i}(y) \right)
        \frac{ \| x - y \| }{\|x\|}
        \\&\leq
        \leq
        \left\| x - y \right\|
        \frac{ \lambda_{0}(x) }{\|x\|}
        +
        \left| \lambda_{0}(x) - \lambda_{0}(y) \right|
        +
        \left( \lambda_{0}(y) \right)
        \frac{ \| x - y \| }{\|y\|}
        .    
    \end{align}
    On the one hand,
    we compute the Jacobian  
    \begin{align}
        \Jacobian F(x) 
        = 
        \frac{ \Id \|x\| - x \|x\|^{-1} x^t }{\|x\|^2} 
        \left( \sum_{i=1}^{n} \lambda_{i}(x) \right)
        + 
        \frac{x}{\|x\|} \left( \sum_{i=1}^{n} \nabla\lambda_{i}(x) \right)
        .
    \end{align}
    
    

    
    
    
    
    
    
    
    
    
    
    
    
    
    
    
    
    
    
    
    
    
    
    
    
    
    
    
    
    
    
    
    
    
    
    
    
    
    
    
    
    
    
    
    
    
    
    
    
% Please cite the original version:
% Alestalo, P., & Trotsenko, D. A. (2018). Radial extensions of bilipschitz maps between unit spheres. Siberian
% Electronic Mathematical Reports, 15, 839-843. https://doi.org/10.17377/semi.2018.15.071
% Theorem 3: radial extension of bi lipschitz between unit spheres, with same constant 


We recall a specialization of this result.

\begin{lemma}\label{theorem:starreflection:facepatch}

\end{lemma}





% https://mathoverflow.net/questions/232302/lipschitz-constant-of-cental-projection-of-unit-ball-to-surface-of-convex-body
% https://eudml.org/doc/258483
% https://math.stackexchange.com/questions/1476085/uniformly-convex-approximation-of-convex-domain
% SMOOTH EXHAUSTION FUNCTIONS IN CONVEX DOMAINS
\begin{lemma}
    Let $K \subseteq \bbR^{n}$ be a compact set star-shaped with respect to the origin 
    that contains the ball $B_{r}(0)$ and that is contained in the ball $B_{R}(0)$. Then the function 
    \begin{align*}
        f : S \rightarrow \partial K, \quad x \mapsto \arg( \partial K \cap \bbR x )
    \end{align*}
    satisfies 
    \begin{align*}
        R^{2}/r % euclidean 
    \end{align*}
    If $K$ is convex, then 
    \begin{align*}
        g : S \rightarrow \bbR, \quad x \mapsto \| \bbR x \cap \partial K \|
    \end{align*}
    satisfies 
    \begin{align*}
        R \sqrt{ R^{2} / r^{2} - 1 } % angular distance 
        \\
        \frac R r \sqrt{ R^{2} - r^{2} } % angular distance 
    \end{align*}
\end{lemma}
\begin{proof}
    .
\end{proof}


    
    
    
    
    
    
    
    
    
    
\end{document}
    
    
    
    
    
    
    
    
    
    
    

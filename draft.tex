% Develop construction of complexes via adding patches 
\subsection{Patch coverings}

We are interested in exhausting a simplicial complex in a controlled manner. 
Suppose that $\calT$ is a manifold-like $n$-dimensional simplicial complex. 
We call $n$-simplices $S,T \in \calT$ \emph{face-connected} if there exists a sequence $S_0=S,S_1,\dots,T=S_m$ such that $S_{i} \cap S_{i-1}$ is a face for all $1 \leq i \leq m$. Clearly, face-connected is an equivalence relation. A \emph{connected component} of $\calT$ is an equivalence class under the face-connected relation, and we call $\calT$ connected if all its $n$-simplices are face-connected. 
In a manifold-like simplicial complex, two $n$-simplices have non-empty intersection if and only if they are face-connected. 

It is a consequence of Lemma~\ref{lemma:characterizationofmanifoldcomplexes} that,
if $\calT$ is a manifold-like simplicial complex, then $\patch_{\calT}(S)$ is connected for all $S \in \calT$.
Moreover, a manifold-like simplicial complex $\calT$ is connected if and only if $\bigcup\calT$ is a connected topological space. 

We let \emph{$k$-patching} refer to enumerations $S_1, S_2, \dots$ of $\subsimplex_{k}(\calT)$
such that for any $0 \leq m$, the union 
$\calU_{m} := \patch_{\calT}(S_0) \cup \patch_{\calT}(S_1) \cup \patch_{\calT}(S_2) \cup \dots \cup \patch_{\calT}(S_m)$
shares an $n$-simplex with the patch $\patch_{\calT}(S_{m+1})$.
We call the $k$-patching \emph{manifold-like} if $\calU_{m}$ is manifold-like for all $0 \leq m$.
Clearly, a $k$-patching only exists if $k < n$ and if $\calT$ is connected. 
We refer the reader to Figure~\ref{figure:illustrationpatching} for an illustration.

\begin{lemma}
    A connected manifold-like $n$-dimensional simplicial complex $\calT$ admits a $k$-patching for any $k < n$.
\end{lemma}
\begin{proof}
    We define an undirected graph by letting $\subsimplex_{k}(\calT)$ be the set of nodes 
    and connecting any $S, S' \in \subsimplex_{k}(\calT)$ if there exists $T \in \subsimplex_{k+1}(\calT)$ with $S, S' \subseteq T$. 
    Note that in this case $\patch_{\calT}(S) \cap \patch_{\calT}(S') = \patch_{\calT}(T)$. 
    Since $\calT$ is connected, the graph $\calG$ is connected. 
    
    We let $S_0, S_1, S_2, \dots$ be an enumeration of the $k$-simplices generated by a breadth-first traversal of the graph $\calG$,
    starting at some arbitrary but fixed $S_0 \in \calT$. 
    Since $\calG$ is locally finite and connected, 
    this breadth-first traversal includes all $k$-simplices of $\calT$.
    
    Let $m \in \bbZ$ with $m \geq 1$. If $S_m$ has distance $d \in \bbN$ from $S_1$ in the graph $\calG$,
    then there exists $0 \leq l \leq m$ such that $S_l$ has distance $d-1$ from $S_1$ and is connected to $S_m$. 
    It follows that the sequence $S_0, S_1, S_2, \dots$ is a $k$-patching of $\calT$. 
\end{proof}

The preceding lemma is generally not true when the $k$-patching is required to be manifold-like,
as can be verified from some simple examples;~see Figure~\ref{figure:annuluscounterexample}.

\color{Emerald}
\begin{align}
    \min\limits_{ c \in \bbR }
    \| u - c \|_{L^{p}(\Omega)}
    \leq 
    % 2^{ - \frac{p-1}{p}}
    %2^{ - 1 + \frac{1}{p}}
    \frac{1}{ 2^{ 1 - \frac{1}{p} } }
    \diam(\Omega)
    \| \nabla u \|_{L^{p}(\Omega)}
    ,
    \quad 
    u \in W^{1,p}(\Omega)
    ,
\end{align}

\begin{proof}
    Given $z \in \Omega$, we define 
    \[
        w(x,z) 
        := 
        \int_0^1 \bff( z + t(x-z) ) \cdot (x-z)
        =
        u(x) - u(z)
        .
    \]
    Subsequently,
    \begin{align*}
        w(x) = |\Omega|^{-1} \int_{\Omega} w(x,z) dz.
    \end{align*}
    Clearly, $\nabla_{x} u(x) = \nabla_{x} w(x,z) = \nabla_{x} w(x)$.
    Next, 
    \begin{align*}
        \int_{\Omega} | w(x) |^{p} dx
        &\leq 
        \int_{\Omega} |\Omega|^{-1} \int_{\Omega} | w(x,z) |^{p} dz dx
        \\&\leq 
        \int_{\Omega} |\Omega|^{-1} \int_{\Omega} \int_0^1 | \bff( z + t(x-z) ) \cdot (x-z) |^{p} dz dx
        \\&= 
        |\Omega|^{-1} \int_{\Omega} \int_{\Omega} \int_0^{D/|x-z|} | \bff( z + t(x-z) ) \cdot (x-z) |^{p} dz dx
        .
    \end{align*}

\end{proof}
\color{black}

\begin{lemma}
    Suppose that $T, T'$ are two $n$-simplices whose intersection is a common face $F := T_1 \cap T_2$. Then 
    \begin{align*}
        \| u - u_{\Omega} \|_{L^p(\Omega)} 
        &
        \leq 
        \frac{2}{((n-1)!)^n }
        \left( \frac{\diam(T)^n}{\vol(T)} \right)^n
        \left( \frac{\diam(T')^n}{n! \vol(T')} \right)^n
        \diam(\Omega)
        \| \nabla u \|_{L^p(\Omega)} 
    \end{align*}
    for any $u \in W^{1,p}(\Omega)$ with $p \in [1,\infty)$.
\end{lemma}
\begin{proof}
    Suppose a tetrahedron $T$ has vertices $v_0,v_1,\dots,v_n$ and let $F_0,F_1\dots,F_n$ be the faces opposite to those vertices. 
    The height $h(T,k)$ of the vertex $v_k$ in $T$ equals 
    \[
        h(T,k) := n \frac{\vol(T)}{\vol(F_k)}
    \]
    A lower bound for the height in terms of the volume and diameter of $T$ is 
    \[
        h(T,k) 
        = 
        n \frac{\vol(T)}{\diam(T)^n}
        \cdot 
        \frac{\diam(T)^n}{\vol(F_k)}
        \geq 
        n \frac{\vol(T)}{\diam(T)^n}
        \cdot 
        (n-1)! \frac{\diam(T)^n}{\diam(T)^{n-1}}
        \geq 
        \frac{\vol(T)}{\diam(T)^n}
        \cdot 
        n! \diam(T) 
        .
    \]
    Any subsimplex $S$ of $T$ adjacent to $v_k$ thus satisfies 
    \[
        \diam(S)
        \geq 
        \frac{\vol(T)}{\diam(T)^n}
        \cdot 
        n! \diam(T) 
        .
    \]
    We let $w \in F_0$ be the barycenter in the face opposite to $v_0$. 
    Then the convex combination of $w$ and any of the faces $F_1,\dots,F_n$ defines simplices $S_1,\dots,S_n$ which satisfy 
    \[
        \vol(S_1) = \dots = \vol(S_n) = \frac{\vol(T)}{n}
        .
    \]
    Consequently, 
    the height $h_k := $ of the vertex $w$ in $S_k$ equals 
    \[
        h_k := n \frac{\vol(S_k)}{\vol(F_k)} = \frac{\vol(T)}{\vol(F_k)}
    \]
    It follows that $T$ is star-shaped with respect to the intersection of $T$
    with the ball around $w$ of radius $h$, where $h := \min_{1 \leq k \leq n} h_k$.
    A lower bound for height in terms of the volume and diameter of $T$ is 
    \[
        h_k 
        = 
        \frac{\vol(T)}{\diam(T)^n}
        \cdot 
        \frac{\diam(T)^n}{\vol(F_k)}
        \geq 
        \frac{\vol(T)}{\diam(T)^n}
        \cdot 
        (n-1)! \frac{\diam(T)^n}{\diam(T)^{n-1}}
        \geq 
        \frac{\vol(T)}{\diam(T)^n}
        \cdot 
        (n-1)! \diam(T) 
        .
    \]
    Let now $T'$ be another $n$-simplex with vertices $v_0',v_1,\dots,v_n$.
    We construct $h'$ analogous to above and $h'' := \min(h,h')$. 
    Hence $\Omega := T \cup T'$ is star-shaped with respect to a ball around $w$ of radius $h''$.
    Using a result by Farweg and Rosteck,
    \begin{align*}
        \| u - u_{\Omega} \|_{L^p(\Omega)} 
        &
        \leq 
        2 \diam(\Omega)^n
        \omega_n^{ 1 - \frac 1 n }
        \frac{|\Omega|^{\frac{1}{n}}}{|\Ball(w,h'')|}
        \| \nabla u \|_{L^p(\Omega)} 
        \\&
        \leq 
        2 \diam(\Omega)^n
        \omega_n^{ - \frac 1 n }
        \frac{|\Omega|^{\frac{1}{n}}}{(h'')^n}
        \| \nabla u \|_{L^p(\Omega)} 
        .
    \end{align*}
    Without loss of generality, $\diam(T) \leq \diam(T')$. Now, 
    \begin{align*}
        \| u - u_{\Omega} \|_{L^p(\Omega)} 
        &
        \leq 
        \left( \frac{\diam(T)^n}{\vol(T)} \right)^n
        2 \diam(\Omega)^n
        \omega_n^{ - \frac 1 n }
        \frac{|\Omega|^{\frac{1}{n}}}{((n-1)!)^n \diam(T)^n }
        \| \nabla u \|_{L^p(\Omega)} 
        \\&
        \leq 
        \frac{2}{((n-1)!)^n }
        \left( \frac{\diam(T)^n}{\vol(T)} \right)^n
        \frac{ \diam(T')^n }{ \diam(T)^n }
        \diam(T')
        \| \nabla u \|_{L^p(\Omega)} 
        .
    \end{align*}
    We know that 
    \begin{align*}
        \diam(T')
        \leq 
        \frac{\diam(T')^n}{n! \vol(T')}
        \diam(F)
        \leq 
        \frac{\diam(T')^n}{n! \vol(T')}
        \diam(T)
        .
    \end{align*}
    The final estimate is
    \begin{align*}
        \| u - u_{\Omega} \|_{L^p(\Omega)} 
        &
        \leq 
        \frac{2}{((n-1)!)^n }
        \left( \frac{\diam(T)^n}{\vol(T)} \right)^n
        \left( \frac{\diam(T')^n}{n! \vol(T')} \right)^n
        \diam(\Omega)
        \| \nabla u \|_{L^p(\Omega)} 
        .
    \end{align*}
    This finishes the proof. 
\end{proof}


We use the following trace inequality, which can be found in the literature. 

\begin{lemma}\label{lemma:traceinequality}
    Let $p \in [1,\infty)$ and let $T$ be an $n$-dimensional simplex with a face $F$. Then 
    \begin{align*}
        \| u \|_{L^{p}(F)}^{p}
        \leq 
        \frac{|F|}{|T|} 
        \| u \|_{L^{p}(T)}^{p}
        +
        p \cdot \diam(T) 
        \frac{|F|}{|T|} 
        \| u \|_{L^{p}(T)}^{p-1}
        \| \nabla u \|_{\bfL^{p}(T)}
        .
    \end{align*}
    for all $u \in W^{1,p}(T)$. Moreover, 
    \begin{align*}
        \| u \|_{L^{\infty}(F)}
        \leq 
        \| u \|_{L^{\infty}(T)}
        .
    \end{align*}
    for all $u \in W^{1,\infty}(T)$. \qed
\end{lemma}

\begin{remark}
    Lemma~\cite{veeser2012poincare} has been stated by Veeser and Verfürth~\cite[Lemma 2.8]{veeser2012poincare} when $1 \leq p < \infty$. 
    The case $p = \infty$ is obvious because functions in $W^{1,\infty}(T)$ are already Lipschitz. 
    \color{red}\textbf{The following is probably wrong} 
    However, note that for any $p \in [1,\infty)$ and $u \in W^{1,p}(T)$ we have 
    \begin{align*}
        \| u \|_{L^{p}(F)}
        &\leq 
        \| u \|_{L^{p}(T)}^{\frac{p-1}{p}}
        \left(
            \frac{|F|}{|T|}
            \| u \|_{L^{p}(T)}
            +
            p \cdot \diam(T)
            \frac{|F|}{|T|}
            \| \nabla u \|_{\bfL^{p}(T)}
        \right)^{\frac{1}{p}}
        \\&\leq 
        \| u \|_{L^{p}(T)}^{\frac{p-1}{p}}
        \left( \frac{|F|}{|T|} \right)^{\frac 1 p}
        \left(
            \| u \|_{L^{p}(T)}^{\frac{1}{p}}
            +
            \sqrt[p]{p} \cdot \diam(T)^{\frac 1 p} 
            \| \nabla u \|_{\bfL^{p}(T)}^{\frac 1 p} 
        \right)
        .
    \end{align*}
    The limit as $p$ goes to infinity leads to the inequality as stated in the preceding lemma,
    which is therefore the natural limit case of the aforementioned trace inequality.
\end{remark}


\begin{lemma}
    Let $T_1$ and $T_{2}$ be two $n$-simplices such that $F = T_1 \cap T_2$ is a common subsimplex of dimension $n-1$. Then 
    \color{red} Here we estimate the Poincar\'e constant over a face patch.
\end{lemma}

\color{red}
Possible approaches: 
\begin{itemize}
 \item Take the trace jump (requires trace inequality with explicit constant) and extend the resulting constant function
 \item The union is star-shaped with respect to some (hopefully large) convex set, then use inequality for that situation (available in literature)
 \item Reflection trick
 \item Paper by Veeser and Verfuerth?
\end{itemize}
Most importantly, all constants must be explicitly computable!
We can do all these methods and compare the results. 
\color{black}
    
\begin{proof}
    We write $U := T_1 \cup T_2$. Let $u \in W^{1,p}(U)$.
    Suppose $u_1 \in W^{1,p}(T_1)$ and $u_2 \in W^{1,p}(T_2)$
    satisfy $\nabla u_1 = \nabla u_{|T_1}$ and $\nabla u_2 = \nabla u_{|T_2}$.
    Then there exists a constant $c \in \bbR$ such that $c = \trace_{T_1,F} u_1 - \trace_{T_1,F} u_2$.
    Consequently, if we define $u_2' = u_2 + c$, then we obtain a function $u = u_1 \cup u_2' \in W^{1,p}(T)$. 
    Notice that 
    \begin{align*}
        \| u \|_{L^{p}(U)}^{p}
        &\leq 
        \| u_1 \|_{L^{p}(T_1)}^{p}
        +
        \| u'_2 \|_{L^{p}(T_2)}^{p}
        \\
        &\leq 
        \| u_1 \|_{L^{p}(T_1)}^{p}
        +
        \left( 
            \| u_2 \|_{L^{p}(T_2)}
            +
            \left(\frac{|T_2|}{|F|}\right)^{\frac 1 p}
            \| c \|_{L^{p}(F)}
        \right)^{p}
    \end{align*}
    \begin{align*}
        \color{red}
        \| u'_2 \|_{L^{p}(T_2)}
        \leq 
        \| u_2 \|_{L^{p}(T_2)}
        +
        \| c \|_{L^{p}(T_2)}
        = 
        \| u_2 \|_{L^{p}(T_2)}
        +
        \left(\frac{|T_2|}{|F|}\right)^{\frac 1 p}
        \| c \|_{L^{p}(F)}
        .
    \end{align*}
    Recall that 
    \begin{align*}
        \| v \|_{L^{p}(F)}^{p}
        \leq 
        \frac{|F|}{|T|} 
        \| v \|_{L^{p}(T)}^{p}
        +
        p \cdot \diam(T) 
        \frac{|F|}{|T|} 
        \| v \|_{L^{p}(T)}^{p-1}
        \| \nabla v \|_{\bfL^{p}(T)}
        ,
        \quad 
        v \in W^{1,p}(T)
        .
    \end{align*}
    Now 
    \begin{align*}
        \| c \|_{L^{p}(F)}^{p}
        &\leq 
        2^{p-1}
        \| u_1 \|_{L^{p}(F)}^{p}
        +
        2^{p-1}
        \| u_2 \|_{L^{p}(F)}^{p}
    \end{align*}
    We thus proceed with 
    \begin{align*}
        \| u_1 \|_{L^{p}(F)}^{p}
        +
        \| u_2 \|_{L^{p}(F)}^{p}
        &
        \leq 
        \frac{|F|}{|T_1|} 
        \| u_1 \|_{L^{p}(T_1)}^{p}
        +
        p \cdot \diam(T_1) 
        \frac{|F|}{|T_1|} 
        \| u_1 \|_{L^{p}(T_1)}^{p-1}
        \| \nabla u_1 \|_{\bfL^{p}(T_1)}
        \\&\quad\quad
        +
        \frac{|F|}{|T_2|} 
        \| u_2 \|_{L^{p}(T_2)}^{p}
        +
        p \cdot \diam(T_2) 
        \frac{|F|}{|T_2|} 
        \| u_2 \|_{L^{p}(T_2)}^{p-1}
        \| \nabla u_2 \|_{\bfL^{p}(T_2)}
    \end{align*}
    If $p = 1$, then we summarize this as 
    \begin{align*}
        \| u \|_{L^{1}(U)} 
        &\leq 
        \| u_1 \|_{L^{1}(T_1)} 
        +
        \| u'_2 \|_{L^{1}(T_2)} 
        \\
        &\leq 
        \| u_1 \|_{L^{1}(T_1)} 
        +
        \| u_2 \|_{L^{1}(T_2)}
        +
        \left(\frac{|T_2|}{|F|}\right)
        \| c \|_{L^{1}(F)}
        \\
        &\leq 
        \| u_1 \|_{L^{1}(T_1)} 
        +
        \| u_2 \|_{L^{1}(T_2)}
        +
        \left(\frac{|T_2|}{|F|}\right)
        \| u_1 \|_{L^{1}(F)} 
        +
        \left(\frac{|T_2|}{|F|}\right)
        \| u_2 \|_{L^{1}(F)} 
        \\
        &\leq 
        \| u_1 \|_{L^{1}(T_1)} 
        +
        \| u_2 \|_{L^{1}(T_2)}
        \\&\qquad 
        +
        \left(\frac{|T_2|}{|F|}\right)
        \frac{|F|}{|T_1|} 
        \| u_1 \|_{L^{1}(T_1)} 
        +
        \diam(T_1) 
        \left(\frac{|T_2|}{|F|}\right)
        \frac{|F|}{|T_1|} 
        \| \nabla u_1 \|_{\bfL^{1}(T_1)}
        \\&\quad\quad
        +
        \left(\frac{|T_2|}{|F|}\right)
        \frac{|F|}{|T_2|} 
        \| u_2 \|_{L^{1}(T_2)} 
        +
        \diam(T_2) 
        \left(\frac{|T_2|}{|F|}\right)
        \frac{|F|}{|T_2|} 
        \| \nabla u_2 \|_{\bfL^{1}(T_2)}
        \\
        &\leq 
        \| u_1 \|_{L^{1}(T_1)} 
        +
        \| u_2 \|_{L^{1}(T_2)}
        \\&\qquad 
        +
        \frac{|T_2|}{|T_1|} 
        \| u_1 \|_{L^{1}(T_1)} 
        +
        \diam(T_1) 
        \frac{|T_2|}{|T_1|} 
        \| \nabla u_1 \|_{\bfL^{1}(T_1)}
        \\&\quad\quad
        +
        \| u_2 \|_{L^{1}(T_2)} 
        +
        \diam(T_2) 
        \| \nabla u_2 \|_{\bfL^{1}(T_2)}
        \\
        &\leq 
        \left( 1 + \frac{|T_2|}{|T_1|} \right)
        \| u_1 \|_{L^{1}(T_1)} 
        +
        2
        \| u_2 \|_{L^{1}(T_2)}
        +
        \diam(T_1) 
        \frac{|T_2|}{|T_1|} 
        \| \nabla u_1 \|_{\bfL^{1}(T_1)}
        +
        \diam(T_2) 
        \| \nabla u_2 \|_{\bfL^{1}(T_2)}
        .
    \end{align*}
    When $1 < p < \infty$, then Young's inequality implies 
    \begin{align*}
        \| u_1 \|_{L^{p}(F)}^{p}
        +
        \| u_2 \|_{L^{p}(F)}^{p}
        &
        \leq 
        \frac{|F|}{|T_1|} 
        \| u_1 \|_{L^{p}(T_1)}^{p}
        +
        \frac{|F|}{|T_1|} 
        \| u_1 \|_{L^{p}(T_1)}^{q(p-1)}
        +
        \frac p q \cdot \diam(T_1)^{p}
        \frac{|F|}{|T_1|} 
        \| \nabla u_1 \|_{\bfL^{p}(T_1)}^{p}
        \\&\quad\quad
        +
        \frac{|F|}{|T_2|} 
        \| u_2 \|_{L^{p}(T_2)}^{p}
        +
        \frac{|F|}{|T_2|} 
        \| u_2 \|_{L^{p}(T_2)}^{q(p-1)}
        +
        \frac p q \cdot \diam(T_2)^{p} 
        \frac{|F|}{|T_2|} 
        \| \nabla u_2 \|_{\bfL^{p}(T_2)}^{p}
        \\
        &
        \leq 
        \frac{|F|}{|T_1|} 
        \| u_1 \|_{L^{p}(T_1)}^{p}
        +
        \frac{|F|}{|T_1|} 
        \| u_1 \|_{L^{p}(T_1)}^{p}
        +
        (p-1) \cdot \diam(T_1)^{p} 
        \frac{|F|}{|T_1|} 
        \| \nabla u_1 \|_{\bfL^{p}(T_1)}^{p}
        \\&\quad\quad
        +
        \frac{|F|}{|T_2|} 
        \| u_2 \|_{L^{p}(T_2)}^{p}
        +
        \frac{|F|}{|T_2|} 
        \| u_2 \|_{L^{p}(T_2)}^{p}
        +
        (p-1) \cdot \diam(T_2)^{p} 
        \frac{|F|}{|T_2|} 
        \| \nabla u_2 \|_{\bfL^{p}(T_2)}^{p}
        .
    \end{align*}
    \color{red}The last terms vanish in the limit $p \rightarrow 1$. 
    It is not clear whether $p=1$ emerges as a limit case: 
    we have gained an additional $p$-norm of $u$ but applying the Poincare inequality introduces the Poincare constant, 
    whereas there is no such dependence on the Poincare constant in the case $p=1$ above. 
    \color{black}
    In summary, 
    \begin{align*}
        2^{1-p}
        \left(\frac{|T_2|}{|F|}\right)
        \| c \|_{L^{p}(F)}^{p}
        &
        \leq 
        2
        \frac{|T_2|}{|T_1|} 
        \| u_1 \|_{L^{p}(T_1)}^{p}
        +
        (p-1) \cdot \diam(T_1)^{p} 
        \frac{|T_2|}{|T_1|} 
        \| \nabla u_1 \|_{\bfL^{p}(T_1)}^{p}
        \\&\quad\quad\quad\quad
        +
        2
        \| u_2 \|_{L^{p}(T_2)}^{p}
        +
        (p-1) \cdot \diam(T_2)^{p} 
        \| \nabla u_2 \|_{\bfL^{p}(T_2)}^{p}
        .
    \end{align*}
    We thus find 
    %{\tiny
    \begin{align*}
        \| u \|_{L^{p}(U)}^{p}
        &\leq 
        \| u_1 \|_{L^{p}(T_1)}^{p}
        +
        \| u'_2 \|_{L^{p}(T_2)}^{p}
        \\
        &\leq 
        \| u_1 \|_{L^{p}(T_1)}^{p}
        +
        \left( 
            \| u_2 \|_{L^{p}(T_2)}
            +
            \left(\frac{|T_2|}{|F|}\right)^{\frac 1 p}
            \| c \|_{L^{p}(F)}
        \right)^{p}
        \\
        &\leq 
        \| u_1 \|_{L^{p}(T_1)}^{p}
        +
        2^{p-1}
        \| u_2 \|_{L^{p}(T_2)}^{p}
        +
        2^{p-1}
        \left(\frac{|T_2|}{|F|}\right)
        \| c \|_{L^{p}(F)}^{p}
        \\
        % &\leq 
        % \| u_1 \|_{L^{p}(T_1)}^{p}
        % +
        % 2^{p-1}
        % \| u_2 \|_{L^{p}(T_2)}^{p}
        % +
        % 4^{p-1}
        % \left( 
        %     2 \max\left( 
        %         \frac{|T_2|}{|T_1|} 
        %         ,
        %         1
        %     \right)
        %     \| u_1 \cup u_2 \|_{L^{p}(U)}^{p}
        %     +
        %     (p-1) \cdot 
        %     \max\left( 
        %         \diam(T_1)^p \frac{|T_2|}{|T_1|} 
        %         ,
        %         \diam(T_2)^p 
        %         %
        %     \right) 
        %     \| \nabla u_1 \cup \nabla u_2 \|_{\bfL^{p}(U)}^{p}
        % \right)
        % \\
        &\leq 
        \| u_1 \|_{L^{p}(T_1)}^{p}
        +
        2^{p-1}
        \| u_2 \|_{L^{p}(T_2)}^{p}
        \\&\quad\quad\quad\quad
        +
        4^{p-1}
        2
        \frac{|T_2|}{|T_1|} 
        \| u_1 \|_{L^{p}(T_1)}^{p}
        +
        4^{p-1}
        (p-1) \cdot \diam(T_1)^{p} 
        \frac{|T_2|}{|T_1|} 
        \| \nabla u_1 \|_{\bfL^{p}(T_1)}^{p}
        \\&\quad\quad\quad\quad
        +
        4^{p-1}
        2
        \| u_2 \|_{L^{p}(T_2)}^{p}
        +
        4^{p-1}
        (p-1) \cdot \diam(T_2)^{p} 
        \| \nabla u_2 \|_{\bfL^{p}(T_2)}^{p}
    \end{align*}
    %}
    We can now apply the Poincar\'e--Friedrichs inequality over each simplex. Hence,
    \begin{align*}
        \| u \|_{L^{p}(U)}^{p}
        &\leq 
        C_1 D^{p} \| \nabla u_1 \|_{L^{p}(T_1)}^{p}
        +
        2^{p-1}
        C_2 D^{p} \| \nabla u_2 \|_{L^{p}(T_2)}^{p}
        \\&\quad\quad\quad\quad
        +
        4^{p-1}
        2
        Q
        C_1 D^{p} \| \nabla u_1 \|_{L^{p}(T_1)}^{p}
        +
        4^{p-1}
        (p-1) \cdot D^{p} 
        Q
        \| \nabla u_1 \|_{\bfL^{p}(T_1)}^{p}
        \\&\quad\quad\quad\quad
        +
        4^{p-1}
        2
        C_2 D^{p} \| \nabla u_2 \|_{L^{p}(T_2)}^{p}
        +
        4^{p-1}
        (p-1) \cdot D^{p} 
        \| \nabla u_2 \|_{\bfL^{p}(T_2)}^{p}
        \\
        &\leq 
        \left( 
            C_1 
            +
            4^{p-1}
            2
            Q
            C_1 
            +
            4^{p-1}
            (p-1)
            Q
        \right) 
        D^{p}
        \| \nabla u_1 \|_{L^{p}(T_1)}^{p}
        \\&\quad\quad\quad\quad
        +
        \left( 
        2^{p-1}
        C_2 
        +
        4^{p-1}
        2
        C_2 
        +
        4^{p-1}
        (p-1)
        \right) 
        D^{p}
        \| \nabla u_2 \|_{L^{p}(T_2)}^{p}
        \\
        &\leq 
        \left( 
            2^{p-1}
            C  
            +
            4^{p-1}
            2
            Q'
            C_1 
            +
            4^{p-1}
            (p-1)
            Q'
        \right) 
        D^{p} 
        \| \nabla u \|_{L^{p}(U)}^{p}
        .
    \end{align*}
    
    \color{red} Possibly the estimate can be tweaked to obtain nicer constants.
\end{proof}



%  
 % \color{red}We want to summarize the Lp norms so that the Poincare inequality over $U_F$ is applied only once. In the special cases $p=1$ and $p=\infty$, this works as above. Do we have a formula for general $p$ that contains these as limit cases? \color{black}
 % In the special case $p < \infty$, we use the finite-dimensional H\"older inequality to find 
 % \begin{align*}
 %    \| w_{m} \|_{L^{p}(T_m)}^{p}
 %    &\leq 
 %    3^{p-1}
 %    \| u_{F_m} \|_{L^{p}(T_m)}^{p}
 %    +
 %    3^{p-1}
 %    \frac{ \vol(T_m) }{ \vol(T_{j(m)}) }
 %    \| u_{F_{m}} \|_{L^{p}(T_{j(m)})}^{p}
 %    +
 %    3^{p-1}
 %    \frac{ \vol(T_m) }{ \vol(T_{j(m)}) }
 %    \| w_{m-1}\|_{L^{p}(T_{j(m)})}^{p}
 %    \\&\leq 
 %    3^{p-1}
 %    \max\left( 1, \frac{ \vol(T_m) }{ \vol(T_{j(m)}) } \right)
 %    \| u_{F_m} \|_{L^{p}(U_{F_m})}^{p}
 %    +
 %    3^{p-1}
 %    \frac{ \vol(T_m) }{ \vol(T_{j(m)}) }
 %    \| w_{m-1}\|_{L^{p}(T_{j(m)})}^{p}
 %    ,
 % \end{align*}
 % and therefore
 % \begin{align*}
 %    \| w_{m} \|_{L^{p}(T_m)}
 %    &\leq 
 %    3^{\frac{p-1}{p}}
 %    \max\left( 1, \frac{ \vol(T_m) }{ \vol(T_{j(m)}) } \right)^{\frac 1 p}
 %    \| u_{F_m} \|_{L^{p}(U_{F_m})} 
 %    +
 %    3^{\frac{p-1}{p}}
 %    \frac{ \vol(T_m)^{\frac 1 p} }{ \vol(T_{j(m)})^{\frac 1 p} }
 %    \| w_{m-1}\|_{L^{p}(T_{j(m)})} 
 %    .
 % \end{align*}
 % \color{blue}
 % In the special case $p < \infty$, we use the finite-dimensional H\"older inequality to find 
 % \begin{align*}
 %    \| w_{m} \|_{L^{p}(T_m)}^{p}
 %    &\leq 
 %    \left( 
 %        1 + 2 \frac{ \vol(T_m)^{\frac{1}{p-1}} }{ \vol(T_{j(m)})^{\frac{1}{p-1}} }
 %    \right)^{ p-1 }
 %    \left( 
 %        \| u_{F_m} \|_{L^{p}(T_m)}^{p}
 %        +
 %        \| u_{F_{m}} \|_{L^{p}(T_{j(m)})}^{p}
 %        +
 %        \| w_{m-1}\|_{L^{p}(T_{j(m)})}^{p}
 %    \right)
 %    ,
 % \end{align*}
 % and therefore
 % \begin{align*}
 %    \| w_{m} \|_{L^{p}(T_m)}
 %    &\leq 
 %    \left( 
 %        1 + 2 \frac{ \vol(T_m)^{\frac{1}{p-1}} }{ \vol(T_{j(m)})^{\frac{1}{p-1}} }
 %    \right)^{ \frac{p-1}{p} }
 %    \left( 
 %        \| u_{F_m} \|_{L^{p}(T_m)} 
 %        +
 %        \| u_{F_{m}} \|_{L^{p}(T_{j(m)})} 
 %        +
 %        \| w_{m-1}\|_{L^{p}(T_{j(m)})} 
 %    \right)
 %    .
 % \end{align*}
 % \color{black}
 
 
 



we have any of the following two conditions:
\begin{itemize}
 \item 
 there exists $\epsilon > 0$ 
 and a homeomorphism 
 \begin{align*}
    \phi : \Ball_\epsilon(x) \cap \underlying{\calT}  
    \rightarrow 
    \left\{ x \in \bbR^{n} \suchthat |x| \leq 1 \right\}
 \end{align*}
 such that $\phi(x) = 0$,
 \item 
 there exists $\epsilon > 0$ and a homeomorphism 
 \begin{align*}
    \phi : \Ball_\epsilon(x) \cap \underlying{\calT} 
    \rightarrow 
    \left\{ x \in \bbR^{n} \suchthat |x| \leq 1, x_n \leq 0 \right\}
 \end{align*}
 such that $\phi(x) = 0$.
\end{itemize}
This characterizes the union of the simplicial complex $\calT$ as a topological manifold with boundary. 















\subsection{On the topology of shellable triangulations}
    
A shellable simplicial complex that triangulates a manifold has a very restricted topology. 
The Mayer-Vietoris theorem implies restrictions on the homology groups of simplicial complexes iteratively constructed via shelling. 

    Suppose furthermore that $T$ is an $n$-simplex, $T \notin \calT$, 
    whose boundary complex intersects with $\partial\calT$ at a non-empty simplicial complex $\calS$ of dimension $n-1$. 
    In particular, $\calS$ is either the whole boundary complex of $T$ or a patch around a proper subsimplex of $T$. 
    We write $\calT'$ for the  simplicial complex that contains $\calT \cup T$.


Suppose that $\calT$ is a strongly connected $n$-dimensional simplicial complex that triangulates a manifold, and suppose that its boundary complex $\partial\calT$ is either empty or strongly connected.  
Suppose furthermore that $T$ is an $n$-simplex, $T \notin \calT$, whose boundary complex intersects with $\partial\calT$ at a non-empty simplicial complex $\calS$ of dimension $n-1$. In particular, $\calS$ is either the whole boundary complex of $T$ or a patch around a proper subsimplex of $T$. We write $\calT'$ be the  simplicial complex that contains $\calT \cup T$.
We exclude the case $n \leq 1$ since otherwise the discussion is trivial.

According to the simplicial Mayer-Vietoris theorem, there exists an exact sequence 
\begin{align*}
    \begin{CD}
        \dots \longrightarrow H_{k}(\calS) \longrightarrow H_{k}(\calT \oplus T) \longrightarrow H_{k}(\calT') \longrightarrow H_{k-1}(\calS) \longrightarrow\dots 
    \end{CD}
\end{align*}
We notice that $\calS$, $\calT$, $T$ and $\calT'$ are each connected, and hence their zeroth homology groups are isomorphic to ${G}$. 
By the nature of the mappings in the Mayer-Vietoris sequence, 
\color{red}which uses some specific knowledge about the arrows\color{black}, 
we see that 
\begin{align*}
    \begin{CD}
        0 \longrightarrow H_{0}(\calS) \longrightarrow H_{0}(\calT) \oplus H_{0}(T) \longrightarrow H_{0}(\calT') \longrightarrow 0 
    \end{CD}
\end{align*}
is an exact sequence. 
Moreover, $H_{k}(\calS) = H_{k}(T) = 0$ for $0 < k < n-1$. Hence, for $1 < k < n-1$,
the Mayer-Vietoris sequence includes the exact sequence 
\begin{align*}
    \begin{CD}
        0 \longrightarrow H_{k}(\calT \oplus T) \longrightarrow H_{k}(\calT') \longrightarrow 0 
    \end{CD}
\end{align*}
and we conclude that $H_{k}(\calT) = 0$ for $0 < k < n-1$.
Finally, the Mayer-Vietoris sequence includes the exact sequence 
\begin{align*}
    \begin{CD}
        0 \longrightarrow H_{n}(\calT \oplus T) \longrightarrow H_{n}(\calT') \longrightarrow H_{n-1}(\calS) \longrightarrow H_{n-1}(\calT \oplus T) \longrightarrow H_{n-1}(\calT') \longrightarrow 0 
    \end{CD}
\end{align*}

% Here, if $n=1$, it already follows that $H_{n}(\calT \oplus T) \simeq H_{n}(\calT')$.
% % We know that  $H_{0}(\calT') \simeq H_{0}(\calT) \simeq H_{0}(T) \simeq H_{0}(\calS) \simeq \bbZ$ because the respective simplicial complexes are path-connected. 
% Hence $H_{1}(\calT) \oplus H_{1}(T) \simeq H_{1}(\calT')$. 
% one easily sees that $H_{0}(\calT') \simeq \bbZ$ and $H_{1}(\calT) \oplus H_{1}(T) \simeq H_{1}(\calT')$. 
Since $\calT$ and $T$ are triangulations of manifolds with non-empty boundary, their $n$-th simplicial homology groups vanish. 
then $H_{n}(\calT') \simeq H_{n}(\calT \oplus T)$ are trivial and $H_{n-1}(\calT') \simeq H_{n-1}(\calT \oplus T)$ follows.
Consider now the case that $\calS$ is isomorphic to a sphere.
Since the boundary complex of $\calT$ is strongly connected and non-branching, it must coincide with $\calS$. Hence $\calT'$ triangulates a manifold without boundary, and thus $H_{n}(\calT') \simeq \bbZ$. 
\color{red}By some hand-waving argument, $H_{n}(\calT') \longrightarrow H_{n-1}(\calS)$ is an isomorphism.\color{black}
Hence $H_{n-1}(\calT) \rightarrow H_{n-1}(\calT')$ is an isomorphism.

In the case that $\calS$ is isomorphic to a disk, we have $H_{n-1}(\calS) = 0$,
and then $H_{n}(\calT') \simeq H_{n}(\calT \oplus T) = 0$ are trivial 
and $H_{n-1}(\calT') \simeq H_{n-1}(\calT \oplus T)$ follows.
Consider now the case that $\calS$ is isomorphic to a sphere.
Since the boundary complex of $\calT$ is strongly connected and non-branching, it must coincide with $\calS$. Hence $\calT'$ triangulates a manifold without boundary, and thus $H_{n}(\calT') \simeq {G}$. 
\color{red}By some hand-waving argument, $H_{n}(\calT') \longrightarrow H_{n-1}(\calS)$ is an isomorphism.\color{black}
Hence $H_{n-1}(\calT) \rightarrow H_{n-1}(\calT')$ is an isomorphism.


We conclude that the simplicial homology groups of a shellable $n$-dimensional simplicial complex $\calT$ that triangulates a manifold satisfy 
\begin{align*}
    H_0(\calT) \simeq \bbZ, 
    \quad 
    H_1(\calT) \simeq \dots \simeq H_{n-1}(\calT) \simeq 0.
\end{align*}
Moreover, $H_n(\calT) \simeq {G}$ if $\calT$ has no boundary and $H_n(\calT) \simeq 0$ otherwise. 
Moreover, $H_n(\calT) \simeq \bbZ$ if $\calT$ has no boundary and $H_n(\calT) \simeq 0$ otherwise. 

% then $H_{n}(\calT') \simeq H_{n}(\calT \oplus T)$ and $H_{n-1}(\calT') \simeq H_{n-1}(\calT \oplus T)$ follow immediately.
% Moreover, Thus $H_{n}(\calT') = H_{n}(\calT) \oplus H_{n}(\calS)$.
% We conclude that any shellable simplicial complex has the homology groups of a disk or a sphere. 
% NB: the zeroth homology is always free, and the n-th homology group, being the kernel of a mapping free abelian group, is free as well.
% \begin{align*}
%     \begin{CD}
%         0 \longrightarrow H_{1}(\calT) \oplus H_{1}(T) \longrightarrow H_{1}(\calT') \longrightarrow H_{0}(\calS) \longrightarrow H_{0}(\calT) \oplus H_{0}(T) \longrightarrow H_{0}(\calT') \longrightarrow 0 
%     \end{CD}
% \end{align*}

\begin{remark}    
    % % We know that  $H_{0}(\calT') \simeq H_{0}(\calT) \simeq H_{0}(T) \simeq H_{0}(\calS) \simeq {G}$ because the respective simplicial complexes are path-connected. 
    % Hence $H_{1}(\calT) \oplus H_{1}(T) \simeq H_{1}(\calT')$. 
    % one easily sees that $H_{0}(\calT') \simeq {G}$ and $H_{1}(\calT) \oplus H_{1}(T) \simeq H_{1}(\calT')$. 
     
    % then $H_{n}(\calT') \simeq H_{n}(\calT \oplus T)$ and $H_{n-1}(\calT') \simeq H_{n-1}(\calT \oplus T)$ follow immediately.
    % Moreover, Thus $H_{n}(\calT') = H_{n}(\calT) \oplus H_{n}(\calS)$.
    % We conclude that any shellable simplicial complex has the homology groups of a disk or a sphere. 
    % NB: the zeroth homology is always free, and the n-th homology group, being the kernel of a mapping free abelian group, is free as well.
    % \begin{align*}
    %     \begin{CD}
    %         0 \longrightarrow H_{1}(\calT) \oplus H_{1}(T) \longrightarrow H_{1}(\calT') \longrightarrow H_{0}(\calS) \longrightarrow H_{0}(\calT) \oplus H_{0}(T) \longrightarrow H_{0}(\calT') \longrightarrow 0 
    %     \end{CD}
    % \end{align*}
\end{remark}













\newpage\color{red}
 We want 
 \begin{align*}
    \| w_{m} \|_{L^{p}(T_m)}^{p}
    \leq
    \left( 
        1 
        + 
        2
        \frac{ \vol(T_m)^{\frac q p} }{ \vol(T_{j(m)})^{\frac q p} }
    \right)^{p/q}
    \left( 
        \| u_{F_m} \|_{L^{p}(T_m)}^{p}
        +
        \| u_{F_{m}} \|_{L^{p}(T_{j(m)})}^{p}
        +
        \| w_{m-1}\|_{L^{p}(T_{j(m)})}^{p}
    \right)
    .
 \end{align*}
 In the special case $p=1$ and $p=\infty$,
 \begin{align*}
    \| w_{m} \|_{L^{1}(T_m)}
    &\leq 
    \max\left(
        1, \frac{ \vol(T_m) }{ \vol(T_{j(m)}) } 
    \right)
    \| u_{F_m} \|_{L^{1}(U_{F_m})}
    +
    \max\left(
        1, \frac{ \vol(T_m) }{ \vol(T_{j(m)}) } 
    \right)
    \| w_{m-1}\|_{L^{1}(T_{j(m)})}
    .
    \\
    \| w_{m} \|_{L^{2}(T_m)}^{2}
    &\leq
    \left( 
        1 
        + 
        2
        \frac{ \vol(T_m) }{ \vol(T_{j(m)}) }
    \right)
    \left( 
        \| u_{F_m} \|_{L^{2}(T_m)}^{2}
        +
        \| u_{F_{m}} \|_{L^{2}(T_{j(m)})}^{2}
        +
        \| w_{m-1}\|_{L^{2}(T_{j(m)})}^{2}
    \right)
    .
    \\
    \| w_{m} \|_{L^{\infty}(T_m)}
    &\leq 
    3
    \max\left( 
        \| u_{F_m} \|_{L^{\infty}(U_{F_m})}
        ,
        \| w_{m-1}\|_{L^{\infty}(T_{j(m)})}
    \right)
    .
 \end{align*}
 \color{orange}
 \begin{align*}
    \| w_{m} \|_{L^{p}(T_m)}^{p}
    \leq
    \left( 
        2 
        + 
        \frac{ \vol(T_m)^{\frac q p} }{ \vol(T_{j(m)})^{\frac q p} }
    \right)^{p/q}
    \left( 
        \| u_{F_m} \|_{L^{p}(T_m)}^{p}
        +
        \| u_{F_{m}} \|_{L^{p}(T_{j(m)})}^{p}
        +
        \frac{ \vol(T_m) }{ \vol(T_{j(m)}) }
        \| w_{m-1}\|_{L^{p}(T_{j(m)})}^{p}
    \right)
    .
 \end{align*}
 In the special case $p=1$ and $p=\infty$, 
 \begin{align*}
    \| w_{m} \|_{L^{1}(T_m)}
    &\leq 
    \max\left(
        1, \frac{ \vol(T_m) }{ \vol(T_{j(m)}) } 
    \right)
    \| u_{F_m} \|_{L^{1}(U_{F_m})}
    +
    \max\left(
        1, \frac{ \vol(T_m) }{ \vol(T_{j(m)}) } 
    \right)
    \frac{ \vol(T_m) }{ \vol(T_{j(m)}) }
    \| w_{m-1}\|_{L^{1}(T_{j(m)})}
    .
    \\
    \| w_{m} \|_{L^{2}(T_m)}^{2}
    &\leq
    \left( 
        2 
        + 
        \frac{ \vol(T_m) }{ \vol(T_{j(m)}) }
    \right) 
    \left( 
        \| u_{F_m} \|_{L^{2}(T_m)}^{2}
        +
        \| u_{F_{m}} \|_{L^{2}(T_{j(m)})}^{2}
        +
        \frac{ \vol(T_m) }{ \vol(T_{j(m)}) }
        \| w_{m-1}\|_{L^{2}(T_{j(m)})}^{2}
    \right)
    \\
    \| w_{m} \|_{L^{\infty}(T_m)}
    &\leq 
    3
    \max\left( 
        \| u_{F_m} \|_{L^{\infty}(U_{F_m})}
        ,
        \| w_{m-1}\|_{L^{\infty}(T_{j(m)})}
    \right)
    .
 \end{align*}
 \color{JungleGreen}
 \begin{align*}
    \| w_{m} \|_{L^{p}(T_m)}^{p}
    &\leq
    \left( 
        3
    \right)^{p/q}
    \left( 
        \| u_{F_m} \|_{L^{p}(T_m)}^{p}
        +
        \frac{ \vol(T_m) }{ \vol(T_{j(m)}) }
        \| u_{F_{m}} \|_{L^{p}(T_{j(m)})}^{p}
        +
        \frac{ \vol(T_m) }{ \vol(T_{j(m)}) }
        \| w_{m-1}\|_{L^{p}(T_{j(m)})}^{p}
    \right)
    \\&
    \leq
    \left( 
        3
    \right)^{p/q}
    \left( 
        \max\left( 1, \frac{ \vol(T_m) }{ \vol(T_{j(m)}) } \right)
        \| u_{F_{m}} \|_{L^{p}(T_m \cup T_{j(m)})}^{p}
        +
        \frac{ \vol(T_m) }{ \vol(T_{j(m)}) }
        \| w_{m-1}\|_{L^{p}(T_{j(m)})}^{p}
    \right)
    .
 \end{align*}
 In the special case $p=1$ and $p=\infty$, ????
 \begin{align*}
    \| w_{m} \|_{L^{1}(T_m)}
    &\leq 
    \left( 
        \max\left( 1, \frac{ \vol(T_m) }{ \vol(T_{j(m)}) } \right)
        \| u_{F_{m}} \|_{L^{1}(T_m \cup T_{j(m)})}^{1}
        +
        \frac{ \vol(T_m) }{ \vol(T_{j(m)}) }
        \| w_{m-1}\|_{L^{1}(T_{j(m)})}^{1}
    \right)
    .
    \\
    \| w_{m} \|_{L^{2}(T_m)}^{2}
    &\leq
    3
    \left( 
        \max\left( 1, \frac{ \vol(T_m) }{ \vol(T_{j(m)}) } \right)
        \| u_{F_{m}} \|_{L^{2}(T_m \cup T_{j(m)})}^{2}
        +
        \frac{ \vol(T_m) }{ \vol(T_{j(m)}) }
        \| w_{m-1}\|_{L^{2}(T_{j(m)})}^{2}
    \right)
    \\
    \| w_{m} \|_{L^{\infty}(T_m)}
    &\leq 
    3
    \max\left( 
        \| u_{F_m} \|_{L^{\infty}(U_{F_m})}
        ,
        \| w_{m-1}\|_{L^{\infty}(T_{j(m)})}
    \right)
    .
 \end{align*}
 \color{MidnightBlue}
 \begin{align*}
    \| w_{m} \|_{L^{p}(T_m)}^{p}
    &\leq
    \left( 
        2 
        + 
        \frac{ \vol(T_m)^{\frac q p} }{ \vol(T_{j(m)})^{\frac q p} }
    \right)^{p/q}
    \left( 
        \| u_{F_m} \|_{L^{p}(T_m)}^{p}
        +
        \frac{ \vol(T_m) }{ \vol(T_{j(m)}) }
        \| u_{F_{m}} \|_{L^{p}(T_{j(m)})}^{p}
        +
        \| w_{m-1}\|_{L^{p}(T_{j(m)})}^{p}
    \right)
    \\&
    \leq
    \left( 
        2 
        + 
        \frac{ \vol(T_m)^{\frac q p} }{ \vol(T_{j(m)})^{\frac q p} }
    \right)^{p/q}
    \left( 
        \max\left( 1, \frac{ \vol(T_m) }{ \vol(T_{j(m)}) } \right)
        \| u_{F_{m}} \|_{L^{p}(T_m \cup T_{j(m)})}^{p}
        +
        \| w_{m-1}\|_{L^{p}(T_{j(m)})}^{p}
    \right)
    .
 \end{align*}
 In the special case $p=1$ and $p=\infty$, ????
 \begin{align*}
    \| w_{m} \|_{L^{1}(T_m)}
    &\leq 
    \max\left( 1, \frac{ \vol(T_m) }{ \vol(T_{j(m)}) } \right)
    \left( 
        \max\left( 1, \frac{ \vol(T_m) }{ \vol(T_{j(m)}) } \right)
        \| u_{F_{m}} \|_{L^{1}(T_m \cup T_{j(m)})} 
        +
        \| w_{m-1}\|_{L^{1}(T_{j(m)})} 
    \right)
    .
    \\
    \| w_{m} \|_{L^{2}(T_m)}^{2}
    &\leq 
    \left( 
        2 
        + 
        \frac{ \vol(T_m) }{ \vol(T_{j(m)}) }
    \right) 
    \left( 
        \max\left( 1, \frac{ \vol(T_m) }{ \vol(T_{j(m)}) } \right)
        \| u_{F_{m}} \|_{L^{2}(T_m \cup T_{j(m)})}^{2}
        +
        \| w_{m-1}\|_{L^{2}(T_{j(m)})}^{2}
    \right)
    \\
    \| w_{m} \|_{L^{\infty}(T_m)}
    &\leq 
    3
    \max\left( 
        \| u_{F_m} \|_{L^{\infty}(U_{F_m})}
        ,
        \| w_{m-1}\|_{L^{\infty}(T_{j(m)})}
    \right)
    .
 \end{align*}
 
 
 
 
 
 
 
 
 
 We introduce the operator $B^{k}$, which can be written in several alternative forms
\begin{align*}
    B^{k} u(x) 
    &= 
    - |\Omega|^{-1}
    \int_{\Omega} \,(x-a) \lrcorner \int_1^\infty t^{{k}-1}\,u\left( a+t(x-a) \right) \,dt\,da
    \\&= 
    |\Omega|^{-1}
    \int_{\Omega} \,(x-a) \lrcorner \int_0^1 t^{-{k}-1}\,u\left( a+(x-a)/s \right) \,ds\,da
    .
\end{align*}
If $u \in C^{\infty}_{c}(\bbR^{n},\Alt^{k})$, 
then $B^{k} u$ is smooth with support in $\bbR^{n}\setminus\overline\Omega$ and $B^{k} u(x) = 0$ when $x \notin \Omega$. 
Since $\Omega$ is convex, $u \in C^{\infty}_{c}(\Omega,\Alt^{k})$ implies $\supp B^{k} u \subset\Omega$, 
Since $\Omega$ is bounded, $\supp u \subseteq \overline\Omega$ implies $\supp B^{k} u \subset\overline\Omega$. 
% The fact that $B^{k}$ indeed maps 
% $C^{\infty}_{c}(\bbR^{n},\Alt^{k})$ to $C^{\infty}_{c}(\bbR^{n},\Alt^{{k}-1})$ will be a the following theorem.

\begin{align*}
    \int_{\Omega}
    \left| B^{k} u(x) \right|^{p}
    dx
    &
    = 
    |\Omega|^{-1}
    \int_{\Omega}
    \int_{\Omega} 
    \int_0^1
    s^{(-{k}-1)p}\, |x-a|^{p} \left| u\left( a+(x-a)/s \right) \right|^{p} \,ds\,da
    dx
    \\&
    = 
    |\Omega|^{-1}
    \int_{\Omega}
    \int_{\Omega-a} 
    \int_{|z|/D}^{1}
    s^{(-{k}-1)p}\, |z|^{p} \left| u\left( a+z/s \right) \right|^{p} \,ds\,da
    dz
    \\&
    = 
    |\Omega|^{-1}
    \int_{\Omega}
    \int_{2\Omega} 
    \int_{|z|/D}^{1}
    s^{(-{k}-1)p}\, |z|^{p} \left| u\left( a+z/s \right) \right|^{p} \,ds\,da
    dz
    \\&
    = 
    |\Omega|^{-1}
    \int_{\Omega + z/s}
    \int_{2\Omega} 
    \int_{|z|/D}^{1} % y = a + z/s
    s^{(-{k}-1)p}\, |z|^{p} \left| u\left( y \right) \right|^{p} \,ds\,dy
    dz
    \\&
    = 
    |\Omega|^{-1}
    \int_{2\Omega}
    \int_{2\Omega} 
    \int_{|z|/D}^{1} % y = a + z/s
    s^{(-{k}-1)p}\, |z|^{p} \left| u\left( y \right) \right|^{p} \,ds\,dy
    dz
    .
\end{align*}



\begin{align*}
    G_\ell(x,y) 
    &\leq 
    \int_{0}^{\frac{D}{|x-y|}} \tau^{n-\ell} (\tau+1)^{\ell-1} \,d\tau
    \\
    &\leq 
    \frac{|x-y|}{D}
    \int_{0}^{\frac{D}{|x-y|}} \frac{D}{|x-y|} \tau^{n-\ell} (\tau+1)^{\ell-1} \,d\tau
    .
\end{align*}
\begin{align*}
    G_\ell(x,y)^{p} 
    &\leq 
    \frac{|x-y|}{D}
    \int_{0}^{\frac{D}{|x-y|}} \frac{D^{p}}{|x-y|^{p}} \tau^{pn-p\ell} (\tau+1)^{p\ell-p} \,d\tau
    \\&\leq 
    \frac{|x-y|}{D}
    \frac{D^{p}}{|x-y|^{p}} \int_{0}^{\frac{D}{|x-y|}} (\tau+1)^{pn-p} \,d\tau
    \\&\leq 
    \frac{D^{p-1}}{|x-y|^{p-1}} \left( \left( \frac{D}{|x-y|} + 1 \right)^{pn-p+1} - 1 \right)
    \\&\leq 
    \frac{D^{p-1}}{|x-y|^{p-1}} \left( 2^{pn-p+1} \left( \frac{D}{|x-y|} \right)^{pn-p+1} - 1 \right)
    \\&\leq 
    2^{pn-p+1} \frac{D^{pn}}{|x-y|^{pn}} - \frac{D^{p-1}}{|x-y|^{p-1}}
    .
\end{align}





Moreover,  
\begin{align*}
    \int_{\Omega} \left| B_{\ell} u(x) \right|^{p} dx
    &=
    |\Omega|^{-1} 
    \int_{\Omega} 
    \int_{\Omega} 
    \left( 2^{pn-p+1} \frac{D^{pn}}{|x-y|^{pn}} - \frac{D^{p-1}}{|x-y|^{p-1}} \right)
    \left| (x-y)\lrcorner u(y) \right|^{p} 
    dy dx
    \\&=
    |\Omega|^{-1} 
    \int_{\Omega} 
    \int_{\Omega} \left( 2^{pn-p+1} D^{pn} |x-y|^{-pn+p} - D^{p-1} |x-y| \right) dx
    \left| u(y) \right|^{p} 
    dy
    .
\end{align*}






We use the radial integrals
\begin{align*}
    \int_{B_D(0)} |z|^{m} dz
    =
    \vol_{n-1}(S_1) \int_0^D r^{m} r^{n-1} dr
    =
    \vol_{n-1}(S_1) \int_0^D r^{m+n-1} dr
    =
    \vol_{n-1}(S_1) \frac{D^{m+n}}{m+n}
    .
\end{align*}







In order to see other properties of the operators, we apply a different change of variables. 
Let us write this in detail for the operator $R_{\ell}$. 
We substitute $y$ by the new variable $a=y+t(x-y)$.
Whence $da = (1-t)^n dy$ and $t(x-y) = a-y$ and $x-y = (a-y)/t$ and $x - (a-x)/t = y$
\begin{align*}
    \Bogov_{\ell} u(x) 
    &= 
    \int \int_{1}^\infty (t-1)^{n-\ell}t^{\ell-1} 
    \chi_{\Omega}\left(y+t(y-x)\right) 
    (x-y)\lrcorner u(y) \,dt\,dy 
    \\&= 
    \int \int_{1}^\infty (t-1)^{n-\ell}t^{\ell-1} 
    \chi_{\Omega}\left(y+(x-y)-(x-y)+t(y-x)\right) 
    (x-y)\lrcorner u(y) \,dt\,dy 
    \\&= 
    \int \int_{1}^\infty (t-1)^{n-\ell}t^{\ell-1} 
    \chi_{\Omega}\left(x+(t+1)(y-x)\right) 
    (x-y)\lrcorner u(y) \,dt\,dy 
    \\
    &= 
    \int \int_{1}^\infty (t-1)^{n-\ell}t^{\ell-1} (t+1)^{n}
    \chi_{\Omega}\left(a\right) 
    \frac{ x-a }{t+1}\lrcorner u( x + (a-x)/(t+1) ) \,dt\,dy 
\end{align*}
Then we substitute $1-s = 1/(t+1)$, that is, $s = t/(t+1)$:
Whence $t = \left( 1/s-1 \right)^{-1}$.
\begin{align*}
    \Bogov_{\ell} u(x) 
    &= 
    \int \int_{1}^\infty (t-1)^{n-\ell} \left( 1/s-1 \right)^{\ell-1} (t+1)^{n}
    \chi_{\Omega}\left(a\right) 
    \frac{ x-a }{t+1}\lrcorner u( x + (a-x)/(t+1) ) \,dt\,dy 
    \\
    &= 
    .
\end{align*}


Then we substitute $s = (t-1)/t$:
\begin{align*}
    &= 
    \int \int_{1}^\infty (t-1)^{n-\ell}t^{-n+\ell-1} 
    \chi_{\Omega}\left(a\right)
    \frac{(a-y)}{t} \lrcorner u(x+(a-x)/t) \,dt\,da 
    \\
    &= 
    \int \int_{1}^\infty (t-1)^{n-\ell}t^{\ell-1-n-1} \chi_{\Omega}\left(a\right)\, (x-a)\lrcorner u(x+(a-x)/t) \,dt\,da
    .
\end{align*}
Then we substitute $s = (t-1)/t$:
\begin{align*}
    \Bogov_{\ell} u(x) 
    &= 
    \int \int_{1}^\infty s^{n-\ell} (s-1)^{2} \chi_{\Omega}\left(a\right)\, (x-a)\lrcorner u( x+(s-1)(a-x) ) \frac{-1}{(s-1)^2}\,ds\,da
    \\
    &= 
    - \int \int_{1}^\infty s^{n-\ell} \chi_{\Omega}\left(a\right)\, (x-a)\lrcorner u( s(a-x) - a ) \,ds\,da
    \\
    &=
    - \int \chi_{\Omega}(a) \,(x-a)\lrcorner \int_{1}^\infty t^{\ell-1} u\left(a+t(x-a)\right)\,dt\,da\;
    .
\end{align*}
We copy a statement by McIntosh and Costabel:
\begin{align*}
    \Bogov_{\ell} u(x) 
    = 
    - \int \chi_{\Omega}(a) \,(x-a)\lrcorner \int_{1}^\infty t^{\ell-1} u\left(a+t(x-a)\right)\,dt\,da\;
    .
\end{align*}
$a = y + t(x-y)$
$s = \frac{t}{t-1}$
From that representation we can tell that whenever $u \in C^{\infty}(\bbR^{n})$ has support contained in $\overline\Omega$, we immediately have that $\Bogov_{\ell} u(x)$ is smooth with support contained in $\overline\Omega$ as well. 











From this form of $\Bogov_{\ell}$, because of the unbounded interval of integration in $t$, one cannot immediately conclude that $\Bogov_{\ell}$ maps ${C^\infty}$ functions to ${C^\infty}$ functions.
But if $u\in{C^\infty_c}(\bbR^n,\Alt^\ell)$, one sees that $\Bogov_{\ell} u$ is ${C^\infty}$ on 
$\bbR^n\setminus\supp\chi_{\Omega}$, and that 
$\Bogov_{\ell} u(x)=0$ unless $x$ lies in the starlike hull of $\supp u$ with respect to $B$. 
Thus if $\Omega$ is open and starlike with respect to $B$, then   
$u\in{C^\infty_c}(\Omega,\Alt^\ell)$ implies $\supp \Bogov_{\ell} u \subset\Omega$, and, if $\Omega$ is bounded, then
$u\in{C^\infty}_{\overline\Omega}(\bbR^n,\Alt^\ell)$ implies $\supp \Bogov_{\ell} u \subset\overline\Omega$. 
The fact that $\Bogov_{\ell}$ indeed maps 
${C^\infty_c}(\bbR^n,\Alt^\ell)$ to ${C^\infty_c}(\bbR^n,\Alt^{\ell-1})$ will be a consequence of Theorem~\ref{T:pseudo} below.




\subsection{Homotopy relations}

Cartan's formula for the Lie derivative of a differential form with respect to a vector field can be written as
$$
  \frac{d}{dt}F^*_t u = F^*_t \left(d(X_t\lrcorner u) + X_t\lrcorner du\right)\;,
$$
where $F^*_t$ denotes the pull-back by the flow $F_t$ associated with the vector field $X_t$. Here we consider the special case of the dilation flow with center $a$
$$
  F_t(x) = a+t(x-a) \quad\text{ with vector field } X_t=x-a\;,
$$
which gives a pull-back of
$$
  F^*_t u(x) = t^\ell \,u\left(a+t(x-a)\right) \quad \text{for an $\ell$-form } u \;.
$$
This leads to the formula
\begin{equation}
\label{eq:Cartan}
 \frac{d}{dt}(t^\ell u\left(a+t(x-a)\right) =
    d\Bigl(t^{\ell-1}(x-a)\lrcorner u\left(a+t(x-a)\right)\Bigr) +
    t^{\ell}(x-a)\lrcorner du\left(a+t(x-a)\right)
\end{equation}
which can also be verified elementarily from the formulas we gave in Section~\ref{S:notation}.

Integrating \eqref{eq:Cartan} from $0$ to $1$ and comparing with \eqref{eq:regPoincare}, 
we find the homotopy relations, valid for all $u\in{C^\infty_c}(\bbR^n,\Alt^\ell)$ 
\begin{equation}
\begin{aligned}
\label{eq:dR+Rd=1}
 dR_{\ell} u + R_{\ell+1} du &= u\; &&(1\le\ell\le n-1)\;;\\
 R_1 du &= u - \left(\chi_{\Omega},u\right)\; &&(\ell=0)\;;\\
 dR_nu &=u &&(\ell=n)\;.
\end{aligned}
\end{equation}
One could be tempted to integrate Cartan's formula from $1$ to $\infty$ and compare with \eqref{eq:Bogo}, thus formally obtaining a similar homotopy relation for $\Bogov_{\ell}$ directly. The result is indeed true except for $\ell=n$, but for a rigorous proof we prefer to use the duality relation \eqref{eq:dualRT} to deduce corresponding anticommutativity relations for $\Bogov_{\ell}$ from the relations \eqref{eq:dR+Rd=1} which are already proved. Here is what one obtains for $u\in{C^\infty_c}(\bbR^n,\Alt^\ell)$:
\begin{equation}
\begin{aligned}
\label{eq:dT+Td=1}
  d\Bogov_{\ell} u + T_{\ell+1} du &= u\; &&(1\le\ell\le n-1)\;;\\
  T_1 du &= u \; &&(\ell=0)\;;\\
 dT_n u &=u - (\int\! u)\star\chi_{\Omega}&&(\ell=n)\;.
\end{aligned}
\end{equation}
Here we consider $\chi_{\Omega}$ as an element of ${C^\infty_c}(\bbR^n,\Alt^0)$, so that for another $0$-form $u$ we have the $L^2$ scalar product $\left(\chi_{\Omega},u\right)=\int\chi_{\Omega}(a)u(a)da$, and $\star\chi_{\Omega}$ is the $n$-form $\chi_{\Omega}(x)dx_1\wedge\dots\wedge dx_n$. 

The formulas for the endpoints $\ell=0$ and $\ell=n$ correspond to the two extended de Rham complexes without boundary conditions and with compact support, see \eqref{eq:edRwobc} and \eqref{eq:edRwcs}. To see this, let us extend the definition of the exterior derivative by writing $\overline d$ for all the mappings of the complex
$$
 0 \to\bbR\to^\iota{C^\infty}(\overline\Omega,\Alt^0) \to^d {C^\infty}(\overline\Omega,\Alt^1)
 \to^d\cdots \to^d {C^\infty}(\overline\Omega,\Alt^n) \to 0
$$ 
and $\underline d$ for all the mappings of the complex
$$
 0 \to {C^\infty}_{\overline\Omega}(\bbR^n,\Alt^0) \to^d {C^\infty}_{\overline\Omega}(\bbR^n,\Alt^1) \to^d\cdots
 \to^d {C^\infty}_{\overline\Omega}(\bbR^n,\Alt^n) \to^{\iota^*} \bbR \to 0
$$
where $\iota$ is the inclusion mapping for constant functions and 
$\iota^*=(\star\iota)'$ denotes the integral $u\mapsto\int\!u$ for $n$-forms.

If we correspondingly extend the definitions of $R_{\ell}$ and $\Bogov_{\ell}$ by
$$
\begin{aligned}
  R_0u &:=\left(\chi_{\Omega},u\right)\,\text{ for $0$-forms }u\,, &
  R_{n+1} &:=0\,,\\
  T_{n+1}u &:= \star(u\chi_{\Omega}) \,\text{ for }u\in\bbR\,,  &
  T_0 &:= 0\,,
\end{aligned}
$$
then we can write the relations \eqref{eq:dR+Rd=1} and \eqref{eq:dT+Td=1} simply as
\begin{equation}\label{eq:R&T&d}
 \overline d\,R_{\ell} u + R_{\ell+1} \,\overline d u  = u
 \quad \text{ and }\quad
 \underline d\,\Bogov_{\ell} u + T_{\ell+1} \,\underline du = u\quad \text{ for all }\;
  0\le\ell\le n.
\end{equation}

 
 
 


We introduce the operator $B^{k}$, which is
\begin{equation}\label{eq:Bogo} % TODO: rewrite 
    B^{k} u(x) = - \int_{\Omega} \,(x-a) \lrcorner \int_1^\infty t^{{k}-1}\,u\left( a+t(x-a) \right) \,dt\,da.
\end{equation}
If $u \in C^{\infty}_{c}(\bbR^{n},\Alt^{k})$, 
then $B^{k} u$ is smooth with support in $\bbR^{n}\setminus\overline\Omega$ and $B^{k} u(x) = 0$ when $x \notin \Omega$. 
Since $\Omega$ is convex, $u \in C^{\infty}_{c}(\Omega,\Alt^{k})$ implies $\supp B^{k} u \subset\Omega$, 
Since $\Omega$ is bounded, $u \in C^{\infty}_{\overline\Omega}(\bbR^{n},\Alt^{k})$ implies $\supp B^{k} u \subset\overline\Omega$. 
% The fact that $B^{k}$ indeed maps C^{\infty}_{c}(\bbR^{n},\Alt^{k})$ to $C^{\infty}_{c}(\bbR^{n},\Alt^{{k}-1})$ will be a the following theorem.

 
 
 






































\section*{Theorem 8.1}

\noindent \textbf{Theorem 8.1 (Discrete Poincaré inequality).} There holds
\begin{align*}
    \|g\|^2_{0,\Omega} 
    \leq 
    C_P |g|^2_{1,T} 
    + 
    \frac{4}{|\Omega|} \left( \int_{\Omega} g(x) \, dx \right)^2
        \quad 
    \forall g \in W(T_h), \forall h > 0
\end{align*}
with
\begin{align*}
C_P = 4C_d C_\Omega \frac{C_{d,T}}{\kappa_T} [\text{diam}(\Omega)]^2 + 8c_d h^2,
\end{align*}
where \(C_\Omega\) is given by (27) when \(\Omega\) is convex and by (28) otherwise, \(C_{d,T}\) is given by (12) when Assumption (B) is satisfied and by (15) in the general case, \(c_d\) is given by (22), and \(C_d\) is given by (26).
\\

\noindent Obviously
\begin{align*}
    \|g\|_{0,\Omega} 
    \leq 
    \|g - I(g)\|_{0,\Omega} + \|I(g)\|^2_{0,\Omega}
    ,
\end{align*}
\begin{align*}
    \|g\|^2_{0,\Omega} 
    &\leq 
    2\|g - I(g)\|^2_{0,\Omega} + 2\|I(g)\|^2_{0,\Omega} 
    ,
    \|g\|_{0,\Omega} 
    &\leq
    \sqrt{2}
    \sqrt{ \|g - I(g)\|^2_{0,\Omega} + 2\|I(g)\|^2_{0,\Omega} }
    .
\end{align*}
We apply the cellwise local approximation estimate, which coincides with the cellwise Poincare inequality:
\begin{align*}
    \|g - I(g)\|_{0,\Omega}
    &\leq 
    \frac{ h_T }{\pi}
    \| \nabla_\calT g \|_{0,\Omega}
    ,
    \|g - I(g)\|_{0,\Omega}
    &\leq 
    \frac{ h_T }{\pi}
    \| \nabla_\calT g \|_{0,\Omega}
    .
\end{align*}



\begin{align}
        v_0 + \langle v_k - v_0, v_{k-1} - v_0, \dots, v_1 - v_0 \rangle,
        \\
        \langle v_n - v_{k+1}, \dots, v_n - v_{n-1} \rangle + v_{0},
        \\
        v_0 + \sum_{i=1}^{k} \lambda_{i} ( v_i - v_0 ),
        \\
        v_0 + \sum_{i=k+1}^{n} \lambda_{i} ( v_n - v_i ),
        \\
        x = v_0 + \sum_{i=1}^{k} \lambda_{i} ( v_i - v_0 ) + \sum_{i=k+1}^{n} \lambda_{i} ( v_n - v_i ),
    \end{align}
    \begin{align}
        v_0 \left( 1 - \sum_{i=1}^{k} \lambda_{i} \right)
        +
        \sum_{i=1}^{k} \lambda_{i} v_i 
        +
        \sum_{i=k+1}^{n-1} \lambda_{i} v_n
        -
        \sum_{i=k+1}^{n-1} \lambda_{i} v_i
    \end{align}
    \begin{align}
        v_0 \left( 1 - \sum_{i=1}^{k} \lambda_{i} \right)
        +
        \sum_{i=1}^{k} \lambda_{i} v_i 
        +
        \sum_{i=k+1}^{n-1} (-\lambda_{i}) v_i
        +
        \sum_{i=k+1}^{n-1} \lambda_{i} v_n
    \end{align}
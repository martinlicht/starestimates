% Develop construction of complexes via adding patches 
\subsection{Patch coverings}

We are interested in exhausting a simplicial complex in a controlled manner. 
Suppose that $\calT$ is a manifold-like $n$-dimensional simplicial complex. 
We call $n$-simplices $S,T \in \calT$ \emph{face-connected} if there exists a sequence $S_0=S,S_1,\dots,T=S_m$ such that $S_{i} \cap S_{i-1}$ is a face for all $1 \leq i \leq m$. Clearly, face-connected is an equivalence relation. A \emph{connected component} of $\calT$ is an equivalence class under the face-connected relation, and we call $\calT$ connected if all its $n$-simplices are face-connected. 
In a manifold-like simplicial complex, two $n$-simplices have non-empty intersection if and only if they are face-connected. 

It is a consequence of Lemma~\ref{lemma:characterizationofmanifoldcomplexes} that,
if $\calT$ is a manifold-like simplicial complex, then $\patch_{\calT}(S)$ is connected for all $S \in \calT$.
Moreover, a manifold-like simplicial complex $\calT$ is connected if and only if $\bigcup\calT$ is a connected topological space. 

We let \emph{$k$-patching} refer to enumerations $S_1, S_2, \dots$ of $\subsimplex_{k}(\calT)$
such that for any $0 \leq m$, the union 
$\calU_{m} := \patch_{\calT}(S_0) \cup \patch_{\calT}(S_1) \cup \patch_{\calT}(S_2) \cup \dots \cup \patch_{\calT}(S_m)$
shares an $n$-simplex with the patch $\patch_{\calT}(S_{m+1})$.
We call the $k$-patching \emph{manifold-like} if $\calU_{m}$ is manifold-like for all $0 \leq m$.
Clearly, a $k$-patching only exists if $k < n$ and if $\calT$ is connected. 
We refer the reader to Figure~\ref{figure:illustrationpatching} for an illustration.

\begin{lemma}
    A connected manifold-like $n$-dimensional simplicial complex $\calT$ admits a $k$-patching for any $k < n$.
\end{lemma}
\begin{proof}
    We define an undirected graph by letting $\subsimplex_{k}(\calT)$ be the set of nodes 
    and connecting any $S, S' \in \subsimplex_{k}(\calT)$ if there exists $T \in \subsimplex_{k+1}(\calT)$ with $S, S' \subseteq T$. 
    Note that in this case $\patch_{\calT}(S) \cap \patch_{\calT}(S') = \patch_{\calT}(T)$. 
    Since $\calT$ is connected, the graph $\calG$ is connected. 
    
    We let $S_0, S_1, S_2, \dots$ be an enumeration of the $k$-simplices generated by a breadth-first traversal of the graph $\calG$,
    starting at some arbitrary but fixed $S_0 \in \calT$. 
    Since $\calG$ is locally finite and connected, 
    this breadth-first traversal includes all $k$-simplices of $\calT$.
    
    Let $m \in \bbZ$ with $m \geq 1$. If $S_m$ has distance $d \in \bbN$ from $S_1$ in the graph $\calG$,
    then there exists $0 \leq l \leq m$ such that $S_l$ has distance $d-1$ from $S_1$ and is connected to $S_m$. 
    It follows that the sequence $S_0, S_1, S_2, \dots$ is a $k$-patching of $\calT$. 
\end{proof}

The preceding lemma is generally not true when the $k$-patching is required to be manifold-like,
as can be verified from some simple examples;~see Figure~\ref{figure:annuluscounterexample}.
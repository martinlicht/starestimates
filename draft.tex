% Develop construction of complexes via adding patches 
\subsection{Patch coverings}

We are interested in exhausting a simplicial complex in a controlled manner. 
Suppose that $\calT$ is a manifold-like $n$-dimensional simplicial complex. 
We call $n$-simplices $S,T \in \calT$ \emph{face-connected} if there exists a sequence $S_0=S,S_1,\dots,T=S_m$ such that $S_{i} \cap S_{i-1}$ is a face for all $1 \leq i \leq m$. Clearly, face-connected is an equivalence relation. A \emph{connected component} of $\calT$ is an equivalence class under the face-connected relation, and we call $\calT$ connected if all its $n$-simplices are face-connected. 
In a manifold-like simplicial complex, two $n$-simplices have non-empty intersection if and only if they are face-connected. 

It is a consequence of Lemma~\ref{lemma:characterizationofmanifoldcomplexes} that,
if $\calT$ is a manifold-like simplicial complex, then $\patch_{\calT}(S)$ is connected for all $S \in \calT$.
Moreover, a manifold-like simplicial complex $\calT$ is connected if and only if $\bigcup\calT$ is a connected topological space. 

We let \emph{$k$-patching} refer to enumerations $S_1, S_2, \dots$ of $\subsimplex_{k}(\calT)$
such that for any $0 \leq m$, the union 
$\calU_{m} := \patch_{\calT}(S_0) \cup \patch_{\calT}(S_1) \cup \patch_{\calT}(S_2) \cup \dots \cup \patch_{\calT}(S_m)$
shares an $n$-simplex with the patch $\patch_{\calT}(S_{m+1})$.
We call the $k$-patching \emph{manifold-like} if $\calU_{m}$ is manifold-like for all $0 \leq m$.
Clearly, a $k$-patching only exists if $k < n$ and if $\calT$ is connected. 
We refer the reader to Figure~\ref{figure:illustrationpatching} for an illustration.

\begin{lemma}
    A connected manifold-like $n$-dimensional simplicial complex $\calT$ admits a $k$-patching for any $k < n$.
\end{lemma}
\begin{proof}
    We define an undirected graph by letting $\subsimplex_{k}(\calT)$ be the set of nodes 
    and connecting any $S, S' \in \subsimplex_{k}(\calT)$ if there exists $T \in \subsimplex_{k+1}(\calT)$ with $S, S' \subseteq T$. 
    Note that in this case $\patch_{\calT}(S) \cap \patch_{\calT}(S') = \patch_{\calT}(T)$. 
    Since $\calT$ is connected, the graph $\calG$ is connected. 
    
    We let $S_0, S_1, S_2, \dots$ be an enumeration of the $k$-simplices generated by a breadth-first traversal of the graph $\calG$,
    starting at some arbitrary but fixed $S_0 \in \calT$. 
    Since $\calG$ is locally finite and connected, 
    this breadth-first traversal includes all $k$-simplices of $\calT$.
    
    Let $m \in \bbZ$ with $m \geq 1$. If $S_m$ has distance $d \in \bbN$ from $S_1$ in the graph $\calG$,
    then there exists $0 \leq l \leq m$ such that $S_l$ has distance $d-1$ from $S_1$ and is connected to $S_m$. 
    It follows that the sequence $S_0, S_1, S_2, \dots$ is a $k$-patching of $\calT$. 
\end{proof}

The preceding lemma is generally not true when the $k$-patching is required to be manifold-like,
as can be verified from some simple examples;~see Figure~\ref{figure:annuluscounterexample}.

\color{Emerald}
\begin{align}
    \min\limits_{ c \in \bbR }
    \| u - c \|_{L^{p}(\Omega)}
    \leq 
    % 2^{ - \frac{p-1}{p}}
    %2^{ - 1 + \frac{1}{p}}
    \frac{1}{ 2^{ 1 - \frac{1}{p} } }
    \diam(\Omega)
    \| \nabla u \|_{L^{p}(\Omega)}
    ,
    \quad 
    u \in W^{1,p}(\Omega)
    ,
\end{align}

\begin{proof}
    Given $z \in \Omega$, we define 
    \[
        w(x,z) 
        := 
        \int_0^1 \bff( z + t(x-z) ) \cdot (x-z)
        =
        u(x) - u(z)
        .
    \]
    Subsequently,
    \begin{align*}
        w(x) = |\Omega|^{-1} \int_{\Omega} w(x,z) dz.
    \end{align*}
    Clearly, $\nabla_{x} u(x) = \nabla_{x} w(x,z) = \nabla_{x} w(x)$.
    Next, 
    \begin{align*}
        \int_{\Omega} | w(x) |^{p} dx
        &\leq 
        \int_{\Omega} |\Omega|^{-1} \int_{\Omega} | w(x,z) |^{p} dz dx
        \\&\leq 
        \int_{\Omega} |\Omega|^{-1} \int_{\Omega} \int_0^1 | \bff( z + t(x-z) ) \cdot (x-z) |^{p} dz dx
        \\&= 
        |\Omega|^{-1} \int_{\Omega} \int_{\Omega} \int_0^{D/|x-z|} | \bff( z + t(x-z) ) \cdot (x-z) |^{p} dz dx
        .
    \end{align*}

\end{proof}
\color{black}

\begin{lemma}
    Suppose that $T, T'$ are two $n$-simplices whose intersection is a common face $F := T_1 \cap T_2$. Then 
    \begin{align*}
        \| u - u_{\Omega} \|_{L^p(\Omega)} 
        &
        \leq 
        \frac{2}{((n-1)!)^n }
        \left( \frac{\diam(T)^n}{\vol(T)} \right)^n
        \left( \frac{\diam(T')^n}{n! \vol(T')} \right)^n
        \diam(\Omega)
        \| \nabla u \|_{L^p(\Omega)} 
    \end{align*}
    for any $u \in W^{1,p}(\Omega)$ with $p \in [1,\infty)$.
\end{lemma}
\begin{proof}
    Suppose a tetrahedron $T$ has vertices $v_0,v_1,\dots,v_n$ and let $F_0,F_1\dots,F_n$ be the faces opposite to those vertices. 
    The height $h(T,k)$ of the vertex $v_k$ in $T$ equals 
    \[
        h(T,k) := n \frac{\vol(T)}{\vol(F_k)}
    \]
    A lower bound for the height in terms of the volume and diameter of $T$ is 
    \[
        h(T,k) 
        = 
        n \frac{\vol(T)}{\diam(T)^n}
        \cdot 
        \frac{\diam(T)^n}{\vol(F_k)}
        \geq 
        n \frac{\vol(T)}{\diam(T)^n}
        \cdot 
        (n-1)! \frac{\diam(T)^n}{\diam(T)^{n-1}}
        \geq 
        \frac{\vol(T)}{\diam(T)^n}
        \cdot 
        n! \diam(T) 
        .
    \]
    Any subsimplex $S$ of $T$ adjacent to $v_k$ thus satisfies 
    \[
        \diam(S)
        \geq 
        \frac{\vol(T)}{\diam(T)^n}
        \cdot 
        n! \diam(T) 
        .
    \]
    We let $w \in F_0$ be the barycenter in the face opposite to $v_0$. 
    Then the convex combination of $w$ and any of the faces $F_1,\dots,F_n$ defines simplices $S_1,\dots,S_n$ which satisfy 
    \[
        \vol(S_1) = \dots = \vol(S_n) = \frac{\vol(T)}{n}
        .
    \]
    Consequently, 
    the height $h_k := $ of the vertex $w$ in $S_k$ equals 
    \[
        h_k := n \frac{\vol(S_k)}{\vol(F_k)} = \frac{\vol(T)}{\vol(F_k)}
    \]
    It follows that $T$ is star-shaped with respect to the intersection of $T$
    with the ball around $w$ of radius $h$, where $h := \min_{1 \leq k \leq n} h_k$.
    A lower bound for height in terms of the volume and diameter of $T$ is 
    \[
        h_k 
        = 
        \frac{\vol(T)}{\diam(T)^n}
        \cdot 
        \frac{\diam(T)^n}{\vol(F_k)}
        \geq 
        \frac{\vol(T)}{\diam(T)^n}
        \cdot 
        (n-1)! \frac{\diam(T)^n}{\diam(T)^{n-1}}
        \geq 
        \frac{\vol(T)}{\diam(T)^n}
        \cdot 
        (n-1)! \diam(T) 
        .
    \]
    Let now $T'$ be another $n$-simplex with vertices $v_0',v_1,\dots,v_n$.
    We construct $h'$ analogous to above and $h'' := \min(h,h')$. 
    Hence $\Omega := T \cup T'$ is star-shaped with respect to a ball around $w$ of radius $h''$.
    Using a result by Farweg and Rosteck,
    \begin{align*}
        \| u - u_{\Omega} \|_{L^p(\Omega)} 
        &
        \leq 
        2 \diam(\Omega)^n
        \omega_n^{ 1 - \frac 1 n }
        \frac{|\Omega|^{\frac{1}{n}}}{|\Ball(w,h'')|}
        \| \nabla u \|_{L^p(\Omega)} 
        \\&
        \leq 
        2 \diam(\Omega)^n
        \omega_n^{ - \frac 1 n }
        \frac{|\Omega|^{\frac{1}{n}}}{(h'')^n}
        \| \nabla u \|_{L^p(\Omega)} 
        .
    \end{align*}
    Without loss of generality, $\diam(T) \leq \diam(T')$. Now, 
    \begin{align*}
        \| u - u_{\Omega} \|_{L^p(\Omega)} 
        &
        \leq 
        \left( \frac{\diam(T)^n}{\vol(T)} \right)^n
        2 \diam(\Omega)^n
        \omega_n^{ - \frac 1 n }
        \frac{|\Omega|^{\frac{1}{n}}}{((n-1)!)^n \diam(T)^n }
        \| \nabla u \|_{L^p(\Omega)} 
        \\&
        \leq 
        \frac{2}{((n-1)!)^n }
        \left( \frac{\diam(T)^n}{\vol(T)} \right)^n
        \frac{ \diam(T')^n }{ \diam(T)^n }
        \diam(T')
        \| \nabla u \|_{L^p(\Omega)} 
        .
    \end{align*}
    We know that 
    \begin{align*}
        \diam(T')
        \leq 
        \frac{\diam(T')^n}{n! \vol(T')}
        \diam(F)
        \leq 
        \frac{\diam(T')^n}{n! \vol(T')}
        \diam(T)
        .
    \end{align*}
    The final estimate is
    \begin{align*}
        \| u - u_{\Omega} \|_{L^p(\Omega)} 
        &
        \leq 
        \frac{2}{((n-1)!)^n }
        \left( \frac{\diam(T)^n}{\vol(T)} \right)^n
        \left( \frac{\diam(T')^n}{n! \vol(T')} \right)^n
        \diam(\Omega)
        \| \nabla u \|_{L^p(\Omega)} 
        .
    \end{align*}
    This finishes the proof. 
\end{proof}


We use the following trace inequality, which can be found in the literature. 

\begin{lemma}\label{lemma:traceinequality}
    Let $p \in [1,\infty)$ and let $T$ be an $n$-dimensional simplex with a face $F$. Then 
    \begin{align*}
        \| u \|_{L^{p}(F)}^{p}
        \leq 
        \frac{|F|}{|T|} 
        \| u \|_{L^{p}(T)}^{p}
        +
        p \cdot \diam(T) 
        \frac{|F|}{|T|} 
        \| u \|_{L^{p}(T)}^{p-1}
        \| \nabla u \|_{\bfL^{p}(T)}
        .
    \end{align*}
    for all $u \in W^{1,p}(T)$. Moreover, 
    \begin{align*}
        \| u \|_{L^{\infty}(F)}
        \leq 
        \| u \|_{L^{\infty}(T)}
        .
    \end{align*}
    for all $u \in W^{1,\infty}(T)$. \qed
\end{lemma}

\begin{remark}
    Lemma~\cite{veeser2012poincare} has been stated by Veeser and Verfürth~\cite[Lemma 2.8]{veeser2012poincare} when $1 \leq p < \infty$. 
    The case $p = \infty$ is obvious because functions in $W^{1,\infty}(T)$ are already Lipschitz. 
    \color{red}\textbf{The following is probably wrong} 
    However, note that for any $p \in [1,\infty)$ and $u \in W^{1,p}(T)$ we have 
    \begin{align*}
        \| u \|_{L^{p}(F)}
        &\leq 
        \| u \|_{L^{p}(T)}^{\frac{p-1}{p}}
        \left(
            \frac{|F|}{|T|}
            \| u \|_{L^{p}(T)}
            +
            p \cdot \diam(T)
            \frac{|F|}{|T|}
            \| \nabla u \|_{\bfL^{p}(T)}
        \right)^{\frac{1}{p}}
        \\&\leq 
        \| u \|_{L^{p}(T)}^{\frac{p-1}{p}}
        \left( \frac{|F|}{|T|} \right)^{\frac 1 p}
        \left(
            \| u \|_{L^{p}(T)}^{\frac{1}{p}}
            +
            \sqrt[p]{p} \cdot \diam(T)^{\frac 1 p} 
            \| \nabla u \|_{\bfL^{p}(T)}^{\frac 1 p} 
        \right)
        .
    \end{align*}
    The limit as $p$ goes to infinity leads to the inequality as stated in the preceding lemma,
    which is therefore the natural limit case of the aforementioned trace inequality.
\end{remark}


\begin{lemma}
    Let $T_1$ and $T_{2}$ be two $n$-simplices such that $F = T_1 \cap T_2$ is a common subsimplex of dimension $n-1$. Then 
    \color{red} Here we estimate the Poincar\'e constant over a face patch.
\end{lemma}

\color{red}
Possible approaches: 
\begin{itemize}
 \item Take the trace jump (requires trace inequality with explicit constant) and extend the resulting constant function
 \item The union is star-shaped with respect to some (hopefully large) convex set, then use inequality for that situation (available in literature)
 \item Reflection trick
 \item Paper by Veeser and Verfuerth?
\end{itemize}
Most importantly, all constants must be explicitly computable!
We can do all these methods and compare the results. 
\color{black}
    
\begin{proof}
    We write $U := T_1 \cup T_2$. Let $u \in W^{1,p}(U)$.
    Suppose $u_1 \in W^{1,p}(T_1)$ and $u_2 \in W^{1,p}(T_2)$
    satisfy $\nabla u_1 = \nabla u_{|T_1}$ and $\nabla u_2 = \nabla u_{|T_2}$.
    Then there exists a constant $c \in \bbR$ such that $c = \trace_{T_1,F} u_1 - \trace_{T_1,F} u_2$.
    Consequently, if we define $u_2' = u_2 + c$, then we obtain a function $u = u_1 \cup u_2' \in W^{1,p}(T)$. 
    Notice that 
    \begin{align*}
        \| u \|_{L^{p}(U)}^{p}
        &\leq 
        \| u_1 \|_{L^{p}(T_1)}^{p}
        +
        \| u'_2 \|_{L^{p}(T_2)}^{p}
        \\
        &\leq 
        \| u_1 \|_{L^{p}(T_1)}^{p}
        +
        \left( 
            \| u_2 \|_{L^{p}(T_2)}
            +
            \left(\frac{|T_2|}{|F|}\right)^{\frac 1 p}
            \| c \|_{L^{p}(F)}
        \right)^{p}
    \end{align*}
    \begin{align*}
        \color{red}
        \| u'_2 \|_{L^{p}(T_2)}
        \leq 
        \| u_2 \|_{L^{p}(T_2)}
        +
        \| c \|_{L^{p}(T_2)}
        = 
        \| u_2 \|_{L^{p}(T_2)}
        +
        \left(\frac{|T_2|}{|F|}\right)^{\frac 1 p}
        \| c \|_{L^{p}(F)}
        .
    \end{align*}
    Recall that 
    \begin{align*}
        \| v \|_{L^{p}(F)}^{p}
        \leq 
        \frac{|F|}{|T|} 
        \| v \|_{L^{p}(T)}^{p}
        +
        p \cdot \diam(T) 
        \frac{|F|}{|T|} 
        \| v \|_{L^{p}(T)}^{p-1}
        \| \nabla v \|_{\bfL^{p}(T)}
        ,
        \quad 
        v \in W^{1,p}(T)
        .
    \end{align*}
    Now 
    \begin{align*}
        \| c \|_{L^{p}(F)}^{p}
        &\leq 
        2^{p-1}
        \| u_1 \|_{L^{p}(F)}^{p}
        +
        2^{p-1}
        \| u_2 \|_{L^{p}(F)}^{p}
    \end{align*}
    We thus proceed with 
    \begin{align*}
        \| u_1 \|_{L^{p}(F)}^{p}
        +
        \| u_2 \|_{L^{p}(F)}^{p}
        &
        \leq 
        \frac{|F|}{|T_1|} 
        \| u_1 \|_{L^{p}(T_1)}^{p}
        +
        p \cdot \diam(T_1) 
        \frac{|F|}{|T_1|} 
        \| u_1 \|_{L^{p}(T_1)}^{p-1}
        \| \nabla u_1 \|_{\bfL^{p}(T_1)}
        \\&\quad\quad
        +
        \frac{|F|}{|T_2|} 
        \| u_2 \|_{L^{p}(T_2)}^{p}
        +
        p \cdot \diam(T_2) 
        \frac{|F|}{|T_2|} 
        \| u_2 \|_{L^{p}(T_2)}^{p-1}
        \| \nabla u_2 \|_{\bfL^{p}(T_2)}
    \end{align*}
    If $p = 1$, then we summarize this as 
    \begin{align*}
        \| u \|_{L^{1}(U)} 
        &\leq 
        \| u_1 \|_{L^{1}(T_1)} 
        +
        \| u'_2 \|_{L^{1}(T_2)} 
        \\
        &\leq 
        \| u_1 \|_{L^{1}(T_1)} 
        +
        \| u_2 \|_{L^{1}(T_2)}
        +
        \left(\frac{|T_2|}{|F|}\right)
        \| c \|_{L^{1}(F)}
        \\
        &\leq 
        \| u_1 \|_{L^{1}(T_1)} 
        +
        \| u_2 \|_{L^{1}(T_2)}
        +
        \left(\frac{|T_2|}{|F|}\right)
        \| u_1 \|_{L^{1}(F)} 
        +
        \left(\frac{|T_2|}{|F|}\right)
        \| u_2 \|_{L^{1}(F)} 
        \\
        &\leq 
        \| u_1 \|_{L^{1}(T_1)} 
        +
        \| u_2 \|_{L^{1}(T_2)}
        \\&\qquad 
        +
        \left(\frac{|T_2|}{|F|}\right)
        \frac{|F|}{|T_1|} 
        \| u_1 \|_{L^{1}(T_1)} 
        +
        \diam(T_1) 
        \left(\frac{|T_2|}{|F|}\right)
        \frac{|F|}{|T_1|} 
        \| \nabla u_1 \|_{\bfL^{1}(T_1)}
        \\&\quad\quad
        +
        \left(\frac{|T_2|}{|F|}\right)
        \frac{|F|}{|T_2|} 
        \| u_2 \|_{L^{1}(T_2)} 
        +
        \diam(T_2) 
        \left(\frac{|T_2|}{|F|}\right)
        \frac{|F|}{|T_2|} 
        \| \nabla u_2 \|_{\bfL^{1}(T_2)}
        \\
        &\leq 
        \| u_1 \|_{L^{1}(T_1)} 
        +
        \| u_2 \|_{L^{1}(T_2)}
        \\&\qquad 
        +
        \frac{|T_2|}{|T_1|} 
        \| u_1 \|_{L^{1}(T_1)} 
        +
        \diam(T_1) 
        \frac{|T_2|}{|T_1|} 
        \| \nabla u_1 \|_{\bfL^{1}(T_1)}
        \\&\quad\quad
        +
        \| u_2 \|_{L^{1}(T_2)} 
        +
        \diam(T_2) 
        \| \nabla u_2 \|_{\bfL^{1}(T_2)}
        \\
        &\leq 
        \left( 1 + \frac{|T_2|}{|T_1|} \right)
        \| u_1 \|_{L^{1}(T_1)} 
        +
        2
        \| u_2 \|_{L^{1}(T_2)}
        +
        \diam(T_1) 
        \frac{|T_2|}{|T_1|} 
        \| \nabla u_1 \|_{\bfL^{1}(T_1)}
        +
        \diam(T_2) 
        \| \nabla u_2 \|_{\bfL^{1}(T_2)}
        .
    \end{align*}
    When $1 < p < \infty$, then Young's inequality implies 
    \begin{align*}
        \| u_1 \|_{L^{p}(F)}^{p}
        +
        \| u_2 \|_{L^{p}(F)}^{p}
        &
        \leq 
        \frac{|F|}{|T_1|} 
        \| u_1 \|_{L^{p}(T_1)}^{p}
        +
        \frac{|F|}{|T_1|} 
        \| u_1 \|_{L^{p}(T_1)}^{q(p-1)}
        +
        \frac p q \cdot \diam(T_1)^{p}
        \frac{|F|}{|T_1|} 
        \| \nabla u_1 \|_{\bfL^{p}(T_1)}^{p}
        \\&\quad\quad
        +
        \frac{|F|}{|T_2|} 
        \| u_2 \|_{L^{p}(T_2)}^{p}
        +
        \frac{|F|}{|T_2|} 
        \| u_2 \|_{L^{p}(T_2)}^{q(p-1)}
        +
        \frac p q \cdot \diam(T_2)^{p} 
        \frac{|F|}{|T_2|} 
        \| \nabla u_2 \|_{\bfL^{p}(T_2)}^{p}
        \\
        &
        \leq 
        \frac{|F|}{|T_1|} 
        \| u_1 \|_{L^{p}(T_1)}^{p}
        +
        \frac{|F|}{|T_1|} 
        \| u_1 \|_{L^{p}(T_1)}^{p}
        +
        (p-1) \cdot \diam(T_1)^{p} 
        \frac{|F|}{|T_1|} 
        \| \nabla u_1 \|_{\bfL^{p}(T_1)}^{p}
        \\&\quad\quad
        +
        \frac{|F|}{|T_2|} 
        \| u_2 \|_{L^{p}(T_2)}^{p}
        +
        \frac{|F|}{|T_2|} 
        \| u_2 \|_{L^{p}(T_2)}^{p}
        +
        (p-1) \cdot \diam(T_2)^{p} 
        \frac{|F|}{|T_2|} 
        \| \nabla u_2 \|_{\bfL^{p}(T_2)}^{p}
        .
    \end{align*}
    \color{red}The last terms vanish in the limit $p \rightarrow 1$. 
    It is not clear whether $p=1$ emerges as a limit case: 
    we have gained an additional $p$-norm of $u$ but applying the Poincare inequality introduces the Poincare constant, 
    whereas there is no such dependence on the Poincare constant in the case $p=1$ above. 
    \color{black}
    In summary, 
    \begin{align*}
        2^{1-p}
        \left(\frac{|T_2|}{|F|}\right)
        \| c \|_{L^{p}(F)}^{p}
        &
        \leq 
        2
        \frac{|T_2|}{|T_1|} 
        \| u_1 \|_{L^{p}(T_1)}^{p}
        +
        (p-1) \cdot \diam(T_1)^{p} 
        \frac{|T_2|}{|T_1|} 
        \| \nabla u_1 \|_{\bfL^{p}(T_1)}^{p}
        \\&\quad\quad\quad\quad
        +
        2
        \| u_2 \|_{L^{p}(T_2)}^{p}
        +
        (p-1) \cdot \diam(T_2)^{p} 
        \| \nabla u_2 \|_{\bfL^{p}(T_2)}^{p}
        .
    \end{align*}
    We thus find 
    %{\tiny
    \begin{align*}
        \| u \|_{L^{p}(U)}^{p}
        &\leq 
        \| u_1 \|_{L^{p}(T_1)}^{p}
        +
        \| u'_2 \|_{L^{p}(T_2)}^{p}
        \\
        &\leq 
        \| u_1 \|_{L^{p}(T_1)}^{p}
        +
        \left( 
            \| u_2 \|_{L^{p}(T_2)}
            +
            \left(\frac{|T_2|}{|F|}\right)^{\frac 1 p}
            \| c \|_{L^{p}(F)}
        \right)^{p}
        \\
        &\leq 
        \| u_1 \|_{L^{p}(T_1)}^{p}
        +
        2^{p-1}
        \| u_2 \|_{L^{p}(T_2)}^{p}
        +
        2^{p-1}
        \left(\frac{|T_2|}{|F|}\right)
        \| c \|_{L^{p}(F)}^{p}
        \\
        % &\leq 
        % \| u_1 \|_{L^{p}(T_1)}^{p}
        % +
        % 2^{p-1}
        % \| u_2 \|_{L^{p}(T_2)}^{p}
        % +
        % 4^{p-1}
        % \left( 
        %     2 \max\left( 
        %         \frac{|T_2|}{|T_1|} 
        %         ,
        %         1
        %     \right)
        %     \| u_1 \cup u_2 \|_{L^{p}(U)}^{p}
        %     +
        %     (p-1) \cdot 
        %     \max\left( 
        %         \diam(T_1)^p \frac{|T_2|}{|T_1|} 
        %         ,
        %         \diam(T_2)^p 
        %         %
        %     \right) 
        %     \| \nabla u_1 \cup \nabla u_2 \|_{\bfL^{p}(U)}^{p}
        % \right)
        % \\
        &\leq 
        \| u_1 \|_{L^{p}(T_1)}^{p}
        +
        2^{p-1}
        \| u_2 \|_{L^{p}(T_2)}^{p}
        \\&\quad\quad\quad\quad
        +
        4^{p-1}
        2
        \frac{|T_2|}{|T_1|} 
        \| u_1 \|_{L^{p}(T_1)}^{p}
        +
        4^{p-1}
        (p-1) \cdot \diam(T_1)^{p} 
        \frac{|T_2|}{|T_1|} 
        \| \nabla u_1 \|_{\bfL^{p}(T_1)}^{p}
        \\&\quad\quad\quad\quad
        +
        4^{p-1}
        2
        \| u_2 \|_{L^{p}(T_2)}^{p}
        +
        4^{p-1}
        (p-1) \cdot \diam(T_2)^{p} 
        \| \nabla u_2 \|_{\bfL^{p}(T_2)}^{p}
    \end{align*}
    %}
    We can now apply the Poincar\'e--Friedrichs inequality over each simplex. Hence,
    \begin{align*}
        \| u \|_{L^{p}(U)}^{p}
        &\leq 
        C_1 D^{p} \| \nabla u_1 \|_{L^{p}(T_1)}^{p}
        +
        2^{p-1}
        C_2 D^{p} \| \nabla u_2 \|_{L^{p}(T_2)}^{p}
        \\&\quad\quad\quad\quad
        +
        4^{p-1}
        2
        Q
        C_1 D^{p} \| \nabla u_1 \|_{L^{p}(T_1)}^{p}
        +
        4^{p-1}
        (p-1) \cdot D^{p} 
        Q
        \| \nabla u_1 \|_{\bfL^{p}(T_1)}^{p}
        \\&\quad\quad\quad\quad
        +
        4^{p-1}
        2
        C_2 D^{p} \| \nabla u_2 \|_{L^{p}(T_2)}^{p}
        +
        4^{p-1}
        (p-1) \cdot D^{p} 
        \| \nabla u_2 \|_{\bfL^{p}(T_2)}^{p}
        \\
        &\leq 
        \left( 
            C_1 
            +
            4^{p-1}
            2
            Q
            C_1 
            +
            4^{p-1}
            (p-1)
            Q
        \right) 
        D^{p}
        \| \nabla u_1 \|_{L^{p}(T_1)}^{p}
        \\&\quad\quad\quad\quad
        +
        \left( 
        2^{p-1}
        C_2 
        +
        4^{p-1}
        2
        C_2 
        +
        4^{p-1}
        (p-1)
        \right) 
        D^{p}
        \| \nabla u_2 \|_{L^{p}(T_2)}^{p}
        \\
        &\leq 
        \left( 
            2^{p-1}
            C  
            +
            4^{p-1}
            2
            Q'
            C_1 
            +
            4^{p-1}
            (p-1)
            Q'
        \right) 
        D^{p} 
        \| \nabla u \|_{L^{p}(U)}^{p}
        .
    \end{align*}
    
    \color{red} Possibly the estimate can be tweaked to obtain nicer constants.
\end{proof}



%  
 % \color{red}We want to summarize the Lp norms so that the Poincare inequality over $U_F$ is applied only once. In the special cases $p=1$ and $p=\infty$, this works as above. Do we have a formula for general $p$ that contains these as limit cases? \color{black}
 % In the special case $p < \infty$, we use the finite-dimensional H\"older inequality to find 
 % \begin{align*}
 %    \| w_{m} \|_{L^{p}(T_m)}^{p}
 %    &\leq 
 %    3^{p-1}
 %    \| u_{F_m} \|_{L^{p}(T_m)}^{p}
 %    +
 %    3^{p-1}
 %    \frac{ \vol(T_m) }{ \vol(T_{j(m)}) }
 %    \| u_{F_{m}} \|_{L^{p}(T_{j(m)})}^{p}
 %    +
 %    3^{p-1}
 %    \frac{ \vol(T_m) }{ \vol(T_{j(m)}) }
 %    \| w_{m-1}\|_{L^{p}(T_{j(m)})}^{p}
 %    \\&\leq 
 %    3^{p-1}
 %    \max\left( 1, \frac{ \vol(T_m) }{ \vol(T_{j(m)}) } \right)
 %    \| u_{F_m} \|_{L^{p}(U_{F_m})}^{p}
 %    +
 %    3^{p-1}
 %    \frac{ \vol(T_m) }{ \vol(T_{j(m)}) }
 %    \| w_{m-1}\|_{L^{p}(T_{j(m)})}^{p}
 %    ,
 % \end{align*}
 % and therefore
 % \begin{align*}
 %    \| w_{m} \|_{L^{p}(T_m)}
 %    &\leq 
 %    3^{\frac{p-1}{p}}
 %    \max\left( 1, \frac{ \vol(T_m) }{ \vol(T_{j(m)}) } \right)^{\frac 1 p}
 %    \| u_{F_m} \|_{L^{p}(U_{F_m})} 
 %    +
 %    3^{\frac{p-1}{p}}
 %    \frac{ \vol(T_m)^{\frac 1 p} }{ \vol(T_{j(m)})^{\frac 1 p} }
 %    \| w_{m-1}\|_{L^{p}(T_{j(m)})} 
 %    .
 % \end{align*}
 % \color{blue}
 % In the special case $p < \infty$, we use the finite-dimensional H\"older inequality to find 
 % \begin{align*}
 %    \| w_{m} \|_{L^{p}(T_m)}^{p}
 %    &\leq 
 %    \left( 
 %        1 + 2 \frac{ \vol(T_m)^{\frac{1}{p-1}} }{ \vol(T_{j(m)})^{\frac{1}{p-1}} }
 %    \right)^{ p-1 }
 %    \left( 
 %        \| u_{F_m} \|_{L^{p}(T_m)}^{p}
 %        +
 %        \| u_{F_{m}} \|_{L^{p}(T_{j(m)})}^{p}
 %        +
 %        \| w_{m-1}\|_{L^{p}(T_{j(m)})}^{p}
 %    \right)
 %    ,
 % \end{align*}
 % and therefore
 % \begin{align*}
 %    \| w_{m} \|_{L^{p}(T_m)}
 %    &\leq 
 %    \left( 
 %        1 + 2 \frac{ \vol(T_m)^{\frac{1}{p-1}} }{ \vol(T_{j(m)})^{\frac{1}{p-1}} }
 %    \right)^{ \frac{p-1}{p} }
 %    \left( 
 %        \| u_{F_m} \|_{L^{p}(T_m)} 
 %        +
 %        \| u_{F_{m}} \|_{L^{p}(T_{j(m)})} 
 %        +
 %        \| w_{m-1}\|_{L^{p}(T_{j(m)})} 
 %    \right)
 %    .
 % \end{align*}
 % \color{black}
 
 
 



we have any of the following two conditions:
\begin{itemize}
 \item 
 there exists $\epsilon > 0$ 
 and a homeomorphism 
 \begin{align*}
    \phi : \Ball_\epsilon(x) \cap \underlying{\calT}  
    \rightarrow 
    \left\{ x \in \bbR^{n} \suchthat |x| \leq 1 \right\}
 \end{align*}
 such that $\phi(x) = 0$,
 \item 
 there exists $\epsilon > 0$ and a homeomorphism 
 \begin{align*}
    \phi : \Ball_\epsilon(x) \cap \underlying{\calT} 
    \rightarrow 
    \left\{ x \in \bbR^{n} \suchthat |x| \leq 1, x_n \leq 0 \right\}
 \end{align*}
 such that $\phi(x) = 0$.
\end{itemize}
This characterizes the union of the simplicial complex $\calT$ as a topological manifold with boundary. 

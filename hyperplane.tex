We begin with the following observation. 
Suppose that $T$ is any $n$-dimensional simplex with some proper subsimplex $F$.
We find a hyperplane of codimension one that intersects $T$ only at the $F$ and such that $T$ is on only one side of the hyperplane. 

\begin{lemma}\label{lemma:oppositesubsimplex}
    Let $v_0, v_1, \dots, v_n \in \bbR^n$ be the vertices of a simplex $T$.
    Let $F, F'$ be the two subsimplices of $T$ with respective vertices $v_0,\dots,v_k$ and $v_{k+1},\dots,v_n$.
    Then
    \begin{gather*}
        A 
        := 
        v_0 
        + 
        \operatorname{lin}\{ v_1 - v_0, \dots, v_k - v_0 \} 
        +  
        \operatorname{lin}\{ v_{k+1} - v_n, \dots, v_{n-1} - v_n \}.
    \end{gather*}
	is an affine subspace of dimension $n-1$, the simplex $T$ lies on one side of $A$,
	and $F = T \cap A$.
\end{lemma}
\begin{proof}
    Obviously, $F \subset A \cap T$. 
    Suppose now that $x \in A \cap T$.
    There exist $\lambda_1,\dots,\lambda_{n-1} \in \bbR$ such that 
    \begin{align*}
        x &= 
        v_0 + \sum_{i=1}^{k} \lambda_{i} ( v_i - v_0 ) + \sum_{i=k+1}^{n-1} \lambda_{i} ( v_n - v_i )
        \\&
        = 
        \left( 1 - \sum_{i=1}^{k} \lambda_{i} \right) v_0 
        + 
        \sum_{i=1  }^{k  } \lambda_{i} v_i 
        + 
        \left( \sum_{i=k+1}^{n-1} \lambda_{i} \right) v_n
        + 
        \sum_{i=k+1}^{n-1} (-\lambda_{i}) v_i 
        .
    \end{align*}
    This is the unique expression of $x$ as non-negative linear combination of the vertices of $T$.
    Hence the $\lambda_{k+1}, \dots, \lambda_{n-1}$ are non-positive and their sum is non-negative.
    Hence $\lambda_{k+1} = \dots = \lambda_{n-1} = 0$, implying $x \in F$.
    % 
    Lastly, the linear spans 
    \begin{gather*}
        A_{0} := \operatorname{lin}\{ v_1 - v_0, \dots, v_k - v_0 \},
        \qquad 
        A_{1} := \operatorname{lin}\{ v_{k+1} - v_n, \dots, v_{n-1} - v_n \}
    \end{gather*}
    intersect only at the origin, since otherwise a non-trivial affine combination of the vertices of $T$ equals zero, 
	contradicting their affine independence. %the vertices of $T$ are not affinely independent. 
    So $A$ has dimension $n-1$. 
\end{proof}
